\section{Proof of lemmas and propositions for biorthogonal schemes}\label{app: lemmas}

\subsection{Discontinuity of $\m{j}$}\label{app: discontinuity}
\begin{lemma}\label{lem: subM-singular-sys}
Define $d_{i,j}(\V{\omega}) = \det([\mrow{k_1}(\V{\omega})^\top,\cdots,\mrow{k_6}(\V{\omega})^\top]),\;$ where $0\leq k_1<\cdots<k_6\leq 7,\, s.t.\, k_l\neq i,j.$
\eqref{eq: LS-new} is solvable $\forall \V{\omega}$ if and only if \vspace{.5em}
\begin{align}
\label{eq: singular-cond}
\mathfrak{D}(\V{\omega})\begin{bmatrix}
\sbarm{0}\\
\sbarmp{0}{2}\\
\sbarmp{0}{4}\\
\sbarmp{0}{6}
\end{bmatrix}
\doteq
\begin{bmatrix}
0 & d_{0,2} & d_{0,4} & d_{0,6}\\
-d_{0,2} & 0 & d_{2,4} & d_{2,6}\\
-d_{0,4} & -d_{2,4} & 0 & d_{4,6}\\
-d_{0,6} & -d_{2,6} & -d_{4,6} & 0
\end{bmatrix}
\begin{bmatrix}
\sbarm{0}\\
\sbarmp{0}{2}\\
\sbarmp{0}{4}\\
\sbarmp{0}{6}
\end{bmatrix}
= \begin{bmatrix}
0\\0\\0\\0
\end{bmatrix}.
\end{align}
\end{lemma}
\noindent{\it Proof.}
By Lemma \ref{lem: subM-singular} and Lemma \ref{lem: M-symmetry},  $\M[\widehat{k},:],\,k=0,2,4,6$ are singular,
The singularity condition on  $\M[\widehat{0},:](\V{\omega})$ can be rewritten as follows,
\begin{align}\label{eq: singular-omega}
0 &=\det(\M[\widehat{0},:]) \notag\\
&=  \sbarmp{0}{2}\cdot\det(\Msub[\widehat{2},:])\notag\\
&\quad+ \,\sbarmp{0}{4}\cdot\det(\Msub[\widehat{4},:])
+ \sbarmp{0}{6}\cdot\det(\Msub[\widehat{6},:])\notag\\
&= 0\cdot\sbarm{0}\,+\,d_{0,2}\cdot\sbarmp{0}{2} \notag\\
&\quad+\,d_{0,4}\cdot \sbarmp{0}{4}\,+\, d_{0,6}\cdot\sbarmp{0}{6}
\end{align}
%This is the first equation in the linear system \eqref{eq: singular-cond}. Substitute $\V{\omega}$ by $\V{\omega + \pi_2}$ in \eqref{eq: singular-omega} and use the $2\pi-$periodicity of $\V{\omega}$, we have the singularity condition on $\M[-1,:](\V{\omega+\pi_2})$ as follows
%then the above singularity condition on $\M[-1,:]$ at $\V{\omega}$ can be rewritten as follows,
%\begin{align*}
%[0,\, d_{0,2}(\V{\omega}),\, d_{0,4}(\V{\omega}),\, d_{0,6}(\V{\omega})]\,[\widetilde{m_0}(\V{\omega}),\,\widetilde{m_0}(\V{\omega}+\V{\pi}_2),\, \widetilde{m_0}(\V{\omega}+\V{\pi}_4),\,\widetilde{m_0}(\V{\omega}+\V{\pi}_6)]^\top = 0
%\end{align*}
%It is easy to verify that the above singular condition at $\V{\omega}+\V{\pi}_2$ is equivalent to 
%\begin{align*}
%-d_{0,2}(\V{\omega})\cdot \widetilde{m_0}(\V{\omega}) + d_{2,4}(\V{\omega})\cdot \widetilde{m_0}(\V{\omega}+\V{\pi}_4) + d_{2,6}(\V{\omega})\cdot\widetilde{m_0}(\V{\omega}+\V{\pi}_6) = 0,
%[-d_{0,2}(\V{\omega}),\, 0,\,d_{2,4}(\V{\omega}),\,d_{2,6}][\widetilde{m_0}(\V{\omega}),\,\widetilde{m_0}(\V{\omega}+\V{\pi}_2),\, \widetilde{m_0}(\V{\omega}+\V{\pi}_4),\,\widetilde{m_0}(\V{\omega}+\V{\pi}_6)]^\top = 0,
%\end{align*}
%which is the second linear equation in  \eqref{eq: singular-cond}.
Similarly, the second to fourth equations can be obtained by rewriting the singularity condition on $\M[\widehat{2},:]$, $\M[\widehat{4},:]$ and $\M[\widehat{6},:]$ respectively.\qed
% at $\V{\omega}+\V{\pi}_4$ and $\V{\omega}+\V{\pi}_6$ in the coordinate of $\V{\omega}$.\qed
%where $\mathfrak{D}(\V{\omega})$ is anti-symmetric. Because $\mathfrak{D}(\V{\omega})$ is independent of $m_0(\V{\omega})$, \eqref{eq: singular-cond} holds for $\mc{0}$ as well.

%On the other hand, given $m_0$, $\widetilde{m_0}$ has to satisfy the identity constraint \eqref{eq: identity-cond}.
The identity constraint \eqref{eq: identity-cond} on $m_0$ and the singularity condition \eqref{eq: singular-cond} together imply the following proposition,
%Due to the periodic wrapping of the frequency square $S_0$, we only need to consider \eqref{eq: singular-cond} and \eqref{eq: identity-cond} on $S_1$ and they imply the following proposition,
\begin{proposition}\label{prop: feasibility}
Given $\widetilde{m_i}, i = 1,\cdots,6$, \eqref{eq: identity-cond} has no solution for $\widetilde{m_0}$, if $\exists\,\V{\omega}, \,s.t. \; [m_0(\V{\omega}), m_0(\V{\omega}+\V{\pi}_2),m_0(\V{\omega}+\V{\pi}_4),m_0(\V{\omega}+\V{\pi}_6)]$ is a linear combination of the rows of $\mathfrak{D}(\V{\omega})$.% in \eqref{eq: singular-cond}.
\end{proposition}

\noindent{\bf Proof of Lemma \ref{lem: rank1}}:\\[.2em]
{\bf Lemma \ref{lem: rank1}}. {\it 
If $\V{\omega}\in S_\rho$ s.t. \eqref{eq: identity-cond} holds and $\M[\widehat{0},:](\V{\omega})$ is singular, then $rank(\,\mrow{1}(\V{\omega}),\mrow{7}(\V{\omega})\,)=1$ or \\$rank(\,\mrow{3}(\V{\omega}\,),\mrow{5}(\V{\omega})) = 1$.
}\\[1em]
\noindent{\it Proof.}
When $\rho$ is small enough, due to the concentration property, $\m{i}$ is zero on all but a few sets $S_\rho + \V{\pi}_j$ (see Fig.\ref{fig: S-shifts} for reference of $S_\rho$ and its shifts), thus $\mrow{i}(\V{\omega})$ is sparse on $S_\rho$ and $\M[:,\widehat{0}]$ takes the following form
\begin{align}
\label{eq: sparse-mat}
\M[:,\widehat{0}](\V{\omega})=
\begin{bmatrix}
\mrow{0}\\
\mrow{1}\\
\mrow{2}\\
\mrow{3}\\
\mrow{4}\\
\mrow{5}\\
\mrow{6}\\
\mrow{7}
\end{bmatrix}
=
\begin{bmatrix}
0 & 0 & 0 & 0 & 0 & 0\\
* & 0 & 0 & 0 & 0 & *\\
0 & 0 & 0 & * & * & 0\\
0 & 0 & * & * & 0 & 0\\
0 & * & * & 0 & 0 & 0\\
0 & 0 & * & * & 0 & 0\\
* & 0 & * & * & 0 & *\\
%0 & * & * & 0 & 0 & 0\\
%0 & 0 & 0 & * & * & 0\\
%* & 0 & 0 & 0 & 0 & *\\
%* & 0 & 0 & 0 & 0 & *\\
%0 & 0 & * & * & 0 & 0\\
%0 & 0 & * & * & 0 & 0\\
* & 0 & 0 & 0 & 0 & *
\end{bmatrix}
%=\V{P}\,\widetilde{\mathbf{M}}[:,2:7],
\end{align}
where $*$ denote possible non-zero entries.
%where $\V{P}$ is a row permutation matrix. 
We make the following observation of $\mrow{i}$:
\begin{itemize}
\item[(i)] $\mrow{0}$ is a zero vector
\item[(ii)] $\mrow{2}$ and $\mrow{4}$ are linearly independent of each other and the rest of $\mrow{i}$
\item[(iii)] $span\{\mrow{1},\mrow{7}\} \perp span\{\mrow{3},\mrow{5}\}$ and $rank(\mrow{1},\mrow{7}) \leq 2$, \\$rank(\mrow{3},\mrow{5})\leq 2$
\item[(iv)] $span\{\mrow{1}, \mrow{7}, \mrow{3},\mrow{5},\mrow{6}\} \leq 4$
\end{itemize}
Since $m_0(\V{\omega})\neq 0$ on $S_\rho$, \eqref{eq: m0-cramer} then implies that $\det(\Msub[\widehat{k_{\V{\omega}}},:])\neq 0$. Therefore, $\Msub$ is full rank, or equivalently, $rank(\M[:,\widehat{0}]) = 6$. It follows from  (ii) and (iv) that $rank(\mrow{1},\mrow{6},\mrow{7},\mrow{3},\mrow{5})= 4$.\\
On the other hand, (ii) and (iv) imply that $$rank(\Msub(\V{\omega}+\V{\pi}_2))=rank(\mrow{0},\mrow{4},\mrow{6},\mrow{1},\mrow{3},\mrow{5},\mrow{7})= 5$$ and likewise $$rank(\Msub(\V{\omega}+\V{\pi}_4))=rank(\mrow{0},\mrow{2},\mrow{6},\mrow{1},\mrow{3},\mrow{5},\mrow{7})= 5.$$ Therefore, $\det(\Msub(\V{\omega} + \V{\pi}_2)) = \det(\Msub(\V{\omega} + \V{\pi}_4)) = 0$ and \eqref{eq: m0-cramer} implies $m_0(\V{\omega}+\V{\pi}_2) = m_0(\V{\omega}+\V{\pi}_4) = 0$.\\
If $\mrow{1}$ and $\mrow{7}$ are linearly independent and so are $\mrow{3}$ and $\mrow{5}$, then $$rank(\Msub(\V{\omega}+\V{\pi}_6))=rank(\mrow{2},\mrow{4},\mrow{1},\mrow{3},\mrow{5},\mrow{7}) = 6,$$ hence $m_0(\V{\omega}+\V{\pi}_6)\neq 0$. Therefore, $$[m_0(\V{\omega}),m_0(\V{\omega}+\V{\pi}_2),m_0(\V{\omega}+\V{\pi}_4),m_0(\V{\omega}+\V{\pi}_6)] = [*,0,0,*].$$ In addition, $d_{i,j} = 0,\, \forall(i,j)$ except $(0,6)$, so in \eqref{eq: singular-cond} $$\mathfrak{D}(\V{\omega}) = [d_{0,6}, 0, 0,0]^\top [0,0,0,1] + [0,0,0,d_{0,6}]^\top [-1,0,0,0].$$  By Proposition \ref{prop: feasibility}, \eqref{eq: identity-cond} cannot be satisfied and this proofs the lemma.\qed

\begin{lemma}\label{lem: full-rank-m35}
Let $\widetilde{S_\rho} = S_\rho\cap\{\V{\omega}:\, rank(\,\mrow{3}(\V{\omega}),\,\mrow{5}(\V{\omega})\,)= 1\}$, if $\m{3}$ and $\m{4}$ concentrate in $T_3$ and $T_4$ respectively, then $|\widetilde{S_\rho}|=0$.
\end{lemma}
\noindent{\it Proof.}
Let $\widetilde{S_\rho} + \V{\pi}_3 = \{\V{\omega} + \V{\pi}_3,\, \V{\omega}\in \widetilde{S_\rho}\}$ and $\Omega'$ be the set symmetric to a set $\Omega\subset S_0$ with respect to the diagonal $\omega_1 = -\omega_2$. If $|\widetilde{S_\rho}|>0$, by the concentration of $\m{3}$ in $T_3$, $\forall\, \Omega\subset \widetilde{S_\rho} + \V{\pi}_3\subset T_3$ s.t. $|\Omega|> 0$, $\int_{\Omega}|\widetilde{m_3}| > \int_{\Omega'}|\widetilde{m_3}|$. Due to the symmetry between $|\widetilde{m_3}|$ and $|\widetilde{m_4}|$ defined in \eqref{eq: sym-m34}, $\int_{\Omega'}|\widetilde{m_3}| = \int_{\Omega}|\widetilde{m_4}|$. Therefore, $\int_{\Omega}|\widetilde{m_3}| >  \int_\Omega|\widetilde{m_4}|$ which implies that $|\m{3}| > |\m{4}|\; a.e.$ on $\widetilde{S_\rho}+\V{\pi}_3$ or equivalently $|\widetilde{m_3}(\V{\omega}+\V{\pi}_3)| > |\widetilde{m_4}(\V{\omega}+\V{\pi}_3)|\; a.e.$ on $\widetilde{S_\rho}$. Similarly, we have $|\widetilde{m_4}(\V{\omega}+\V{\pi}_5)| > |\widetilde{m_3}(\V{\omega}+\V{\pi}_5)|\; a.e.$ on $\widetilde{S_\rho}$ following the same analysis on $\widetilde{S_\rho}+\V{\pi}_5\subset T_4$.
On the other hand, $rank(\,\mrow{3}(\V{\omega}),\,\mrow{5}(\V{\omega})\,)= 1$ on $\widetilde{S_\rho}$, hence $\widetilde{m_3}(\V{\omega}+\V{\pi}_3)\widetilde{m_4}(\V{\omega}+\V{\pi}_5) = \widetilde{m_3}(\V{\omega}+\V{\pi}_5)\widetilde{m_4}(\V{\omega}+\V{\pi}_3)$, which contradicts the previous two inequalities.\qed

\begin{lemma}\label{lem: concentrate}
If $\m{1} $ {\rm (}respectively, $\m{6}${\rm)} concentrates in $T_1$ {\rm(}respectively, $T_6${\rm)}, then $|\m{6}| > |\m{1}|\,$ a.e. on $T_6\bigcap \text{supp}(\widetilde{m_6})$ {\rm (}respectively, $|\m{1}| > |\m{6}|$ 
a.e. on $T_1\bigcap\text{supp}(\widetilde{m_1})${\rm )}.
\end{lemma}
\noindent{\it Proof.}
Let $B_6=\{\V{\omega}: |\m{6}| \leq |\m{1}|\}\bigcap T_6\bigcap supp(\widetilde{m_1})$ and $B_1$ be the set symmetric to $B_6$ with respect to $\omega_1 = \omega_2$ and suppose $|B_6|>0$, then $\int_{B_6}|\m{6}|\leq \int_{B_6}|\m{1}|$. On the other hand, since $\m{1}$ concentrates in $T_1$, we know $\int_{B_1}|\m{1}| > \int_{B_6}|\m{1}|$. Moreover, due to the symmetry of $\m{1},\m{6}$ and $B_1,B_6$, $\int_{B_1}|\m{1}| = \int_{B_6}|\m{6}|$, hence $\int_{B_6}|\m{1}| \geq\int_{B_6}|\m{6}| = \int_{B_1}|\m{1}| $ which results in contradiction.\qed

\begin{proposition}\label{prop: zero-corner}
If  $\m{1}$ {\rm(}respectively, $\m{6}${\rm)} concentrates in $T_1$ {\rm(}respectively, $T_6${\rm)}, then $\m{1} = \m{6} = 0,\, a.e. $ on $ S_\rho + \V{\pi}_1$.
\end{proposition}
\noindent{\it Proof.}
Consider frequency domain $S_\rho' = S_\rho\bigcap\{\omega_1<\omega_2\}.$ By Lemma \ref{lem: rank1}, $\exists\,\alpha_{\V{\omega}}\in\mathbb{C}, s.t.\,\mrow{1}(\V{\omega}) = \alpha_{\V{\omega}}\,\mrow{7}(\V{\omega}),\,$ $\forall\, \V{\omega}\in S_\rho',$ i.e. $\widetilde{m_1}(\V{\omega} + \V{\pi}_1) = \alpha_{\V{\omega}}\cdot\widetilde{m_1}(\V{\omega} + \V{\pi}_7)$ and $\widetilde{m_6}(\V{\omega} + \V{\pi}_1) = \alpha_{\V{\omega}}\cdot\widetilde{m_6}(\V{\omega} + \V{\pi}_7)$. On the other hand, Lemma \ref{lem: concentrate} implies that $|\widetilde{m_1}(\V{\omega} + \V{\pi}_7)| \geq |\widetilde{m_6}(\V{\omega} + \V{\pi}_7)|$, hence $|\widetilde{m_1}(\V{\omega} + \V{\pi}_1)| \geq |\widetilde{m_6}(\V{\omega} + \V{\pi}_1)|$. Let $\Omega_6'\doteq (S_\rho+\V{\pi}_1)\bigcap T_6$, then $\int_{\Omega_6'}|\m{1}| \geq\int_{\Omega_6'}|\m{6}|$, which will contradict Lemma \ref{lem: concentrate} unless $|\Omega_6'\bigcap\text{supp}(\widetilde{m_6})| = 0$, or equivalently $\alpha_{\V{\omega}}=0$ and so $\m{6} = \m{1} = 0,\,a.e.$ on $\Omega_6'$. By symmetry, $\m{6}=\m{1} = 0,\,a.e. $ on $(S_\rho+\V{\pi}_1)\setminus \Omega_6'$ as well.\qed

\subsection{Design of input $\m{j}$}\label{app: input design}
{\bf Proof of Lemma \ref{lem: phase-ineq}}:\\[.2em]
{\bf Lemma \ref{lem: phase-ineq}.} 
{\it If $\exists\,\V{\omega}\in D_1:=\{\omega_1=\omega_2,\,\omega_1\in(-\frac{\pi}{2},0)\},\,s.t. \,|m_0(\V{\omega})|\neq 0,$ then $(\V{\eta}_1-\V{\eta}_6)^\top (\V{\pi}_6-\V{\pi}_7)\neq 0(\text{mod}\,2\pi)$. 
}\\ [1em]
\noindent {\it Proof.}
As $\m{1}$ and $\m{6}$ concentrate in $T_1$ and $T_6$ respectively, $\widetilde{m_1}(\V{\omega} + \V{\pi}_i) = 0$ and  $\widetilde{m_6}(\V{\omega} + \V{\pi}_i) = 0$, $i = 1,\cdots, 5$. Due to symmetry, $|\widetilde{m_1}(\V{\omega})| = |\widetilde{m_6}(\V{\omega})|$ on $\{\omega_1=\omega_2\}$. Let $A = |\widetilde{m_1}(\V{\omega}+\V{\pi}_7)| = |\widetilde{m_6}(\V{\omega}+\V{\pi}_7)|$ and $B=|\widetilde{m_1}(\V{\omega}+\V{\pi}_6)| = |\widetilde{m_6}(\V{\omega}+\V{\pi}_6)|$, then the first and the last columns of $\Msub$ are
  \begin{align*}
  \Msub[:,1] = 
 \begin{bmatrix}
 0\\
 \vdots\\
 0\\
 Ae^{i\V{\eta}_1^\top(\V{\omega}+\V{\pi}_6)}\\
 Be^{i\V{\eta}_1^\top(\V{\omega}+\V{\pi}_7)}
 \end{bmatrix}
 \quad\text{and}\quad
  \Msub[:,6] = 
 \begin{bmatrix}
 0\\
 \vdots\\
 0\\
 Ae^{i\V{\eta}_6^\top(\V{\omega}+\V{\pi}_6)}\\
 Be^{i\V{\eta}_6^\top(\V{\omega}+\V{\pi}_7)}
 \end{bmatrix} .
\end{align*}   
By \eqref{eq: m0-cramer}, if $m_0(\V{\omega})>0, \,\V{\omega}\in D_1$ then $\Msub(\V{\omega})$ is full rank, hence its columns are linearly independent.
In particular, $\Msub[:,1]$ and $\Msub[:,6]$ are linearly independent, which implies that $e^{i(\V{\eta}_1^\top\V{\pi}_6 + \V{\eta}_6^\top\V{\pi}_7)}\neq e^{i(\V{\eta}_6^\top\V{\pi}_6 + \V{\eta}_1^\top\V{\pi}_7)}$%$e^{i(\V{\eta}_1-\V{\eta}_6)^\top(\V{\omega}+\V{\pi}_6)}\neq e^{i(\V{\eta}_1-\V{\eta}_6)^\top(\V{\omega}+\V{\pi}_7)}$ 
 or equivalently $(\V{\eta}_1-\V{\eta}_6)^\top(\V{\pi}_6-\V{\pi}_7)\neq 0(\text{mod}2\pi)$. \qed\\

\noindent{\bf Proof of Proposition \ref{prop: origin-det}}\\[.2em]
\noindent{\bf Proposition \ref{prop: origin-det}.} 
{\it If $\widetilde{m_0}(\V{0})\neq 0,$ then $\V{\pi}_1^\top(\V{\eta}_1-\V{\eta}_6)\neq \pi(\text{mod}\,2\pi)$ or $\V{\pi}_3^\top(\V{\eta}_3-\V{\eta}_4)\neq \pi(\text{mod}\,2\pi)$. }\\[1em]
\noindent{\it Proof.}
%$\Msub(\V{0})$ takes the following form
%$$\begin{bmatrix}
%* & 0 & 0 & 0 & 0 & *\\
%0 & * & 0 & 0 & 0 & 0\\
%0 & 0 & * & * & 0 & 0\\
%0 & 0 & 0 & 0 & * & 0\\
%0 & 0 & * & * & 0 & 0\\
%* & 0 & * & * & 0 & *\\
%* & 0 & 0 & 0 & 0 & *
%\end{bmatrix}$$
%The second and the fifth columns of $\Msub$ have single non-zero entry, $\widetilde{m_2}(\V{\pi}_2)$ and $\widetilde{m_5}(\V{\pi}_4)$ respectively, and are orthogonal to all the rest columns, hence the full-rank constraint of $\Msub$ is reduced to the full-rank constraint on its sub-matrix (with permutation of rows and columns)
 Since $\widetilde{m_0}(\V{0})\neq 0$, as shown in Lemma \ref{lem: rank1}, at $\V{\omega} = \V{0}$ $rank(\mrow{1},\mrow{6},\mrow{7},\mrow{3},\mrow{5})= 4$ . This is equivalent to the matrix $\V{A}$ defined in \eqref{eq: matrix-B} to be full rank.
\begin{align}\label{eq: matrix-B}
%\mbox{\V{A}\strut}=
\V{A} = 
\begin{bmatrix}
& & & \\[-1em]
\widetilde{m_1}(\V{\pi}_6) & \widetilde{m_6}(\V{\pi}_6) & \widetilde{m_3}(\V{\pi}_6) & \widetilde{m_4}(\V{\pi}_6) \\
\widetilde{m_1}(\V{\pi}_1) & \widetilde{m_6}(\V{\pi}_1) & 0 & 0\\
\widetilde{m_1}(\V{\pi}_7) & \widetilde{m_6}(\V{\pi}_7) & 0 & 0\\
0 & 0 & \widetilde{m_3}(\V{\pi}_3) & \widetilde{m_4}(\V{\pi}_3)\\
0 & 0 & \widetilde{m_3}(\V{\pi}_5) & \widetilde{m_4}(\V{\pi}_5)\\
\end{bmatrix}
\end{align}
Let $|\widetilde{m_1}(\V{\pi}_1)| = a, \, |\widetilde{m_1}(\V{\pi}_6)|=b$. Due to the symmetry of $\m{j}$,
$|\widetilde{m_1}(\V{\pi}_1)| = |\widetilde{m_1}(\V{\pi}_7)| = |\widetilde{m_6}(\V{\pi}_1)| = |\widetilde{m_6}(\V{\pi}_7)| = |\widetilde{m_3}(\V{\pi}_3)| = |\widetilde{m_3}(\V{\pi}_5)| = |\widetilde{m_4}(\V{\pi}_3)| = |\widetilde{m_4}(\V{\pi}_5)|$ and $|\widetilde{m_1}(\V{\pi}_6)|=| \widetilde{m_6}(\V{\pi}_6)|= | \widetilde{m_3}(\V{\pi}_6)|=| \widetilde{m_4}(\V{\pi}_6)|$. Rewrite $\V{A}$ as follows,
$$\V{A}=
\begin{bmatrix}
b e^{-i\V{\pi}_6^\top\V{\eta}_1} & b e^{-i\V{\pi}_6^\top\V{\eta}_6} & b e^{-i\V{\pi}_6^\top\V{\eta}_3} & b e^{-i\V{\pi}_6^\top\V{\eta}_4}\\
a e^{-i\V{\pi}_1^\top\V{\eta}_1} & a e^{-i\V{\pi}_1^\top\V{\eta}_6} & 0						& 0 \\
a e^{i\V{\pi}_1^\top\V{\eta}_1} & a e^{i\V{\pi}_1^\top\V{\eta}_6} & 0						& 0 \\
0 					& 0 					& a e^{-i\V{\pi}_3^\top\V{\eta}_3} & a e^{-i\V{\pi}_3^\top\V{\eta}_4}\\
0 					& 0 					& a e^{i\V{\pi}_3^\top\V{\eta}_3} & a e^{i\V{\pi}_3^\top\V{\eta}_4}\\
\end{bmatrix}
$$
The product of singular values of $\V{A}$ is 
\begin{align}\label{eq: detB}
\sqrt{\text{det}(\V{A}^* \V{A})} = 4a^3\sqrt{a^2 K_1^2K_2^2 + b^2(Q_1K_2^2 + Q_2K_1^2)},
\end{align}
where $ Q_1 = 1 - \cos(\V{\pi}_6^\top(\V{\eta}_1-\V{\eta}_6))\cos(\V{\pi}_1^\top(\V{\eta}_1-\V{\eta}_6)), Q_2 = 1 - \cos(\V{\pi}_6^\top(\V{\eta}_3-\V{\eta}_4))\cos(\V{\pi}_3^\top(\V{\eta}_3-\V{\eta}_4)), K_1 = \sin(\V{\pi}_1^\top(\V{\eta}_1-\V{\eta}_6)), K_2 = \sin(\V{\pi}_3^\top(\V{\eta}_3-\V{\eta}_4)).$ If $\V{\pi}_1^\top(\V{\eta}_1-\V{\eta}_6) = \V{\pi}_3^\top(\V{\eta}_3-\V{\eta}_4) = \pi (mod\, 2\pi)$, then $K_1 = K_2 = 0$ and $\V{A}$ becomes singular.\qed
\subsection{Solving \eqref{eq: LS-new} and \eqref{eq: identity-cond} for $m_0,\widetilde{m_0}$ and $m_j$}\label{app: solving}
\begin{lemma}\label{lem: null-space}
Let $\V{P}\in\mathbb{C}^{n\times n}$ be a projection matrix of rank $2$ and $\V{a},\V{b},\V{a}',\V{b}'\in\mathbb{C}^n,\, s.t.\, \V{a}^*\V{b} = (\V{a}')^*\V{b}'=1,\, \V{a}'^*\V{b} = \V{a}^*\V{b}' = \V{b}^*\V{b}' = 0.$ If $\V{P}(\V{I}_n - \V{a}\otimes\V{b} - \V{a}'\otimes\V{b}' ) = \V{0}$, then $\V{P}$ is the projection of $span\{\V{b},\V{b}'\}$.
\end{lemma}
\noindent{\it Proof.}
Since $$rank(\V{I}_n) \leq rank(\V{I}_n-\V{a}\otimes\V{b}-\V{a}'\otimes\V{b}') + rank(\V{a}\otimes\V{b}) + rank(\V{a}'\otimes\V{b}'),$$
it follows that $rank(\V{I}_n-\V{a}\otimes\V{b}-\V{a}'\otimes\V{b}')\geq n - 2$. On the other hand, because $rank(\V{P}) = 2$, $\V{P}(\V{I}_n - \V{a}\otimes\V{b} - \V{a}'\otimes\V{b}' ) = \V{0}$ implies that $rank(\V{I}_n-\V{a}\otimes\V{b}-\V{a}'\otimes\V{b}')\leq n - 2$. Hence $rank(\V{I}_n-\V{a}\otimes\V{b}-\V{a}'\otimes\V{b}') = n - 2$ and $\V{P}$ is the projection of $col(\V{I}_n-\V{a}\otimes\V{b}-\V{a}'\otimes\V{b}')^\bot$. On the other hand,
\begin{align*}
\V{b}^*(\V{I}_n-\V{a}\otimes\V{b}-\V{a}'\otimes\V{b}')
&= \V{b}^* - (\V{b}^*\V{a})\V{b}^* - (\V{b}^*\V{a}')(\V{b}')^*\\
&= \V{b}^* - \V{b}^* - 0\cdot (\V{b}')^* = \V{0}^*.
\end{align*}
Therefore, $\V{P}\V{b} = \V{b}$.
Similarly, $(\V{b}')^* (\V{I}_n-\V{a}\otimes\V{b}-\V{a}'\otimes\V{b}')=\V{0}^*$ and $\V{P}\V{b}' = \V{b}'$. Moreover, as $\V{b}^*\V{b}' = 0$ and $rank(\V{P}) = 2$, $\V{P} = \Vert\V{b}\Vert^{-2}\cdot\V{b}\otimes\V{b} + \Vert\V{b}'\Vert^{-2}\cdot\V{b}'\otimes\V{b}'.$\qed

\begin{lemma}
Given $\M[:,\widehat{0}](\V{\omega})$ is full rank $\forall \V{\omega}$, $\M[\widehat{0},:](\V{\omega})$ is singular if \eqref{eq: identity-cond} holds.
\end{lemma}
\noindent{\it Proof. }
If \eqref{eq: identity-cond} holds, then by Lemma \ref{lem: null-space}, $\meven,\,\modd$ are orthogonal to \\$col(\M[:,\widehat{0}])$, therefore $\big[\,\modd,\meven,\M[:,\widehat{0}]\,\big]\in\mathbb{C}^{8\times 8}$ is full rank. Due to \eqref{eq: identity-cond}, $\meven$ and $\mteven$ are not orthogonal to each other, hence $\big[\,\modd,\mteven,\M[:,\widehat{0}]\,\big] = \big[\,\modd, \M\,\big]$ is full rank as well. Because $(\modd)^*\M[:, i] = 0,\,i= 0,\cdots,7$ and $\modd[\widehat{0}]^*\M[\widehat{0}, i] = (\modd)^*\M[:,i]$, $\modd[\widehat{0}]$ is orthogonal to $col(\M[\widehat{0},:])$. Since $\big[ \modd[\widehat{0}], \M[\widehat{0},:]\,\big]\in\mathbb{C}^{7\times 8}$ is full rank, $\M[\widehat{0},:]$ must be singular.\qed\\[1em]

\noindent{\bf Proof of Proposition \ref{prop: m0_formula}}:\\[.2em]
\noindent{\bf Proposition \ref{prop: m0_formula}.} 
{\it Let $\M[odd,\widehat{0}](\V{\omega}),\M[even,\widehat{0}](\V{\omega})\in\mathbb{C}^{4\times6}$ be the submatrices of $\M[:,\widehat{0}](\V{\omega})$ consisting of odd and even indexed rows respectively. $\forall\V{\omega}\in S_0$, suppose {\rm\ref{cond: i}} and \eqref{eq: identity-cond} are satisfied, then {\rm\ref{cond: ii}} holds if and only if $rank(\,\M[odd,\widehat{0}](\V{\omega})\,) = rank(\,\M[even,\widehat{0}](\V{\omega})\,) = 3$ and 
\begin{align}%\label{eq: m0-even-null}
[m_0(\V{\omega}),m_0(\V{\omega}+\V{\pi}_2), m_0(\V{\omega} +\V{\pi}_4), m_0(\V{\omega}+\V{\pi}_6)]\, \M[even,\widehat{0}](\V{\omega}) = \V{0}, \tag{\ref{eq: m0-even-null}}
\end{align}
\begin{align}%\label{eq: m0-odd-null}
[m_0(\V{\omega}+\V{\pi}_1),m_0(\V{\omega}+\V{\pi}_3), m_0(\V{\omega} +\V{\pi}_5), m_0(\V{\omega}+\V{\pi}_7)] \,\M[odd,\widehat{0}](\V{\omega}) = \V{0}. \tag{\ref{eq: m0-odd-null}}
\end{align}
}
\\[.5em]
\noindent{\it Proof.}
Note that $\M[:,\widehat{0}]$ have the same rows at $\V{\omega} + \V{\pi}_i,\,i = 0,\cdots,7$, we define row permutation matrix $\V{P}_i,\; s.t.\,$ $\V{P}_i\big(\M[:,\widehat{0}](\V{\omega} + \V{\pi}_i)\big) = \M[:,\widehat{0}](\V{\omega}). $ Let $\V{P}_{\M}(\V{\omega})$ be the projection matrix of the $col\big(\M[:,\widehat{0}](\V{\omega})\big)^\bot = null(\M[:,\widehat{0}]^*)$, then \ref{cond: ii} is equivalent to $\V{P}_{\M}\V{b}_0'(\V{\omega}) = \V{0}.$ Group this equality at $\V{\omega}+\V{\pi}_i$, we have 
\begin{align}\label{eq: group-proj}
\V{0} & = [\V{P}_i\V{P}_{\M}\V{b}_0'(\V{\omega} + \V{\pi}_i) ]_{i=0,\cdots,7}\notag\\
&= [\V{P}_i\V{P}_{\M}(\V{\omega}+\V{\pi}_i)\V{P}_i^2\V{b}_0'(\V{\omega}+\V{\pi}_i) ]_{i = 0,\cdots,7}\notag\\
&= [\V{P}_{\M}(\V{\omega})\V{P}_i\V{b}_0'(\V{\omega}+\V{\pi}_i)]_{i = 0,\cdots,7}\notag\\
&= \V{P}_{\M}(\V{\omega})[\V{P}_i\V{b}_0'(\V{\omega}+\V{\pi}_i)]_{i = 0,\cdots,7}
\end{align}
Let 
\begin{align*}
\mteven&= [ (1 + i \bmod 2)\cdot\,\sbarmp{0}{i}]_{i=0,\cdots,7}^\top = \M[:,0](\V{\omega}),\\
\mtodd&= [ (i \bmod 2)\cdot\,\sbarmp{0}{i}]_{i=0,\cdots,7}^\top,\\
\meven&= [ (1 + i \bmod 2)\cdot\,m_0(\V{\omega} + \V{\pi}_i)]_{i=0,\cdots,7}^\top,\\
\modd&= [ ( i \bmod 2)\cdot\,m_0(\V{\omega} + \V{\pi}_i)]_{i=0,\cdots,7}^\top.
\end{align*}
The identity constraint \eqref{eq: identity-cond} thus can be written as $(\overlinespace{\meven})^*\,\mteven = 1$ and $(\overlinespace{\modd})^*\,\mtodd = 1$. By definition, 
$$ \V{P}_i\V{b}_0'(\V{\omega}+\V{\pi}_i) = \V{P}_i\big(\V{b}_0 - m_0\M[:,0](\V{\omega}+\V{\pi}_i)\big)  = \V{b}_i - m_0(\V{\omega}+\V{\pi}_i)\V{P}_i\big(\M[:,0](\V{\omega}+\V{\pi}_i)\big)$$
and 
$$\V{P}_i\big(\M[:,0](\V{\omega}+\V{\pi}_i)\big) = 
\begin{cases}
   \M[:,0] = \mteven, &  i \text{ is even}\\[.2em]
    \mtodd,              & i \text{ is odd}
\end{cases}
$$
Substitute the above expression of $ \V{P}_i\V{b}_0'(\V{\omega}+\V{\pi}_i)$ in \eqref{eq: group-proj} and we have
\begin{align}
\V{0}  
%= \V{P}_{\M}(\omega)([\V{b}_i]_{i=0,\cdots,7} - \mteven\otimes\meven - \mtodd\otimes\modd) 
= \V{P}_{\M}(\V{I}_8 -  \mteven\otimes\overline{\meven} - \mtodd\otimes\overline{\modd})
\end{align}
Therefore, by Lemma \ref{lem: null-space}, $\V{P}_{\M}$ is the projection of $span\{\overlinespace{\modd},\overlinespace{\meven}\}$. This is equivalent to \eqref{eq: m0-even-null} and \eqref{eq: m0-odd-null}.
Finally, since $$6 = rank(\M[:,\widehat{0}]) \leq rank(\M[odd,\widehat{0}]) + rank(\M[even,\widehat{0}]) \leq (4-1) + (4-1),$$ $rank(\M[odd,\widehat{0}]) = rank(\M[even,\widehat{0}]) = 3$.\qed
