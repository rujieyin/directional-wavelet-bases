\section{Introduction}

\iffalse
\begin{itemize}
\item review construction of directional wavelet, shearlet
\item what's new in our construction?
\item summary: framework, technique, reference(Durand, Cohen)
\item organization of the paper
\end{itemize}
\fi

% introducing directional wavelets
In image compression and analysis, 2D tensor wavelet schemes are widely used. Despite the time-frequency localization inherited from 1D wavelet, 2D tensor wavelets suffer from poor orientation selectivity: only horizontal or vertical edges are well represented by tensor wavelets. To obtain better representation of 2D images, several directional wavelet schemes have been proposed and applied to image processing, such as directional wavelet filterbanks(DFB) and various extensions.

% summary
Conventional DFB \cite{DFB92} divides the frequency domain into eight equi-angular pairs of triangles; such schemes can be critically downsampled (maximally decimated) with perfect reconstruction (PR), but they typically do not have a multi-resolution structure. 
Different approaches have been proposed to generalize DFB to multi-resolution systems, including non-uniform DFB (nuDFB), contourlets, curvelets, shearlets and dual-tree wavelets.
nuDFB is introduced in \cite{nuDFB05} based on multi-resolution analysis (MRA), where at each level of decomposition the square frequency domain is divided into a high frequency outer ring and a central low frequency domain. For nuDFB, the high frequency ring is primarily divided further into six equi-angular pairs of trapezoids and the central low frequency square is kept intact for division in the next level of decomposition, see Fig. \ref{fig: partition}. The nuDFB filters are solved by optimization which provides a non-unique near orthogonal or bi-orthogonal solution depending on the initialization.
Contourlets \cite{do2005contourlet} combine the Laplacian pyramid scheme with DFB which has PR but with redundancy $4/3$ inherited from the Laplacian pyramid.
Shearlet \cite{shearlet12book,easley2008sparse} and curvelet \cite{candes2006fast} systems construct a multi-resolution partition of the frequency domain by applying shear or rotation operators to a generator function in each level of frequency decomposition. Available shearlet and curvelet implementations have redundancy at least 4; moreover, the factor may grow with the number of directions in the decomposition level.
% to be modified and add M-band version
Dual-tree wavelets \cite{selesnick2005dual} are linear combinations of 2D tensor wavelets (corresponding to multi-resolution systems) that constitute an approximate Hilbert transform pair, where the high frequency ring is divided into pairs of squares of different directional preference.

% what's the problem we will address, don't focusing too much on shearlet for the moment
However, none of these multi-resolution schemes is PR, critically downsampled and regularized (localized in both time and frequency). In the framework of nuDFB (\cite{nuDFB05}), it was shown by Durand \cite{durand2007} that it is impossible to construct orthonormal filters localized in frequency without discontinuity in their frequency support, or -- equivalently -- regularized filters without aliasing. His construction of directional filters uses compositions of 2-band filters associated to quincunx lattice, similar to that of uniform DFB in \cite{nuDFB05}; as pointed out in \cite{nuDFB05} the overall composed filters are not alias-free. It is not clear whether Durand's argument also precludes the existence of a regularized wavelet system, if one slightly weakens the set of conditions.
%Among these,  shearlets and curvelets have optimal asymptotic rate of approximation for ``cartoon images''(piecewise smooth, with jumps occurring along piecewise $C^2$-curves), due to the parabolic scaling rule in the frequency domain \cite{guo2007optimally,candes2005curvelet}; they have been successfully applied to image denoising \cite{easley2009shearlet}, restoration \cite{candes2002new} and separation \cite{kutyniok2012image}. Despite their theoretical potential, the (often high) redundancy of curvelets and shearlets impedes their practical usage. 
%Redundancy is useful in image processing tasks such as denoising, restoration and reconstruction, but a non-redundant basis decomposition is preferred in tasks where computation cost is of concern.

To study this question, we consider multi-resolution directional wavelets corresponding to the same partition of frequency domain as nuDFB and build a framework to analyze the equivalent conditions of PR for critically downsampled as well as more general redundant schemes. In our previous work \cite{yin2014orthshear}, we show that in MRA, PR is equivalent to an identity condition and shift-cancellation condition closely related to the frequency support of filters and their downsampling scheme. Based on these two conditions, we rederived Durand's discontinuity result of orthonormal schemes; we also show that a slight relaxation of conditions allows frames with redundancy less than 2 that circumvent the regularity limitation. Furthermore, we have an explicit approach to construct such regularized directional wavelet frames by smoothing the Fourier transform of the irregular directional wavelets.
The main contribution of this paper is that we extend our previous work and show that the same obstruction to regularity as in orthonormal schemes exists in bi-orthogonal schemes. Different from our previous approach in the orthonormal case, our analysis of bi-orthogonal schemes is based on a numerical algorithm introduced by Cohen et al in \cite{cohen1993compactly} for constructing compactly supported symmetric wavelet bases on a hexagonal lattice. We extend and adapt this algorithm to our bi-orthogonal setting.

The paper is organized as follows. In section \ref{sec: setup}, we set up the framework of a dyadic MRA with dilated quincunx downsampling. In section \ref{sec: orth}, we review the irregularity of orthonormal schemes in \cite{yin2014orthshear}. In particular, we derive two conditions, {\it identity summation} and {\it shift cancellation}, equivalent to perfect reconstruction in this MRA with critical downsampling. These lead to the classification of {\it regular/singular} boundaries of the frequency partition %and the corresponding smoothing techniques to improve spatial localization. We compare our and Durand's directional wavelet constructions. 
%In section \ref{sec: frame}, 
and a {\it relaxed shift-cancellation} condition for low-redundancy MRA frame allows better regularity of the directional wavelets. 
In section \ref{sec: bi-orth}, we show the irregularity for bi-orthogonal schemes. We first review the wavelet construction algorithm from \cite{cohen1993compactly} which solves linear systems generated from regularity constraints. Next, we extend the algorithm to our framework and show that the resulting linear system does not have feasible solution satisfying all regularity constraints, especially continuity of Fourier transforms of wavelet filters.  
Finally, we conclude our results and discuss future work in section \ref{sec: end}.
