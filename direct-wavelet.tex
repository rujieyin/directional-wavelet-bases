\documentclass[]{article}

\renewcommand\abstractname{\textit{Abstract}}
\usepackage[cmex10]{amsmath}
\usepackage{amsfonts, amssymb}
\usepackage{color}
\usepackage{bbm}
\usepackage{amsthm}
\usepackage{graphicx}
\graphicspath{{./orth-frame/figure/}{./bi-orth/figure/}}
\makeatletter
\newcommand*{\ov}[1]{\m@th\overline{\mbox{#1}\raisebox{.8em}{}}}
\newcommand{\V}[1]{\boldsymbol{#1}}% boldface for vector and matrix
\usepackage[toc,page]{appendix}
\usepackage{wrapfig}
\usepackage{color}
\usepackage{verbatim}
\usepackage{caption}

\let\oldemptyset\emptyset
\let\emptyset\varnothing
\newcommand{\mo}[1]{m_{#1}(\boldsymbol{\omega})}
\newcommand{\m}[1]{\widetilde{m_{#1}}(\boldsymbol{\omega})}
\renewcommand{\mp}[2]{\widetilde{m_{#1}}(\boldsymbol{\omega}+\boldsymbol{\pi}_{#2})}
\newcommand{\mc}[1]{\widetilde{m_{#1}}^C(\boldsymbol{\omega})}
\newcommand{\barm}[1]{\overline{\widetilde{m_{#1}}(\boldsymbol{\omega})}}
\newcommand{\sbarm}[1]{\overline{\widetilde{m_{#1}}}(\boldsymbol{\omega})}
\newcommand{\barmp}[2]{\overline{\widetilde{m_{#1}}(\boldsymbol{\omega} + \boldsymbol{\pi}_{#2})}}
\newcommand{\barmn}[2]{\overline{\widetilde{m_{#1}}(\boldsymbol{\omega} + \boldsymbol{\nu}_{#2})}}
\newcommand{\sbarmp}[2]{\overline{\widetilde{m_{#1}}}(\boldsymbol{\omega} + \boldsymbol{\pi}_{#2})}
\newcommand{\sbarmn}[2]{\overline{\widetilde{m_{#1}}}(\boldsymbol{\omega} + \boldsymbol{\nu}_{#2})}
\newcommand{\M}{\widetilde{\mathbf{M}}}
\newcommand{\Msub}{\widetilde{\mathbf{M}}^{\Box}}
\newcommand{\G}{\mathcal{G}}
\newcommand{\mvec}[1]{\mathbf{\widetilde{m_{#1}}}}
\newcommand{\mhat}[1]{\widehat{\widetilde{m_{#1}}}(\boldsymbol{\omega})}
\newcommand{\xvec}{\mathbf{x}}
\newcommand{\wvec}{\mathbf{w}}
\newcommand{\mrow}[1]{\widetilde{\mathbf{m}}^{#1}}
\newtheorem{theorem}{Theorem}[section]
\newtheorem{lemma}[theorem]{Lemma}
\newtheorem{proposition}[theorem]{Proposition}
\newtheorem{corollary}[theorem]{Corollary}
\renewcommand{\qed}{\hfill\ensuremath{\square}}


\begin{document}

\newtheorem{lem}{Lemma}
\newtheorem*{mydef}{Definition}
\newtheorem{thm}{Theorem}
\newtheorem{prop}{Proposition}
\newtheorem*{notat}{Notation}

\abovedisplayskip=2pt
\belowdisplayskip=2pt
\abovedisplayshortskip=2pt
\belowdisplayshortskip=2pt


\title{Directional Wavelet Bases Construction on Dyadic Quincunx Lattice}
\author{Rujie Yin, Ingrid Daubechies}
\maketitle

\begin{abstract}
We construct directional wavelet systems that have the same direction selectivity as shearlets in the first frequency dyadic ring and non-uniform directional wavelet filterbanks(nuDFB). In particular, dilated quincunx downsampling is used to construct orthonormal and bi-orthogonal bases and standard dyadic downsampling for low-redundancy frames. We prove that the support of orthonormal and bi-orthogonal wavelets is discontinuous in the frequency domain, which can be avoided in frames of redundancy as low as 2. These are the first step towards the construction of efficient shearlet systems.
\end{abstract}

\section{Introduction}

\iffalse
\begin{itemize}
\item review construction of directional wavelet, shearlet
\item what's new in our construction?
\item summary: framework, technique, reference(Durand, Cohen)
\item organization of the paper
\end{itemize}
\fi

% introducing directional wavelets
In image compression and analysis, 2D tensor wavelet schemes are widely used. Despite the time-frequency localization inherited from 1D wavelet, 2D tensor wavelets suffers from poor orientation selectivity: only horizontal or vertical edges are well represented by tensor wavelets. To obtain better representation of 2D images, several directional wavelet schemes have been proposed and applied to image processing, including directional wavelet filterbanks(DFB), contourlet, curvelet, shearlet and dual-tree wavelet.

% summary
Conventional DFB \cite{DFB92} divides the frequency domain into eight equi-angular pairs of triangles and it is critically downsampled (maximally decimated) and perfect reconstruction (PR), but without multi-resolution structure. 
A non-uniform DFB(nuDFB) is introduced in \cite{nuDFB05} where the high frequency ring is divided into six equi-angular pairs of trapezoids and the central low frequency square is kept for division in the next level of decomposition. The nuDFB is solved directly by optimization which provides a non-unique near orthogonal or bi-orthogonal solution depending on the initialization.
Contourlets \cite{do2005contourlet} combine the Laplacian pyramid scheme with DFB which has PR but with redundancy $4/3$ inherited from the Laplacian pyramid.
Shearlet \cite{shearlet12book,easley2008sparse} and curvelet \cite{candes2006fast} systems construct a multi-resolution partition of the frequency domain by applying shear or rotation operators to a generator function in each level. Depending on the generator function and the number of directions, available shearlet and curvelet implementations have redundancy at least 4; moreover, the factor may grow with the decomposition level.
% to be modified and add M-band version
Dual-tree wavelets \cite{selesnick2005dual} are linear combinations of 2D tensor wavelets (corresponding to multi-resolution systems) that constitute an approximate Hilbert transform pair. 

% what's the problem we will address, don't focusing too much on shearlet for the moment
None of these schemes is PR, critically downsampled and regularized (localized in both time and frequency) with multi-resolution structure. 
In the framework of nuDFB (\cite{nuDFB05}), it is shown by Durand \cite{durand2007} that it's impossible to construct orthonormal filters localized in frequency without discontinuity in their frequency support, or equivalently regularized filters without aliasing. His construction of directional filters using compositions of 2-band filters associated to quincunx lattice, similar to that of uniform DFB in \cite{nuDFB05} and as pointed out in \cite{nuDFB05} the overall composed filters are not alias-free.
%Among these,  shearlets and curvelets have optimal asymptotic rate of approximation for ``cartoon images''(piecewise smooth, with jumps occurring along piecewise $C^2$-curves), due to the parabolic scaling rule in the frequency domain \cite{guo2007optimally,candes2005curvelet}; they have been successfully applied to image denoising \cite{easley2009shearlet}, restoration \cite{candes2002new} and separation \cite{kutyniok2012image}. Despite their theoretical potential, the (often high) redundancy of curvelets and shearlets impedes their practical usage. 
%Redundancy is useful in image processing tasks such as denoising, restoration and reconstruction, but a non-redundant basis decomposition is preferred in tasks where computation cost is of concern.

In this paper, we consider multi-resolution directional wavelets corresponding to the same partition of frequency domain as nuDFB and build a framework to analyze the equivalent conditions of PR for critically downsampled and more generally redundant schemes. In our previous work \cite{yin2014orthshear}, we show that in multi-resolution analysis(MRA), PR is equivalent to an identity condition and shift-cancellation condition closely related to the frequency support of filters and their downsampling scheme. Based on these two conditions, we derive Durand's discontinuity result of orthonormal schemes and a relaxation of orthonormal schemes to frame with redundancy 2 that resolves the regularity limitation. A regularized directional wavelet scheme of redundancy 2 that satisfies the identity condition and the relaxed shift-cancellation condition, is constructed directly by smoothing the Fourier transform of the corresponding wavelet.
We extend our previous work here and show that the same irregularity in orthonormal schemes exists in bi-orthogonal schemes. Our analysis of bi-orthogonal schemes is based on a numerical algorithm introduced by Cohen et al in \cite{cohen1993compactly} for constructing compactly supported symmetric wavelet bases on hexagonal lattice. We extend and adapt this algorithm to our bi-orthogonal framework.

The paper is organized as follows, in section \ref{sec: setup}, we set up our framework of a dyadic MRA with dilated quincunx downsampling. In section \ref{sec: orth}, we review the irregularity of orthonormal schemes in \cite{yin2014orthshear}. In particular, we derive two conditions, {\it identity summation} and {\it shift cancellation}, equivalent to perfect reconstruction in this MRA with critical downsampling. These lead to the classification of {\it regular/singular} boundaries of the frequency partition %and the corresponding smoothing techniques to improve spatial localization. We compare our and Durand's directional wavelet constructions. 
%In section \ref{sec: frame}, 
and a {\it relaxed shift-cancellation} condition for low-redundancy MRA frame allows better regularity of the directional wavelets. 
In section \ref{sec: bi-orth}, we show the irregularity for bi-orthogonal schemes. We first review the wavelet construction algorithm in \cite{cohen1993compactly} which solves linear systems generated from regularity constraints. Next, we extend the algorithm to our framework and show that the resulting linear system doesn't have feasible solution satisfying all regularity constraints, especially continuity of Fourier transforms of wavelet filters.  
Finally, we conclude our results and discuss future work in section \ref{sec: end}.

%\section{Introduction}

\iffalse
\begin{itemize}
\item review construction of directional wavelet, shearlet
\item what's new in our construction?
\item summary: framework, technique, reference(Durand, Cohen)
\item organization of the paper
\end{itemize}
\fi

% introducing directional wavelets
In image compression and analysis, 2D tensor wavelet schemes are widely used. Despite the time-frequency localization inherited from 1D wavelet, 2D tensor wavelets suffers from poor orientation selectivity: only horizontal or vertical edges are well represented by tensor wavelets. To obtain better representation of 2D images, several directional wavelet schemes have been proposed and applied to image processing, including directional wavelet filterbanks(DFB), contourlet, curvelet, shearlet and dual-tree wavelet.

% summary
Conventional DFB \cite{DFB92} divides the frequency domain into eight equi-angular pairs of triangles and it is critically downsampled (maximally decimated) and perfect reconstruction (PR), but without multi-resolution structure. 
A non-uniform DFB(nuDFB) is introduced in \cite{nuDFB05} where the high frequency ring is divided into six equi-angular pairs of trapezoids and the central low frequency square is kept for division in the next level of decomposition. The nuDFB is solved directly by optimization which provides a non-unique near orthogonal or bi-orthogonal solution depending on the initialization.
Contourlets \cite{do2005contourlet} combine the Laplacian pyramid scheme with DFB which has PR but with redundancy $4/3$ inherited from the Laplacian pyramid.
Shearlet \cite{shearlet12book,easley2008sparse} and curvelet \cite{candes2006fast} systems construct a multi-resolution partition of the frequency domain by applying shear or rotation operators to a generator function in each level. Depending on the generator function and the number of directions, available shearlet and curvelet implementations have redundancy at least 4; moreover, the factor may grow with the decomposition level.
% to be modified and add M-band version
Dual-tree wavelets \cite{selesnick2005dual} are linear combinations of 2D tensor wavelets (corresponding to multi-resolution systems) that constitute an approximate Hilbert transform pair. 

% what's the problem we will address, don't focusing too much on shearlet for the moment
None of these schemes is PR, critically downsampled and regularized (localized in both time and frequency) with multi-resolution structure. 
In the framework of nuDFB (\cite{nuDFB05}), it is shown by Durand \cite{durand2007} that it's impossible to construct orthonormal filters localized in frequency without discontinuity in their frequency support, or equivalently regularized filters without aliasing. His construction of directional filters using compositions of 2-band filters associated to quincunx lattice, similar to that of uniform DFB in \cite{nuDFB05} and as pointed out in \cite{nuDFB05} the overall composed filters are not alias-free.
%Among these,  shearlets and curvelets have optimal asymptotic rate of approximation for ``cartoon images''(piecewise smooth, with jumps occurring along piecewise $C^2$-curves), due to the parabolic scaling rule in the frequency domain \cite{guo2007optimally,candes2005curvelet}; they have been successfully applied to image denoising \cite{easley2009shearlet}, restoration \cite{candes2002new} and separation \cite{kutyniok2012image}. Despite their theoretical potential, the (often high) redundancy of curvelets and shearlets impedes their practical usage. 
%Redundancy is useful in image processing tasks such as denoising, restoration and reconstruction, but a non-redundant basis decomposition is preferred in tasks where computation cost is of concern.

In this paper, we consider multi-resolution directional wavelets corresponding to the same partition of frequency domain as nuDFB and build a framework to analyze the equivalent conditions of PR for critically downsampled and more generally redundant schemes. In our previous work \cite{yin2014orthshear}, we show that in multi-resolution analysis(MRA), PR is equivalent to an identity condition and shift-cancellation condition closely related to the frequency support of filters and their downsampling scheme. Based on these two conditions, we derive Durand's discontinuity result of orthonormal schemes and a relaxation of orthonormal schemes to frame with redundancy 2 that resolves the regularity limitation. A regularized directional wavelet scheme of redundancy 2 that satisfies the identity condition and the relaxed shift-cancellation condition, is constructed directly by smoothing the Fourier transform of the corresponding wavelet.
We extend our previous work here and show that the same irregularity in orthonormal schemes exists in bi-orthogonal schemes. Our analysis of bi-orthogonal schemes is based on a numerical algorithm introduced by Cohen et al in \cite{cohen1993compactly} for constructing compactly supported symmetric wavelet bases on hexagonal lattice. We extend and adapt this algorithm to our bi-orthogonal framework.

The paper is organized as follows, in section \ref{sec: setup}, we set up our framework of a dyadic MRA with dilated quincunx downsampling. In section \ref{sec: orth}, we review the irregularity of orthonormal schemes in \cite{yin2014orthshear}. In particular, we derive two conditions, {\it identity summation} and {\it shift cancellation}, equivalent to perfect reconstruction in this MRA with critical downsampling. These lead to the classification of {\it regular/singular} boundaries of the frequency partition %and the corresponding smoothing techniques to improve spatial localization. We compare our and Durand's directional wavelet constructions. 
%In section \ref{sec: frame}, 
and a {\it relaxed shift-cancellation} condition for low-redundancy MRA frame allows better regularity of the directional wavelets. 
In section \ref{sec: bi-orth}, we show the irregularity for bi-orthogonal schemes. We first review the wavelet construction algorithm in \cite{cohen1993compactly} which solves linear systems generated from regularity constraints. Next, we extend the algorithm to our framework and show that the resulting linear system doesn't have feasible solution satisfying all regularity constraints, especially continuity of Fourier transforms of wavelet filters.  
Finally, we conclude our results and discuss future work in section \ref{sec: end}.
%\section{Introduction}

\iffalse
\begin{itemize}
\item review construction of directional wavelet, shearlet
\item what's new in our construction?
\item summary: framework, technique, reference(Durand, Cohen)
\item organization of the paper
\end{itemize}
\fi

% introducing directional wavelets
In image compression and analysis, 2D tensor wavelet schemes are widely used. Despite the time-frequency localization inherited from 1D wavelet, 2D tensor wavelets suffers from poor orientation selectivity: only horizontal or vertical edges are well represented by tensor wavelets. To obtain better representation of 2D images, several directional wavelet schemes have been proposed and applied to image processing, including directional wavelet filterbanks(DFB), contourlet, curvelet, shearlet and dual-tree wavelet.

% summary
Conventional DFB \cite{DFB92} divides the frequency domain into eight equi-angular pairs of triangles and it is critically downsampled (maximally decimated) and perfect reconstruction (PR), but without multi-resolution structure. 
A non-uniform DFB(nuDFB) is introduced in \cite{nuDFB05} where the high frequency ring is divided into six equi-angular pairs of trapezoids and the central low frequency square is kept for division in the next level of decomposition. The nuDFB is solved directly by optimization which provides a non-unique near orthogonal or bi-orthogonal solution depending on the initialization.
Contourlets \cite{do2005contourlet} combine the Laplacian pyramid scheme with DFB which has PR but with redundancy $4/3$ inherited from the Laplacian pyramid.
Shearlet \cite{shearlet12book,easley2008sparse} and curvelet \cite{candes2006fast} systems construct a multi-resolution partition of the frequency domain by applying shear or rotation operators to a generator function in each level. Depending on the generator function and the number of directions, available shearlet and curvelet implementations have redundancy at least 4; moreover, the factor may grow with the decomposition level.
% to be modified and add M-band version
Dual-tree wavelets \cite{selesnick2005dual} are linear combinations of 2D tensor wavelets (corresponding to multi-resolution systems) that constitute an approximate Hilbert transform pair. 

% what's the problem we will address, don't focusing too much on shearlet for the moment
None of these schemes is PR, critically downsampled and regularized (localized in both time and frequency) with multi-resolution structure. 
In the framework of nuDFB (\cite{nuDFB05}), it is shown by Durand \cite{durand2007} that it's impossible to construct orthonormal filters localized in frequency without discontinuity in their frequency support, or equivalently regularized filters without aliasing. His construction of directional filters using compositions of 2-band filters associated to quincunx lattice, similar to that of uniform DFB in \cite{nuDFB05} and as pointed out in \cite{nuDFB05} the overall composed filters are not alias-free.
%Among these,  shearlets and curvelets have optimal asymptotic rate of approximation for ``cartoon images''(piecewise smooth, with jumps occurring along piecewise $C^2$-curves), due to the parabolic scaling rule in the frequency domain \cite{guo2007optimally,candes2005curvelet}; they have been successfully applied to image denoising \cite{easley2009shearlet}, restoration \cite{candes2002new} and separation \cite{kutyniok2012image}. Despite their theoretical potential, the (often high) redundancy of curvelets and shearlets impedes their practical usage. 
%Redundancy is useful in image processing tasks such as denoising, restoration and reconstruction, but a non-redundant basis decomposition is preferred in tasks where computation cost is of concern.

In this paper, we consider multi-resolution directional wavelets corresponding to the same partition of frequency domain as nuDFB and build a framework to analyze the equivalent conditions of PR for critically downsampled and more generally redundant schemes. In our previous work \cite{yin2014orthshear}, we show that in multi-resolution analysis(MRA), PR is equivalent to an identity condition and shift-cancellation condition closely related to the frequency support of filters and their downsampling scheme. Based on these two conditions, we derive Durand's discontinuity result of orthonormal schemes and a relaxation of orthonormal schemes to frame with redundancy 2 that resolves the regularity limitation. A regularized directional wavelet scheme of redundancy 2 that satisfies the identity condition and the relaxed shift-cancellation condition, is constructed directly by smoothing the Fourier transform of the corresponding wavelet.
We extend our previous work here and show that the same irregularity in orthonormal schemes exists in bi-orthogonal schemes. Our analysis of bi-orthogonal schemes is based on a numerical algorithm introduced by Cohen et al in \cite{cohen1993compactly} for constructing compactly supported symmetric wavelet bases on hexagonal lattice. We extend and adapt this algorithm to our bi-orthogonal framework.

The paper is organized as follows, in section \ref{sec: setup}, we set up our framework of a dyadic MRA with dilated quincunx downsampling. In section \ref{sec: orth}, we review the irregularity of orthonormal schemes in \cite{yin2014orthshear}. In particular, we derive two conditions, {\it identity summation} and {\it shift cancellation}, equivalent to perfect reconstruction in this MRA with critical downsampling. These lead to the classification of {\it regular/singular} boundaries of the frequency partition %and the corresponding smoothing techniques to improve spatial localization. We compare our and Durand's directional wavelet constructions. 
%In section \ref{sec: frame}, 
and a {\it relaxed shift-cancellation} condition for low-redundancy MRA frame allows better regularity of the directional wavelets. 
In section \ref{sec: bi-orth}, we show the irregularity for bi-orthogonal schemes. We first review the wavelet construction algorithm in \cite{cohen1993compactly} which solves linear systems generated from regularity constraints. Next, we extend the algorithm to our framework and show that the resulting linear system doesn't have feasible solution satisfying all regularity constraints, especially continuity of Fourier transforms of wavelet filters.  
Finally, we conclude our results and discuss future work in section \ref{sec: end}.


\section{Framework Setup}\label{sec: setup}
\begin{itemize}

\item MRA, dilated quincun lattice, shifts associated with lattice
\end{itemize}
We summarize 2D-MRA systems, matrix representations of sub-lattices of $\mathbb{Z}^2$ and the relation between frequency domain partition and sub-lattice with critical downsampling.

\subsection{Notation}
Throughout this paper, we use lower case normal font for function, normal font for scalar, upper case bold font for matrix, lower case bold font for vector and upper case normal font for frequency domain.

\subsection{Multi-resolution analysis and critical downsampling}
In an MRA, given a scaling function $\phi\in L^2(\mathbb{R}^2)$, s.t. $\Vert\phi\Vert_2=1$,
the base approximation space is defined as $V_0 = \overline{span\{\phi_{0,\boldsymbol{k}}\}}_{\boldsymbol{k}\in\mathbb{Z}^2}$, where $\phi_{0,\boldsymbol{k}} = \phi(\boldsymbol{x}-\boldsymbol{k})$. If $\langle \phi_{0,\boldsymbol{k}},\phi_{0,\boldsymbol{k'}}\rangle = \delta_{\boldsymbol{k,k'}}$, then $\{\phi_{0,\boldsymbol{k}}\}$ is an orthogonal basis of $V_0$. Moreover, $\phi$ is associated with a scaling matrix $\mathbf{D}\in\mathbb{Z}^{2\times 2}$ with determinant $|\mathbf{D}|$, s.t. the rescaled 
 $\phi_1(\boldsymbol{x}) = |\mathbf{D}|^{-1/2}\phi(\mathbf{D}^{-1}\boldsymbol{x})$ is a linear combination of $\phi_{0,\boldsymbol{k}}$.
Equivalently, in the frequency domain
\begin{align}\label{eq: m0}
\widehat{\phi}(\mathbf{D}^T\boldsymbol{\omega}) = m_0(\boldsymbol{\omega})\widehat{\phi}(\boldsymbol{\omega}),
\end{align}
where $m_0(\boldsymbol{\omega}) = m_0(\omega_1,\omega_2)$, $2\pi-$periodic in $\omega_1,\omega_2$. Hence
\begin{align}\label{eq: phi-m0}
\textstyle \hat{\phi}(\boldsymbol{\omega}) = (2\pi)^{-1}\prod_{k=1}^{\infty}m_0(\mathbf{D}^{-k} \boldsymbol{\omega}).
\end{align}
\\[2em]
The MRA uses the nested approximation spaces $V_l = \overline{span\{\phi(\mathbf{D}^{-l}\boldsymbol{x}-\boldsymbol{k});\boldsymbol{k}\in\mathbb{Z}^2\}},\,l\in\mathbb{Z}$. 
Next, suppose there are wavelet functions $\psi^j\in L^2(\mathbb{R}^2)$, {\small $1 \leq j \leq J$}, and $\mathbf{Q}\in\mathbb{Z}^{2\times2}$, s.t. the space $W_1 = \bigcup_{j=1}^J W_1^j = \bigcup_{j=1}^J \overline{span\{\psi^j(\mathbf{D}^{-1}\boldsymbol{x-k});\boldsymbol{ k}\in \mathbf{Q}\mathbb{Z}^2\}}$ is the orthogonal complement of $V_1$ with respect to $V_0$. Let $\psi^j_{l,\boldsymbol{k'}} = |\mathbf{D}|^{-l/2}\psi^j(\mathbf{D}^{-l}\boldsymbol{x-k'})$; an $L$-level multi-resolution system with base space $V_0$ is then spanned by
 \begin{align}\label{eq: MRA}
 \{\phi_{L,\boldsymbol{k}}\,,\psi^j_{l,\boldsymbol{k'}}\,, \, {\small 1\leq l \leq L,\, \boldsymbol{k}\in \mathbb{Z}^2,\,\boldsymbol{k'}\in \mathbf{Q}\mathbb{Z}^2,\,1\leq j \leq J\}}.
\end{align}  
 As $W_1\subset V_0$, each rescaled wavelet $\psi^j(\mathbf{D}^{-1}\cdot)$ is also a linear combination of $\phi_{0,\boldsymbol{k}}$, so that $\exists m_j$ analogous to $m_0$
satisfying 
\begin{align}\label{eq: mj}
\widehat{\psi}^j(\mathbf{D}^T\boldsymbol{\omega}) = m_j(\boldsymbol{\omega})\widehat{\phi}(\boldsymbol{\omega}),\hspace{1cm} 1\leq j \leq J.
\end{align}
In this construction of MRA, the scaling function $\phi$ and all the wavelet functions $\psi^j$ share the same scaling matrix $\mathbf{D}$, yet the family of shifted $\phi_{\boldsymbol{k}}$ is defined on $\mathbb{Z}^2$, whereas the family of shifted $\psi^j_{\boldsymbol{k}}$ is defined on a sub-integer lattice $\mathbf{Q}\mathbb{Z}^2$. Hence the corresponding subsampling matrix  of $\phi_{1,\boldsymbol{k}}$ is $\mathbf{D}$ and that of $\psi^j_{1,\boldsymbol{k}}$ is $\mathbf{QD}$, as in \cite{durand2007}. We haven't yet imposed any condition on this MRA, or equivalently, on $m-$functions and the subsampling matrices $\mathbf{D}$ and $\mathbf{Q}$; this comes next.

If the MRA generated by \eqref{eq: MRA} achieves critical downsampling, then $ |\mathbf{D}|^{-1} + J|\mathbf{QD}|^{-1} = 1$ (\cite{durand2007}); 
critical downsampling thus depends only on the subsampling matrices $\mathbf{D}$ and $\mathbf{Q}$. The space decomposition structure $V_0 = V_1\bigoplus W_1$ in MRA and \eqref{eq: m0}, \eqref{eq: mj} require consistency between the $m-$functions and the subsampling matrices $\mathbf{D}$ and $\mathbf{Q}$. 

\subsection{Frequency domain partition and sub-lattice sampling}

\begin{figure}[!t]
\centering
\begin{minipage}[c]{.35\textwidth}
\includegraphics[width=\textwidth]{shannon-marked-new.jpg}
\end{minipage}\hspace*{2em}
\begin{minipage}[c]{.35\textwidth}
\includegraphics[width=\textwidth]{sublattice-2.jpg}
\end{minipage}
\caption{Left: partition of $S_0$ and boundary assignment of $C_j$, $j = 1,\cdots,6$ ( each $C_j$ has boundaries indicated by red line segments), Right: dilated quincunx sub-lattice. }
\label{fig: partition}
\vspace*{-5mm}
\end{figure}

\begin{mydef}
If $\mathcal{L}$ is the lattice generated by $\boldsymbol{a}_1,\boldsymbol{a}_2$, i.e. $\mathcal{L} = \sum_{i=1,2}k_i\boldsymbol{a}_i,\,k_i\in\mathbb{Z}$,
the {\bf reciprocal lattice} $\mathcal{L}^*$ of $\mathcal{L}$ is the lattice generated by the vectors $\boldsymbol{b}_1,\boldsymbol{b}_2$, s.t. $\boldsymbol{b}_i^T\boldsymbol{a}_j = 2\pi\delta_{ij}$. 

\end{mydef}
%\begin{mydef}
%Given a lattice $\mathcal{L}$, a {\bf fundamental domain} $S$ in $\mathbb{R}^2$ with respect to $\mathcal{L}$, denoted as $S = \mathbb{R}^2/\mathcal{L}$, is a subset of $\mathbb{R}^2$, such that $\bigcup_{l\in\mathcal{L}}(S+l) = \mathbb{R}^2$ and $S\cap(S+l)=\varnothing,\,\forall l\in\mathcal{L}\setminus\{\mathbf{0}\}$.
%A set $S$ is a {\bf frequency support} of $\mathcal{L}$ if $S = \mathbb{R}^2/\mathcal{L}^*$.
%\end{mydef}
%Furthermore, each sub-lattice $\mathcal{L}$ of $\mathbb{Z}^2$ is associated with a set of shifts $\Gamma, \, s.t. \bigcup_{\V{\gamma}\in\Gamma}\mathcal{L}+\V{\gamma} = \mathbb{Z}^2$ and $|\Gamma| = |\mathbb{Z}^2/\mathcal{L}|$.

To build our first example, in which $\hat{\phi},\,\widehat{\psi}^j$ are indicator functions in $\mathbb{R}^2$, we consider the case where $\mathcal{L} = \mathbb{Z}^2,\,\mathcal{L}^* = 2\pi\mathbb{Z}^2$ and we pick
$S_0=\mathbb{R}^2/(\mathbb{Z}^2)^*$, the canonical frequency square, $[-\pi,\pi)\times[-\pi,\pi)$. 
Since $\phi_1,\psi^j_1$ and their shifts span the space $V_0$, $supp(\widehat{\phi}_1)$ and $supp(\widehat{\psi}^j)$, together, should thus cover $S_0$. Due to \eqref{eq: m0} and \eqref{eq: mj}, this is equivalent to $S_0=\bigcup_{0\leq j\leq J} supp(m_j\vert_{S_0})$. That is, if $C_j,\,{\small 0\leq j\leq J}$ are the main support of $m_j,\,0\leq j\leq J$ respectively, then they form a partition of $S_0$. An non-uniform admissible partition is defined as follows,

\begin{mydef}
$C_j, 0\leq j\leq J$ is an {\bf admissible} partition of $S_0$ if and only if $\exists \mathbf{D}, \mathbf{Q}\in\mathbb{Z}^{2\times 2}$, s.t. the low frequency piece $C_0 = \mathbb{R}^2/(\mathbf{D}\mathbb{Z}^2)^*,$ and the high frequency pieces $C_j = \mathbb{R}^2/(\mathbf{QD}\mathbb{Z}^2)^*,\,j = 1,\cdots,J$.
\end{mydef}
Let {\small $\V{\pi}_0 = (0,0), \V{\pi}_1 = (\pi/2,\pi/2), \V{\pi}_2 = (\pi,0),\V{\pi}_3 = (-\pi/2,\pi/2), \V{\pi}_4 = (0,\pi), \V{\pi}_5 = (\pi/2,-\pi/2),\V{\pi}_6 = (\pi,\pi), \V{\pi}_7=(-\pi/2,-\pi/2)$}, then 
\begin{align*}
\mathbb{R}^2/(\mathbb{Z}^2)^*=\bigcup_{\V{\pi}\in\Gamma_0}\left(\mathbb{R}^2/(\V{D_2}\mathbb{Z}^2)^* + \V{\pi}\right)
=\bigcup_{\V{\pi}'\in\Gamma_1}\left(\mathbb{R}^2/(\V{QD_2}\mathbb{Z}^2)^* + \V{\pi}'\right),
\end{align*}
where
%then $\mathbf{D_2}\mathbb{Z}^2$ is associated with the set of shifts 
$\Gamma_0= \{\V{\pi}_i,\, i = 0,2,4,6\}$ and 
%$\mathbf{QD_2}\mathbb{Z}^2$ is associated with 
$\Gamma_1 = \{\V{\pi}_i,\, i = 0,\cdots,7\}$.
To build an orthonormal basis with good directional selectivity, we choose the partition of $S_0$ to be that of the least redundant shearlet system, see Fig.\ref{fig: partition} left, which is also Example B in \cite{durand2007}. In this partition, $S_0$ is divided into a central square $C_0 = S_1 := \bigl(\begin{smallmatrix} 2&0\\0&2\end{smallmatrix}\bigr)^{-1}S_0$ and a ring: the ring is further cut into six pairs of directional trapezoids $C_j$'s by lines passing through the origin with slopes $\pm 1, \pm 3$ and $\pm \frac{1}{3}$. The central square $S_1$ can be further partitioned in the same way to obtain a two-level multi-resolution system, as shown in Fig.\ref{fig: partition}. 

This partition is admissible and corresponds to $\mathbf{D} = \mathbf{D_2}\doteq\bigl(\begin{smallmatrix} 2&0\\0&2\end{smallmatrix}\bigr)$ and $\mathbf{Q}:=\bigl(\begin{smallmatrix} 1&1\\-1&1\end{smallmatrix}\bigr)$. The wavelet coefficients are taken on the dilated quincunx sub-lattice $\mathbf{QD_2}\mathbb{Z}^2$ (see the right panel in Fig.\ref{fig: partition}).
In addition, $|\mathbf{D_2}| = 4,\,|\mathbf{Q}| = 2$ so that $1/4 + 6/ (2\cdot 4) = 1$, and the system is critical downsampling.
%\section{Framework Setup}\label{sec: setup}
\begin{itemize}

\item MRA, dilated quincun lattice, shifts associated with lattice
\end{itemize}
We summarize 2D-MRA systems, matrix representations of sub-lattices of $\mathbb{Z}^2$ and the relation between frequency domain partition and sub-lattice with critical downsampling.

\subsection{Notation}
Throughout this paper, we use lower case normal font for function, normal font for scalar, upper case bold font for matrix, lower case bold font for vector and upper case normal font for frequency domain.

\subsection{Multi-resolution analysis and critical downsampling}
In an MRA, given a scaling function $\phi\in L^2(\mathbb{R}^2)$, s.t. $\Vert\phi\Vert_2=1$,
the base approximation space is defined as $V_0 = \overline{span\{\phi_{0,\boldsymbol{k}}\}}_{\boldsymbol{k}\in\mathbb{Z}^2}$, where $\phi_{0,\boldsymbol{k}} = \phi(\boldsymbol{x}-\boldsymbol{k})$. If $\langle \phi_{0,\boldsymbol{k}},\phi_{0,\boldsymbol{k'}}\rangle = \delta_{\boldsymbol{k,k'}}$, then $\{\phi_{0,\boldsymbol{k}}\}$ is an orthogonal basis of $V_0$. Moreover, $\phi$ is associated with a scaling matrix $\mathbf{D}\in\mathbb{Z}^{2\times 2}$ with determinant $|\mathbf{D}|$, s.t. the rescaled 
 $\phi_1(\boldsymbol{x}) = |\mathbf{D}|^{-1/2}\phi(\mathbf{D}^{-1}\boldsymbol{x})$ is a linear combination of $\phi_{0,\boldsymbol{k}}$.
Equivalently, in the frequency domain
\begin{align}\label{eq: m0}
\widehat{\phi}(\mathbf{D}^T\boldsymbol{\omega}) = m_0(\boldsymbol{\omega})\widehat{\phi}(\boldsymbol{\omega}),
\end{align}
where $m_0(\boldsymbol{\omega}) = m_0(\omega_1,\omega_2)$, $2\pi-$periodic in $\omega_1,\omega_2$. Hence
\begin{align}\label{eq: phi-m0}
\textstyle \hat{\phi}(\boldsymbol{\omega}) = (2\pi)^{-1}\prod_{k=1}^{\infty}m_0(\mathbf{D}^{-k} \boldsymbol{\omega}).
\end{align}
\\[2em]
The MRA uses the nested approximation spaces $V_l = \overline{span\{\phi(\mathbf{D}^{-l}\boldsymbol{x}-\boldsymbol{k});\boldsymbol{k}\in\mathbb{Z}^2\}},\,l\in\mathbb{Z}$. 
Next, suppose there are wavelet functions $\psi^j\in L^2(\mathbb{R}^2)$, {\small $1 \leq j \leq J$}, and $\mathbf{Q}\in\mathbb{Z}^{2\times2}$, s.t. the space $W_1 = \bigcup_{j=1}^J W_1^j = \bigcup_{j=1}^J \overline{span\{\psi^j(\mathbf{D}^{-1}\boldsymbol{x-k});\boldsymbol{ k}\in \mathbf{Q}\mathbb{Z}^2\}}$ is the orthogonal complement of $V_1$ with respect to $V_0$. Let $\psi^j_{l,\boldsymbol{k'}} = |\mathbf{D}|^{-l/2}\psi^j(\mathbf{D}^{-l}\boldsymbol{x-k'})$; an $L$-level multi-resolution system with base space $V_0$ is then spanned by
 \begin{align}\label{eq: MRA}
 \{\phi_{L,\boldsymbol{k}}\,,\psi^j_{l,\boldsymbol{k'}}\,, \, {\small 1\leq l \leq L,\, \boldsymbol{k}\in \mathbb{Z}^2,\,\boldsymbol{k'}\in \mathbf{Q}\mathbb{Z}^2,\,1\leq j \leq J\}}.
\end{align}  
 As $W_1\subset V_0$, each rescaled wavelet $\psi^j(\mathbf{D}^{-1}\cdot)$ is also a linear combination of $\phi_{0,\boldsymbol{k}}$, so that $\exists m_j$ analogous to $m_0$
satisfying 
\begin{align}\label{eq: mj}
\widehat{\psi}^j(\mathbf{D}^T\boldsymbol{\omega}) = m_j(\boldsymbol{\omega})\widehat{\phi}(\boldsymbol{\omega}),\hspace{1cm} 1\leq j \leq J.
\end{align}
In this construction of MRA, the scaling function $\phi$ and all the wavelet functions $\psi^j$ share the same scaling matrix $\mathbf{D}$, yet the family of shifted $\phi_{\boldsymbol{k}}$ is defined on $\mathbb{Z}^2$, whereas the family of shifted $\psi^j_{\boldsymbol{k}}$ is defined on a sub-integer lattice $\mathbf{Q}\mathbb{Z}^2$. Hence the corresponding subsampling matrix  of $\phi_{1,\boldsymbol{k}}$ is $\mathbf{D}$ and that of $\psi^j_{1,\boldsymbol{k}}$ is $\mathbf{QD}$, as in \cite{durand2007}. We haven't yet imposed any condition on this MRA, or equivalently, on $m-$functions and the subsampling matrices $\mathbf{D}$ and $\mathbf{Q}$; this comes next.

If the MRA generated by \eqref{eq: MRA} achieves critical downsampling, then $ |\mathbf{D}|^{-1} + J|\mathbf{QD}|^{-1} = 1$ (\cite{durand2007}); 
critical downsampling thus depends only on the subsampling matrices $\mathbf{D}$ and $\mathbf{Q}$. The space decomposition structure $V_0 = V_1\bigoplus W_1$ in MRA and \eqref{eq: m0}, \eqref{eq: mj} require consistency between the $m-$functions and the subsampling matrices $\mathbf{D}$ and $\mathbf{Q}$. 

\subsection{Frequency domain partition and sub-lattice sampling}

\begin{figure}[!t]
\centering
\begin{minipage}[c]{.35\textwidth}
\includegraphics[width=\textwidth]{shannon-marked-new.jpg}
\end{minipage}\hspace*{2em}
\begin{minipage}[c]{.35\textwidth}
\includegraphics[width=\textwidth]{sublattice-2.jpg}
\end{minipage}
\caption{Left: partition of $S_0$ and boundary assignment of $C_j$, $j = 1,\cdots,6$ ( each $C_j$ has boundaries indicated by red line segments), Right: dilated quincunx sub-lattice. }
\label{fig: partition}
\vspace*{-5mm}
\end{figure}

\begin{mydef}
If $\mathcal{L}$ is the lattice generated by $\boldsymbol{a}_1,\boldsymbol{a}_2$, i.e. $\mathcal{L} = \sum_{i=1,2}k_i\boldsymbol{a}_i,\,k_i\in\mathbb{Z}$,
the {\bf reciprocal lattice} $\mathcal{L}^*$ of $\mathcal{L}$ is the lattice generated by the vectors $\boldsymbol{b}_1,\boldsymbol{b}_2$, s.t. $\boldsymbol{b}_i^T\boldsymbol{a}_j = 2\pi\delta_{ij}$. 

\end{mydef}
%\begin{mydef}
%Given a lattice $\mathcal{L}$, a {\bf fundamental domain} $S$ in $\mathbb{R}^2$ with respect to $\mathcal{L}$, denoted as $S = \mathbb{R}^2/\mathcal{L}$, is a subset of $\mathbb{R}^2$, such that $\bigcup_{l\in\mathcal{L}}(S+l) = \mathbb{R}^2$ and $S\cap(S+l)=\varnothing,\,\forall l\in\mathcal{L}\setminus\{\mathbf{0}\}$.
%A set $S$ is a {\bf frequency support} of $\mathcal{L}$ if $S = \mathbb{R}^2/\mathcal{L}^*$.
%\end{mydef}
%Furthermore, each sub-lattice $\mathcal{L}$ of $\mathbb{Z}^2$ is associated with a set of shifts $\Gamma, \, s.t. \bigcup_{\V{\gamma}\in\Gamma}\mathcal{L}+\V{\gamma} = \mathbb{Z}^2$ and $|\Gamma| = |\mathbb{Z}^2/\mathcal{L}|$.

To build our first example, in which $\hat{\phi},\,\widehat{\psi}^j$ are indicator functions in $\mathbb{R}^2$, we consider the case where $\mathcal{L} = \mathbb{Z}^2,\,\mathcal{L}^* = 2\pi\mathbb{Z}^2$ and we pick
$S_0=\mathbb{R}^2/(\mathbb{Z}^2)^*$, the canonical frequency square, $[-\pi,\pi)\times[-\pi,\pi)$. 
Since $\phi_1,\psi^j_1$ and their shifts span the space $V_0$, $supp(\widehat{\phi}_1)$ and $supp(\widehat{\psi}^j)$, together, should thus cover $S_0$. Due to \eqref{eq: m0} and \eqref{eq: mj}, this is equivalent to $S_0=\bigcup_{0\leq j\leq J} supp(m_j\vert_{S_0})$. That is, if $C_j,\,{\small 0\leq j\leq J}$ are the main support of $m_j,\,0\leq j\leq J$ respectively, then they form a partition of $S_0$. An non-uniform admissible partition is defined as follows,

\begin{mydef}
$C_j, 0\leq j\leq J$ is an {\bf admissible} partition of $S_0$ if and only if $\exists \mathbf{D}, \mathbf{Q}\in\mathbb{Z}^{2\times 2}$, s.t. the low frequency piece $C_0 = \mathbb{R}^2/(\mathbf{D}\mathbb{Z}^2)^*,$ and the high frequency pieces $C_j = \mathbb{R}^2/(\mathbf{QD}\mathbb{Z}^2)^*,\,j = 1,\cdots,J$.
\end{mydef}
Let {\small $\V{\pi}_0 = (0,0), \V{\pi}_1 = (\pi/2,\pi/2), \V{\pi}_2 = (\pi,0),\V{\pi}_3 = (-\pi/2,\pi/2), \V{\pi}_4 = (0,\pi), \V{\pi}_5 = (\pi/2,-\pi/2),\V{\pi}_6 = (\pi,\pi), \V{\pi}_7=(-\pi/2,-\pi/2)$}, then 
\begin{align*}
\mathbb{R}^2/(\mathbb{Z}^2)^*=\bigcup_{\V{\pi}\in\Gamma_0}\left(\mathbb{R}^2/(\V{D_2}\mathbb{Z}^2)^* + \V{\pi}\right)
=\bigcup_{\V{\pi}'\in\Gamma_1}\left(\mathbb{R}^2/(\V{QD_2}\mathbb{Z}^2)^* + \V{\pi}'\right),
\end{align*}
where
%then $\mathbf{D_2}\mathbb{Z}^2$ is associated with the set of shifts 
$\Gamma_0= \{\V{\pi}_i,\, i = 0,2,4,6\}$ and 
%$\mathbf{QD_2}\mathbb{Z}^2$ is associated with 
$\Gamma_1 = \{\V{\pi}_i,\, i = 0,\cdots,7\}$.
To build an orthonormal basis with good directional selectivity, we choose the partition of $S_0$ to be that of the least redundant shearlet system, see Fig.\ref{fig: partition} left, which is also Example B in \cite{durand2007}. In this partition, $S_0$ is divided into a central square $C_0 = S_1 := \bigl(\begin{smallmatrix} 2&0\\0&2\end{smallmatrix}\bigr)^{-1}S_0$ and a ring: the ring is further cut into six pairs of directional trapezoids $C_j$'s by lines passing through the origin with slopes $\pm 1, \pm 3$ and $\pm \frac{1}{3}$. The central square $S_1$ can be further partitioned in the same way to obtain a two-level multi-resolution system, as shown in Fig.\ref{fig: partition}. 

This partition is admissible and corresponds to $\mathbf{D} = \mathbf{D_2}\doteq\bigl(\begin{smallmatrix} 2&0\\0&2\end{smallmatrix}\bigr)$ and $\mathbf{Q}:=\bigl(\begin{smallmatrix} 1&1\\-1&1\end{smallmatrix}\bigr)$. The wavelet coefficients are taken on the dilated quincunx sub-lattice $\mathbf{QD_2}\mathbb{Z}^2$ (see the right panel in Fig.\ref{fig: partition}).
In addition, $|\mathbf{D_2}| = 4,\,|\mathbf{Q}| = 2$ so that $1/4 + 6/ (2\cdot 4) = 1$, and the system is critical downsampling.

\section{Orthonormal Bases}\label{sec: orth}

In this section, we discuss the conditions on $m-$ functions such that the corresponding MRA forms an orthonormal bases. 
%consider orthonormal bases with $m-$functions defined in \eqref{eq: m0} and \eqref{eq: mj} whose supports mainly corresponding to the partition of $S_0$ in Fig.\ref{fig: partition}.
%we always consider a multi-resolution system with scaling function $\phi$ and quasi-shearlets $\psi^j$, {\small$ j = 1,\dots,6$} defined by $(M_0, D_2)$ and $(M_j, Q)$, {\small$j = 1,\dots,6$} respectively. Furthermore, the essential support of $M_j$'s corresponds to the partition of $\mathbf{S}_0$ in a shearlet system.

%The construction of \eqref{eq: MRA} reduces to design $m_0$ in \eqref{eq: m0} and $m_j, j= 1,\cdots,6$ in \eqref{eq: mj}.
We begin with the two key conditions, i.e. {\it identity summation} and {\it shift cancellation}, on $m-$functions such that the system \eqref{eq: MRA} is perfect reconstruction (PR) or equivalently a Parseval frame in MRA.% weaker than the orthonormal condition.

%\subsection{Identity summation and shift cancellation}
\subsection{orthonormal conditions on $m-$functions}
In MRA, \eqref{eq: MRA} is PR if $\forall f\in L_2(\mathbb{R}^2)$,
\begin{equation}
\textstyle \sum_k\langle f,\phi_{0,k}\rangle\phi_{0,k} = \sum_k\langle f,\phi_{1,k}\rangle\phi_{1,k} + \sum_j\sum_k\langle f,\psi^{j}_{1,k}\rangle\psi^j_{1,k}.\label{eq: PR}
\end{equation}
Using \eqref{eq: m0} and \eqref{eq: mj} together with the admissibility of the frequency partition \eqref{eq: tiling}, condition \eqref{eq: PR} on $\phi$ and $\psi^j$'s yields:
\begin{thm}\label{thm: conds}
The perfect reconstruction condition holds for \eqref{eq: MRA} iff the following two conditions hold
\begin{align}\label{eq: id-sum}
|m_0(\boldsymbol{\omega})|^2 + \sum_{j = 1}^6|m_j(\boldsymbol{\omega})|^2 = 1
\end{align}
\begin{equation}\label{eq: shift-cancel}
 \begin{cases}
%M_0(\boldsymbol{omega})\overline{M_0(\boldsymbol{omega}+\boldsymbol{\gamma})} + 
\sum_{j = 0}^6m_j(\boldsymbol{\omega})\overline{m_j(\boldsymbol{\omega} + \boldsymbol{\pi})} = 0, & \boldsymbol{\pi}\in \Gamma_0\setminus\{\boldsymbol{0}\}\\[.5em]
\sum_{j=1}^6m_j(\boldsymbol{\omega})\overline{m_j(\boldsymbol{\omega}+\boldsymbol{\pi})} = 0, & \boldsymbol{\pi}\in\Gamma_1\setminus\Gamma_0
\end{cases}
\end{equation}
%where  $\Lambda = (QD\mathbb{Z}^2)^*/(\mathbb{Z}^2)^*,\,\Gamma = (D\mathbb{Z}^2)^*/(\mathbb{Z}^2)^*.$ %$\{(\frac{\pi}{2},\frac{\pi}{2}),(\frac{3\pi}{2},\frac{\pi}{2}),(\frac{\pi}{2},\frac{3\pi}{2}),(\frac{3\pi}{2},\frac{3\pi}{2}),$ $(0,0),(0,\pi),(\pi,0),(\pi,\pi)\}$
\end{thm}

Theorem \ref{thm: conds} is a corollary of Prop. 1 and Prop. 2 in \cite{durand2007}. We give an alternate proof in Appendix \ref{app: cond-thm}.
In Theorem \ref{thm: conds}, Eq. \eqref{eq: id-sum} is the {\it identity summation} condition, guaranteeing conservation of $l_2$ energy; Eq. \eqref{eq: shift-cancel} is the {\it shift cancellation} condition such that aliasing is canceled correctly in reconstruction from wavelet coefficients. %, such that downsampling of scaling and shearlet coefficients is valid. 
 %$\Lambda = \{(\frac{\pi}{2},\frac{\pi}{2}),(\frac{3\pi}{2},\frac{\pi}{2}),(\frac{\pi}{2},\frac{3\pi}{2}),(\frac{3\pi}{2},\frac{3\pi}{2}),$ $(0,0),(0,\pi),(\pi,0),(\pi,\pi)\},\Gamma = \{(0,0),(0,\pi),(\pi,0),(\pi,\pi)\}$ are the sets of shifts associated to quincunx-dyadic dilation $QD$ and dyadic dilation $D$ respectively.
%Each $M_j$ contributes a term $M_j(\boldsymbol{omega})\overline{M_j(\boldsymbol{omega} + \boldsymbol{\nu})}$ in the cancellation condition of any shift $\boldsymbol{\nu}$ corresponding to the downsampling scheme of $M_j$.

%\subsection{Extra condition for basis}
%By Theorem \ref{thm: conds}, the system \eqref{eq: MRA} is a Parseval frame ; 
For  \eqref{eq: MRA} to be an orthonormal basis,  $\{\phi_{\boldsymbol{k}}\}_{\boldsymbol{k}\in\mathbb{Z}^2}$ need to be an orthonormal basis, which is determined by $m_0$ in \eqref{eq: phi-m0}. In 1D MRA, Cohen's theorem in \cite{cohen1992biorthogonal} provides a necessary and sufficient condition on $m_0$ such that \eqref{eq: MRA} is an orthonormal basis. %Such a condition in 1D wavelet MRA is given by Cohen \cite{cohen1992biorthogonal}. 
We generalized this theorem to 2D in \cite{yin2014orthshear}.
\begin{thm}\label{thm: basis cond}
Assume that $m_0$ is a trigonometric polynomial with $m_0(0)=1$, and define $\hat{\phi}(\boldsymbol{\omega})$ as in \eqref{eq: phi-m0}.\\
If $\phi(\cdot - \boldsymbol{k}),\boldsymbol{k}\in\mathbb{Z}^2$ are orthonormal, then $\exists K$ containing a neighborhood of 0, s.t. $\forall\boldsymbol{\omega}\in S_0,\,\boldsymbol{\omega}+2\pi\mathbf{n}\in K$ for some $\mathbf{n}\in\mathbb{Z}^2, $ and $\inf_{k>0,\,\boldsymbol{\omega}\in K}|m_0(\mathbf{D_2}^{-k}\boldsymbol{\omega})| >0$. 
 Further, if $\sum_{\boldsymbol{\V{\pi}}\in \Gamma_0} |m_0(\boldsymbol{\omega}+\boldsymbol{\pi})|^2 = 1$, then the inverse is true.
\end{thm}
%Theorem \ref{thm: basis cond} can be proved similarly to Cohen's theorem (\cite{cohen1992biorthogonal}).
%Because it is difficult to directly design $M_0$ that satisfies the conditions in Theorem \ref{thm: basis cond}, 
%Below, we construct $m-$functions imposing only \eqref{eq: id-sum} and \eqref{eq: shift-cancel} and then check if the resulting Parseval frame is an orthonormal basis by applying Theorem \ref{thm: basis cond} to $m_0$.

\subsection{$m$-function Design and Boundary Regularity}\label{sec: design}
%In this section, we define Shannon-type directional orthonormal basis same as in \cite{durand2007} and \cite{nuDFB05}. Then, we apply direct smoothing to its $m-$functions to improve spatial localization, this leads to a critical analysis of the boundary regularity of $S_1$ and the $C_j$'s.

%\subsection{Shannon-type wavelets and smoothing}
We begin with the Shannon-type wavelet construction.
Let each $m_j$ be an indicator function, $ m_j = \mathbbm{1}_{C_j},\,0\leq j \leq 6,\,$ and we use the boundary assignment of $C_j$ in Fig.\ref{fig: partition},
then the identity summation follows from the partition, and the shift cancellation follows from \eqref{eq: tiling}. % which implies $m_j(\boldsymbol{\omega})\overline{m_j(\boldsymbol{\omega} + \boldsymbol{\pi}_i)}\equiv 0,\,\forall j,\, i\neq 0.$ %Let $\partial\mathbf{C}_j = \overline{\mathbf{C}_j}-\mathbf{C}_j^{\circ}$ be the boundary of $\mathbf{C}_j$,  
Applying Theorem \ref{thm: basis cond} to $m_0$, we check that the Shannon-type wavelets generated from these $m-$functions form an orthonormal basis.

%However, because of the discontinuity of $m_j$ across the boundary of its support, the corresponding wavelet has slow decay in the time domain. 
Such $m_j$'s are not regularized due to the discontinuity on $\partial C_j$'s, the boundaries of $C_j$'s, hence the corresponding wavelets are not spatially localized. $m_j$'s can be regularized by direct smoothing on $\partial C_j$'s.
%We take a different regularization approach from Durand's \cite{durand2007}, where three regular quincunx filter banks are constructed and then composed to obtain the desired regular quincunx dyadic filter banks. Here, we smooth the discontinuous boundaries of $m-$functions directly. 
However, as shown in Proposition 3 in \cite{durand2007}, it is not possible to smooth all the boundaries with discontinuity if $m_j$'s have to satisfy the perfect reconstruction condition.
In \cite{yin2014orthshear}, $\partial C_j$'s are segmented into {\it singular} and {\it regular} pieces. On regular boundaries, pairs of $(m_j,\, m_{j'})$ that share the boundary can be smoothed in a coherent way such that all the constraints in Theorem \ref{thm: basis cond} are satisfied. Yet, on those singular pieces, it is shown that $m_j$ cannot be smoothed without violating the shift cancellation condition. Fig. \ref{fig: boundary} shows the boundary classification, where the corners of $S_0$ and $C_0$ are singular, hence $m_0$ and the diagonal $m-$ functions of an orthonormal bases are discontinuous there. A mechanism of constructing orthonormal bases by smoothing $m_j$ on regular boundaries from indicator functions is provided in \cite{yin2014orthshear}.%However, it remains unclear if some of the discontinuity can be removed by direct smoothing.

%Next, we analyze the limitation of direct smoothing in detail and show that there are regular boundaries which can be smoothed without violating \eqref{eq: id-sum} and \eqref{eq: shift-cancel}, and singular boundaries which cannot be smoothed.

%\textcolor{red}{In the remainder of this paper, we explore how this can be done for the first dyadic layer (without cutting further at higher frequencies); we call the corresponding basis {\it quasi-Shearlets}.}

\begin{figure}[!t]
\centering
%\hspace*{-5mm}
%\begin{minipage}{.6\textwidth}
%\includegraphics[width=\textwidth]{fig_overlap.png}\hspace*{-2mm}
%\end{minipage}
\begin{minipage}{.3\textwidth}
\includegraphics[width=\textwidth]{bdyclass.jpg}
\end{minipage}
%\includegraphics[width=.2\textwidth]{fig/sublattice-2.jpg}
\caption[caption]{
%\textcolor{red}{
%{\it Left top}: the supports of $m_j$ (green) and $m_j(\cdot+\V{\pi}_2$ (red) for $j = 5,6$ after smoothing, overlap on the vertical boundary at $\omega_1 =  \pm \pi/2$ of $C_5$ (green) and its shift (red) by $\boldsymbol{\pi}_2 = (\pi,0)$. Note that two copies of shifted $S_0$ (red) overlap the un-shifted $S_0$(green) due to the $(2\pi,2\pi)$ periodicity of $m_j$. Only $m_0$ and $m_5$ have overlapping smoothed boundaries by $\boldsymbol{\pi}_2$. 
%\\\hspace{\textwidth}
%{\it Left bottom}: intersection of $\mathcal{B}(0,\boldsymbol{\pi}_2)$ and $\mathcal{B}(5,\boldsymbol{\pi}_2)$ in yellow and $\mathcal{C}(0,\boldsymbol{\pi}_2) = \mathcal{B}(0,\boldsymbol{\pi}_2)\setminus\mathcal{B}(5,\boldsymbol{\pi}_2)$ in red. Smoothing $m_0$ in the red (singular) region is impossible without violating \eqref{eq: shift-cancel}. %This leads to the distinction between regular (yellow) and singular (red) boundaries at $omega_1 = \pm\pi/2$.
%\\\hspace{\textwidth}
%{\it Right}: 
Boundary classification, singular (red) and regular (yellow) }%after similar arguments for all shifts $\boldsymbol{\pi}_i$.}
 %}
\label{fig: boundary}
\end{figure}

\begin{comment}
%\subsection{Boundary classification}
 After smoothing the $m_j$'s, the shift cancellation \eqref{eq: shift-cancel} may fail to hold as $supp(m_j)$ and $(supp(m_j)-\boldsymbol{\pi})$ may overlap near the smoothed boundaries, see Fig. \ref{fig: boundary}, illustrating $m_5(\boldsymbol{\omega})\overline{m_5(\boldsymbol{\omega} + \boldsymbol{\pi}_2)}\not\equiv 0.$
For simplicity, we introduce the following notations: let 
$\mathcal{B}(j,\boldsymbol{\pi}) =  supp(m_j)\cap\,(supp(m_j)-\boldsymbol{\pi})$ be the support of  $m_j(\boldsymbol{\omega})\overline{m_j(\boldsymbol{\omega} + \boldsymbol{\pi})}$ associated to $m_j$ and shift $\boldsymbol{\pi}$;
let $\mathcal{C}(j,\boldsymbol{\pi}) = \mathcal{B}(j,\boldsymbol{\pi}) \setminus \bigcup_{j'\neq j}\mathcal{B}(j',\boldsymbol{\pi})$. 
%In addition, we may introduce the (slight abuse of) notation $C_0 = S_1$.
\begin{lem}\label{lem: singular-bdy}
Shift cancellation \eqref{eq: shift-cancel} can hold for shift $\boldsymbol{\pi}\in\Gamma_0\setminus\{\boldsymbol{0}\}$, only if $\mathcal{C}(j,\boldsymbol{\pi})=\varnothing,\, \forall\, 0\leq j\leq 6$; 
it can hold for shift $\boldsymbol{\pi}\in\Gamma_1\setminus\Gamma_0$, only if $\mathcal{C}(j,\boldsymbol{\pi})=\varnothing,\, \forall\, 1\leq j\leq 6$. 
\end{lem}
\noindent{\it Proof.} Observe that, on $\mathcal{C}(j,\boldsymbol{\pi})$, $m_j(\boldsymbol{\omega})\overline{m_{j}(\boldsymbol{\omega}+\boldsymbol{\pi})} \not\equiv 0$ but $m_{j'}(\boldsymbol{\omega})\overline{m_{j'}(\boldsymbol{\omega}+\boldsymbol{\pi})} \equiv 0,\, \forall j' \neq j$, hence \eqref{eq: shift-cancel} doesn't hold. $\hfill\square$
%Furthermore, the identity summation constraint \eqref{eq: id-sum} implies that $|M_j(\boldsymbol{omega})| = 1$ on $\mathcal{C}(j,\boldsymbol{\nu}),\,\forall\boldsymbol{\nu}$, hence the discontinuity of $M$-functions across these boundaries is unavoidable.
 
Therefore, boundaries that after smoothing make $\mathcal{C}(j,\boldsymbol{\pi})$ non-empty are called {\it singular}; the rest are {\it regular}. We next provide an explicit boundary classification method:
\begin{prop}\label{prop: class-bdy}
$\forall\, 1\leq j\leq 6$, let $supp(m_j) = \ov{$C_j$}$, then the boundary of $C_j$ is $\partial C_j=\bigcup_{i \neq 0}\mathcal{B}(j,\boldsymbol{\pi}_i)$. The set of singular boundaries of $\partial C_j$ is $\bigcup_{i\neq 0}\mathcal{C}(j,\boldsymbol{\pi}_i)$, whereas its compliment set is the regular boundary set.
 \end{prop}
\noindent{\it Proof.} Since $\bigcup_i(C_j+\boldsymbol{\pi}_i) = S_0$, $\partial C_j\subset \bigcup_{i\neq 0} (\ov{$C_j$}+\boldsymbol{\pi}_i) $. Therefore, $\partial C_j\subset\bigcup_{i\neq 0}\mathcal{B}(j,\boldsymbol{\pi}_i)$. On the other hand, $\mathcal{B}(j,\boldsymbol{\pi}_i)\subset \partial\mathbf{C}_j, \forall \, i\neq 0$, hence the union of them is a subset of $\partial C_j$. It follows that $\mathcal{B}(j,\boldsymbol{\pi}_i)$ form a partition of the boundary $\partial C_j$. The partition of $\partial C_j$ into singular and regular boundaries follows from Lemma \ref{lem: singular-bdy}.$\hfill\square$

 The case of $\partial S_1$ is similar where $\mathcal{B}(0,\boldsymbol{\pi}),\,\boldsymbol{\pi}\in\Gamma_0\setminus\{0\}$ are considered. We use the notation $\mathcal{B}_s(j,\boldsymbol{\pi}),\mathcal{C}_s(j,\boldsymbol{\pi})$ for the special case $supp(m_j) = \ov{$C_j$}$ hereafter.
The boundary classification based on Proposition \ref{prop: class-bdy} is shown in the right of Fig. \ref{fig: boundary}, where the boundaries on the four corners of both $S_0$ and $S_1$ are singular: smoothing is then not allowed there.

\begin{prop}\label{prop: jump}
Let $\mathcal{C} = \mathcal{C}_s(j_1,\boldsymbol{\pi}_{i_1})\cap \mathcal{C}_s(j_2,\boldsymbol{\pi}_{i_2})$, if $m_{j_1}, m_{j_2}$ satisfy \eqref{eq: id-sum}, \eqref{eq: shift-cancel}, then $|m_{j_1}|=\mathbbm{1}_{\mathbf{C}_{j_1}},\,|m_{j_2}|=\mathbbm{1}_{\mathbf{C}_{j_2}}$ on $\mathcal{C}$.
\end{prop}
\noindent{\it Proof.} Suppose the common singular boundary $\mathcal{C}$ is non-empty and observe that $\mathcal{C}\subset(C_{j_1})\cap(C_{j_2})$. Since $m_{j_1}$ cannot be smoothed on $\mathcal{C}_s(j_1,\boldsymbol{\pi}_{i_1})$, $|m_{j_1}| = 0$ on $\mathcal{C}_s(j_1,\boldsymbol{\pi}_{i_1})\setminus\mathbf{C}_{j_1}$, 
%Because $\mathcal{C}\subset\mathcal{C}_s(j_1,\boldsymbol{\nu}_1)\cap\partial\mathbf{C}_{j_2}$, 
and \eqref{eq: id-sum} implies that $|m_{j_2}| = 1$ there, or equivalently $|m_{j_2}|=\mathbbm{1}_{\mathbf{C}_{j_2}}$ on $\mathcal{C}$. Similarly, $|m_{j_1}|=\mathbbm{1}_{\mathbf{C}_{j_1}}$ on $\mathcal{C}.\hfill\square$

Prop. \ref{prop: jump} shows that if $m_j$ and $m_{j'}$ have common singular boundaries, then both will have a discontinuity across those boundaries. For example, $\mathcal{C}_s(0,(\pi,0))\cap\mathcal{C}_s(4,(\pi/2,\pi/2)) = (\pi/2,(\pi/6,\pi/2))$, hence $m_0$ and $m_4$ both are discontinuous at $(\pi/2,(\pi/6,\pi/2))$. All the singular boundaries related to \eqref{eq: MRA} are such "double" singular boundaries.

\subsection{Pairwise smoothing of regular boundary}
%Despite of the singular boundaries, better spatial localization can be achieved by carefully smoothing the regular boundaries. 
The regular boundaries of both $C_{j_1}$ and $C_{j_2}$ with adjacent supports consist of $\mathcal{B}_s(j_1,\boldsymbol{\pi})\cap\mathcal{B}_s(j_2,\boldsymbol{\pi})$, which we denote by the triple $(j_1,j_2,\boldsymbol{\pi})$. The following proposition shows that the regular boundaries $(j_1,j_2,\boldsymbol{\pi})$ can be paired according to shift pairs $(\boldsymbol{\pi},-\boldsymbol{\pi})$, and the boundaries must be smoothed pairwise within their $\epsilon-$neighborhood, $\mathcal{B}_{\epsilon}(j_1,j_2,\boldsymbol{\pi})$ and $\mathcal{B}_{\epsilon}(j_1,j_2,-\boldsymbol{\pi})$.

\begin{prop}\label{prop: pair-smooth}
Given $(j_1,j_2,\boldsymbol{\pi})\neq \varnothing$, %let $\boldsymbol{\nu}'= -\boldsymbol{\nu}\in \mathbb{R}^2/(\mathbb{Z}^2)^*$, 
then $(j_1,j_2,-\boldsymbol{\pi})\neq\varnothing$. In addition, let $\mathcal{B}=\mathcal{B}_{\epsilon}(j_1,j_2,\boldsymbol{\pi})\cup\mathcal{B}_{\epsilon}(j_1,j_2,-\boldsymbol{\pi})$. Then the identity summation and shift cancellation conditions hold if
\begin{itemize}
\item[{\it (i)}] $m_j = \mathbbm{1}_{C_j},\quad\text{on }S_0,\; j\neq j_1,j_2$
\item[{\it (ii)}] $m_{j_1} =\mathbbm{1}_{C_{j_1}},\, m_{j_2} =\mathbbm{1}_{C_{j_2}},\quad \text{on }\mathcal{B}^c$\\[.5em]
\hspace*{-2em} and on $\mathcal{B}$ the following hold%\vspace*{.1em}
\item[{\it(iii)}] $|m_{j_1}|^2 + |m_{j_2}|^2 = 1,$ %\hfill on $ \mathcal{B}_{\epsilon}(j_1,j_2,\nu)\cup\mathcal{B}_{\epsilon}(j_1,j_2,\nu')$
\item[{\it (iv)}] $\sum_{j_1,j_2} m_j(\cdot)\overline{m_j(\cdot+\widetilde{\boldsymbol{\pi}})} = 0,\, \widetilde{\boldsymbol{\pi}} = \pm\boldsymbol{\pi}$
%\hspace*{8em} on $\mathcal{B}_{\epsilon}(j_1,j_2,\nu)$ and $\mathcal{B}_{\epsilon}(j_1,j_2,\nu)-\nu$
%\item[{\it (v)}] $\sum_{j_1,j_2} M_j(\cdot)\overline{M_j(\cdot+\boldsymbol{\nu}')} = 0.$ 
%\hspace*{7em} on $\mathcal{B}_{\epsilon}(j_1,j_2,\boldsymbol{\nu}')$ and $\mathcal{B}_{\epsilon}(j_1,j_2,\nu')-\nu'$
\end{itemize}
\end{prop}
\noindent{\it Proof}.
 We first show that $(j_1,j_2,-\boldsymbol{\pi})\neq\varnothing.$ By definition, $\mathcal{B}_s(j_1,\boldsymbol{\pi}) = supp(m_{j_1})\cap (supp(m_{j_1})-\boldsymbol{\pi}) = \left((supp(m_{j_1})+\boldsymbol{\pi})\cap supp(m_{j_1})\right)-\boldsymbol{\pi} = \mathcal{B}_s(j_1,-\boldsymbol{\pi})-\boldsymbol{\pi}.$ Rewrite $(j_1,j_2,\boldsymbol{\pi})$ by $\mathcal{B}_s(j_1,-\boldsymbol{\pi})$ and $\mathcal{B}_s(j_2,-\boldsymbol{\pi})$, we have $(j_1,j_2,-\boldsymbol{\pi}) = (j_1,j_2,\boldsymbol{\pi})+\boldsymbol{\pi}$, hence it's non-empty.
 
Because $(j_1,j_2,\pm\boldsymbol{\pi})\subset (\partial C_{j_1} \bigcap \partial C_{j_2})$ and $(\bigcup_{j}C_j)=S_0$, $(\bigcup_{j\neq j_1,j_2}C_j)\cap \mathcal{B} = \varnothing.$ Therefore, smoothing of $m_{j_1}$ and $m_{j_2}$ in $\mathcal{B}$ doesn't impact the region where other $m_j$'s are supported.

We then show that the cancellation conditions \eqref{eq: shift-cancel} hold for all shifts. Condition (i) and (ii) imply that $\mathcal{B}(j,\widetilde{\boldsymbol{\pi}})= \varnothing,\,\forall j,\widetilde{\boldsymbol{\pi}}\neq \pm \boldsymbol{\pi}$, hence \eqref{eq: shift-cancel} hold for $\widetilde{\boldsymbol{\pi}}\neq \pm \boldsymbol{\pi}$. %For $\boldsymbol{\nu}$, 
(i) implies $\mathcal{B}(j,\widetilde{\boldsymbol{\pi}}) = \varnothing,\,\forall j\neq j_1,j_2$, so then \eqref{eq: shift-cancel} is equivalent to (iv). %Similarly, \eqref{eq: shift-cancel} is equivalent to (v) under (i).
The identity summation \eqref{eq: id-sum} holds due to (i), (ii) and (iii).$\hfill\square$.

By Prop. \ref{prop: pair-smooth} we can smooth some pairs of regular boundaries starting from the Shannon-type directional wavelets with the simplified conditions (iii), (iv) and (v); (i) and (ii) can be removed as long as the initial $m_j$ satisfy \eqref{eq: id-sum} and \eqref{eq: shift-cancel} and every $\boldsymbol{\omega}\in S_0$ is not covered by more than two $m$ functions. We can thus smooth regular boundaries pairwise, one by one.

The next proposition gives an explicit design of $(m_{j_1}, m_{j_2})$ satisfying the simplified conditions (iii),(iv) in Proposition \ref{prop: pair-smooth}. %as well as a necessary condition for any valid design.
\begin{prop}\label{prop: m-design}
Let $C\subset S_0$, given $m_{j_1},m_{j_2}\neq 0$ continuous on $ C\cup(C+\boldsymbol{\pi})$, satisfying the following conditions
\begin{itemize}
\item[(i)] $\sum_{j_1,j_2}m_{j}(\boldsymbol{\omega})\overline{m_{j}(\boldsymbol{\omega}+\boldsymbol{\pi})} = 0$ \hspace*{2em} on $C$
\item[(ii)] $\sum_{j_1,j_2}|m_{j}(\boldsymbol{\omega})|^2= 1 $ \hspace*{5em} on $C\cup (C+\boldsymbol{\pi})$
\item[(iii)] $m_{j_1}(\boldsymbol{\omega})m_{j_2}(\boldsymbol{\omega}) = 0$\hspace*{5em} on $\partial C$ ;
\end{itemize}
then
$\quad|m_{j_1}(\boldsymbol{\omega})| = |m_{j_2}(\boldsymbol{\omega}+\boldsymbol{\pi})|,\quad |m_{j_2}(\boldsymbol{\omega})| = |m_{j_1}(\boldsymbol{\omega}+\boldsymbol{\pi})|.$\\[1mm]
Furthermore, if $m_{j} = e^{i\boldsymbol{\omega}^{T}\boldsymbol{\eta}_{j}}\mathrm{m}_{j},\quad j=j_1,j_2, \text{ on }C,$ where $\mathrm{m}_j$ is a real-valued function, % $\mathcal{M}_{j_1}$ and $\mathcal{M}_{j_2}$,  phase $\eta_1,\eta_2$ s.t.
$e^{i\boldsymbol{\pi}^T(\boldsymbol{\eta}_{j1}-\boldsymbol{\eta}_{j2})} = -1,$ and 
\[\mathrm{m}_{j_1}(\boldsymbol{\omega}) = \mathrm{m}_{j_2}(\boldsymbol{\omega}-\boldsymbol{\pi}),\;\mathrm{m}_{j_2}(\boldsymbol{\omega}) = \mathrm{m}_{j_1}(\boldsymbol{\omega}-\boldsymbol{\pi}),\text{ on }C+\boldsymbol{\pi},\] 
%where $\mathcal{M}_{j1},\,\mathcal{M}_{j2}$ are 
then $(i)$ holds.
\end{prop}
\noindent{\it Proof.} 
To prove the necessary condition, note that (i) implies $|m_{j_1}(\boldsymbol{\omega})|^2|m_{j_1}(\boldsymbol{\omega+\pi})|^2 = |m_{j_2}(\boldsymbol{\omega})|^2|m_{j_2}(\boldsymbol{\omega+\pi})|^2$; the condition then follows from (ii).
For the sufficient construction, check by directly substituting the construction into (i). $\hfill\square$
%For the necessary condition, we prove in two cases. Suppose $|M_{j_1}|  = |M_{j_2}|$ on $\omega$, then (ii) implies $|M_{j_1}| = |M_{j_2}| = \frac{1}{2}$ on $\omega$. From (i), it's necessary $|M_{j_1}(\boldsymbol{omega}+\boldsymbol{\nu})|=|M_{j_2}(\boldsymbol{omega}+\boldsymbol{\nu})|$ on $\omega$, or equivalently, $|M_{j_1}| = |M_{j_2}|$ on $\omega + \boldsymbol{\nu}$. Apply (ii) again, we have $|M_{j_1}| = |M_{j_2}| = \frac{1}{2}$ on $\omega+\boldsymbol{\nu}$, therefore, $|M_{j_1}(\boldsymbol{omega})| = |M_{j_2}(\boldsymbol{omega}+\boldsymbol{\nu})|=|M_{j_2}(\boldsymbol{omega})| = |M_{j_1}(\boldsymbol{omega}+\boldsymbol{\nu})|=\frac{1}{2}.$%However, (iii) implies that either $M_{j_1}$ or $M_{j_2}$ decays to zero on the boundary, therefore, $|M_{j_1}$ and $|M_{j_2}|$ cannot be constant.

%Suppose now $|M_{j_1}(\boldsymbol{omega})| = |M_{j_2}(\boldsymbol{omega}+\boldsymbol{\nu})|$ on $\omega$, then (i) implies $|M_{j_2}(\boldsymbol{omega})| = |M_{j_1}(\boldsymbol{omega}+\boldsymbol{\nu})|$ on $\omega$.


Proposition \ref{prop: m-design} breaks down the design of $(m_{j_1},m_{j_2})$ into a pair of real functions $(\mathrm{m}_{j_1}, \mathrm{m}_{j_2})$ on $\mathcal{B}_{\epsilon}(j_1,j_2,\boldsymbol{\pi})$ and two vectors $\boldsymbol{\eta}_1,\boldsymbol{\eta}_2$; then $(\mathrm{m}_{j_1}, \mathrm{m}_{j_2})$ on $\mathcal{B}_{\epsilon}(j_1,j_2,-\boldsymbol{\pi})$ are automatically determined. %When all regular boundaries adopt this smoothing scheme,  each $\mathcal{M}_j$ has to locally match up with every $\mathcal{M}_{j'}$'s that shares a regular boundary $(j,j',\nu)$. Therefore, we may focus on 
The only constraint on $(\mathrm{m}_{j_1},\mathrm{m}_{j_2})$ for (ii) in Proposition \ref{prop: m-design} to hold is that on $\mathcal{B}_{\epsilon}(j_1,j_2,\boldsymbol{\pi})$, $\sum_{j_1,j_2}|\mathrm{m}_{j}(\boldsymbol{\omega})|^2= 1$, which is easy to be satisfied.
We may construct all local pairs of $(\mathrm{m}_{j_1},\mathrm{m}_{j_2})$ separately, and put together afterwards different pieces of each $\mathrm{m}_j$ located in different regular boundary neighborhoods $\mathcal{B}_{\epsilon}(j,j',\boldsymbol{\pi})$. 

\vspace*{.2em}
%The phase term $e^{i\boldsymbol{omega}^T\boldsymbol{\eta}_j}$ is preferably defined on the full frequency domain, hence $\boldsymbol{\eta}_j$'s need to be solved globally. This global phase problem is stated precisely in t
The next proposition gives % conditions for the $\boldsymbol{\eta}_j$ as well as 
one solution, easy to verify.
\begin{prop}\label{prop: phase}
Applying Proposition \ref{prop: m-design} to all regular boundaries requires a set of phases $\{\boldsymbol{\eta}_j\}_{j = 0}^6,$ s.t.\[\textstyle e^{i\boldsymbol{\nu}^T(\boldsymbol{\eta}_{j1}-\boldsymbol{\eta}_{j2})} = -1, \quad\forall (j_1,j_2,\boldsymbol{\pi})\in\Delta,\]
{\small\begin{multline*}
\Delta = \Bigl\{\big(0,2,(0,\pi)\big),\, \big(0,5,(\pi,0)\big),\,\big(1,3,(\pi,0)\big),\,\big(4,6,(0,\pi)\big),\\
\big(1,6,(\pi/2, 3\pi/2)\big),\,\big(2,3,(\pi/2, 3\pi/2)\big),\,\big(4,5,(\pi/2, 3\pi/2)\big), \\
\big(3,4,(\pi/2, \pi/2)\big),\big(1,2,(\pi/2, \pi/2)\big),\,\big(5,6,(\pi/2, \pi/2)\big)\Bigr\}
\end{multline*}}
The following is a (non-unique) solution: 
{\small\begin{multline*}\boldsymbol{\eta}_0 = (0,0),\,\boldsymbol{\eta}_1 = (0,0),\,\boldsymbol{\eta}_2 = (1,1),\,\boldsymbol{\eta}_3 = (1,-1),\\
\boldsymbol{\eta}_4 = (0,2),\,\boldsymbol{\eta}_5=(1,1),\,\boldsymbol{\eta}_6 = (-1,1).
\end{multline*}}
\end{prop}

To summarize, Proposition \ref{prop: m-design} and \ref{prop: phase} introduce the following regular boundary smoothing scheme for the $m$ functions:
\begin{description}% prevent items from splitting
\item[construction of orthonormal basis]\
\begin{itemize}
\item[1.] First, set $\mathrm{m}_j = \mathbbm{1}_{C_j}$; then smoothen these across a pair of regular boundaries $(j_1,j_2,\pm\boldsymbol{\pi})$ following steps 2, 3.
\item[2.]  On $\mathcal{B}_{\epsilon}(j_1,j_2,\boldsymbol{\pi})$,\\
\hspace*{2em} design $(\mathrm{m}_{j_1},\mathrm{m}_{j_2}),\quad$ s.t.
$\sum_{j_1,j_2}|\mathrm{m}_{j}(\boldsymbol{\omega})|^2= 1$.
\item[3.] On $\mathcal{B}_{\epsilon}(j_1,j_2,-\boldsymbol{\pi})$, \\
\hspace*{2em}let $\mathrm{m}_{j_1}(\boldsymbol{\omega}) = \mathrm{m}_{j_2}(\boldsymbol{\omega}-\boldsymbol{\pi})$, $\mathrm{m}_{j_2}(\boldsymbol{\omega}) = \mathrm{m}_{j_1}(\boldsymbol{\omega}-\boldsymbol{\pi})$\vspace*{.1em}
\item[4.] Repeat step 2 and 3 for all $(j_1,j_2,\boldsymbol{\pi})\in\Delta$. 
\item[5.]$m_j(\boldsymbol{\omega}) =e^{i\boldsymbol{\omega}^T\boldsymbol{\eta}_j} \mathrm{m}_j(\boldsymbol{\omega}),$ on $S_0$, with the $\boldsymbol{\eta}_j$ of Prop. \ref{prop: phase}.
\end{itemize}
\end{description}

\begin{figure}[!t]
\centering
\includegraphics[height=.3\textwidth]{contour_design.pdf}\hspace*{2mm}
\includegraphics[height=.3\textwidth]{smsh-sh.jpg}
\caption{ Left: contour design of $supp(m_5)$, Right: frequency support $|\hat{\psi}^j|$}
\label{fig: design}
\end{figure}

%\section{Quasi-shearlet bases construction}\label{sec: bases construction}
%In this section, we present a family of quasi-shearlet bases constructed based on the $M$-function design discussed in section \ref{sec: design} using our proposed MRA framework.

We apply this to smooth all the regular boundaries except those on the boundary of $S_0$. Near a regular boundary $\mathcal{B}_{\epsilon}(j,j',\boldsymbol{\pi})$, the discontinuity of $|m_j|$ from 0 to 1 depends on $\mathrm{m}_j$; the contour of stop-band(pass-band) is the boundary of level set $\{\mathrm{m}_j(\boldsymbol{\omega}) = 0\}\,(\{\mathrm{m}_j(\boldsymbol{\omega}) = 1\}$). Fig. \ref{fig: design} shows our design of the stop-band/pass-band contours of regular boundaries {\small $\big(5,6,(\frac{\pi}{2},\frac{\pi}{2})\big)$} and {\small $\big(0,5,(\pi,0)\big)$}. The contours intersect only at the vertices of $C_5$, e.g. $supp(m_5)\cap supp(m_6)\cap supp(m_0)$ contains just one point. % as the relaxed condition (1) in Proposition \ref{prop: pair-smooth}. 
Moreover, we set $\mathrm{m}_5$ to be symmetric with respect to the origin near both regular boundaries. 

The contours related to other regular boundaries are designed likewise to achieve the best symmetry; the corresponding wavelets are real. Fig.\ref{fig: design} (right) shows the frequency support of directional wavelets generated by such design; Fig.\ref{fig: many-squares}(a) shows the wavelets and scaling function in space domain. One easily checks (using Theorem \ref{thm: basis cond}) that this is an orthonormal basis.

\begin{figure}[!t]
\centering
\hspace*{-5mm}\vspace*{-2mm}
\begin{minipage}[t]{\textwidth}
\includegraphics[width=\textwidth]{many_squares_new.png}
\end{minipage}\hspace*{1mm}
\caption{ directional wavelets $\psi^1$, $\psi^2$ and scaling function $\phi$ in different constructions (a) our directional wavelet orthonormal basis, whose frequency support is shown in Fig. \ref{fig: design};(b) Durand's directional wavelet; (c) $m-$ functions of wavelets in (b). (d) our directional wavelet frame; (e) zoom in on (d); (f) $m-$ functions of wavelets in (d);  Our basis construction in (a) has good frequency localization, but slowly decaying spatial oscillation; Durand's construction in (b) has good spatial localization but non-localized frequency support; our frame construction in (d) has both good frequency localization and spatial localization. Note that plots (a),(b),(e) are at the same resolution.}
\label{fig: many-squares}
\vspace*{-3mm}
\end{figure}

Although the wavelets orient in six directions, they are not very well localized spatially, due to the singular boundaries on the corners of the low-frequency square $S_1$, where the discontinuity in the frequency domain is inevitable. The lack of smoothness at the vertices of $m_2$ and $m_5$ could possibly be avoided by using a more delicate (but more complicated) design around the vertices $(\pm\frac{\pi}{2},\pm\frac{\pi}{6})$ allowing triple overlapping of $m-$functions.% yet Proposition \ref{prop: pair-smooth} and consequently Proposition \ref{prop: M-design} are nolonger helpful.

Allowing a bit of redundancy (abandoning critical downsampling), we show next how to construct a frame with low redundancy that has much better spatial localization.
\end{comment}
%\section{Orthonormal Bases}\label{sec: orth}

In this section, we discuss the conditions on $m-$ functions such that the corresponding MRA forms an orthonormal bases. 
%consider orthonormal bases with $m-$functions defined in \eqref{eq: m0} and \eqref{eq: mj} whose supports mainly corresponding to the partition of $S_0$ in Fig.\ref{fig: partition}.
%we always consider a multi-resolution system with scaling function $\phi$ and quasi-shearlets $\psi^j$, {\small$ j = 1,\dots,6$} defined by $(M_0, D_2)$ and $(M_j, Q)$, {\small$j = 1,\dots,6$} respectively. Furthermore, the essential support of $M_j$'s corresponds to the partition of $\mathbf{S}_0$ in a shearlet system.

%The construction of \eqref{eq: MRA} reduces to design $m_0$ in \eqref{eq: m0} and $m_j, j= 1,\cdots,6$ in \eqref{eq: mj}.
We begin with the two key conditions, i.e. {\it identity summation} and {\it shift cancellation}, on $m-$functions such that the system \eqref{eq: MRA} is perfect reconstruction (PR) or equivalently a Parseval frame in MRA.% weaker than the orthonormal condition.

%\subsection{Identity summation and shift cancellation}
\subsection{orthonormal conditions on $m-$functions}
In MRA, \eqref{eq: MRA} is PR if $\forall f\in L_2(\mathbb{R}^2)$,
\begin{equation}
\textstyle \sum_k\langle f,\phi_{0,k}\rangle\phi_{0,k} = \sum_k\langle f,\phi_{1,k}\rangle\phi_{1,k} + \sum_j\sum_k\langle f,\psi^{j}_{1,k}\rangle\psi^j_{1,k}.\label{eq: PR}
\end{equation}
Using \eqref{eq: m0} and \eqref{eq: mj} together with the admissibility of the frequency partition \eqref{eq: tiling}, condition \eqref{eq: PR} on $\phi$ and $\psi^j$'s yields:
\begin{thm}\label{thm: conds}
The perfect reconstruction condition holds for \eqref{eq: MRA} iff the following two conditions hold
\begin{align}\label{eq: id-sum}
|m_0(\boldsymbol{\omega})|^2 + \sum_{j = 1}^6|m_j(\boldsymbol{\omega})|^2 = 1
\end{align}
\begin{equation}\label{eq: shift-cancel}
 \begin{cases}
%M_0(\boldsymbol{omega})\overline{M_0(\boldsymbol{omega}+\boldsymbol{\gamma})} + 
\sum_{j = 0}^6m_j(\boldsymbol{\omega})\overline{m_j(\boldsymbol{\omega} + \boldsymbol{\pi})} = 0, & \boldsymbol{\pi}\in \Gamma_0\setminus\{\boldsymbol{0}\}\\[.5em]
\sum_{j=1}^6m_j(\boldsymbol{\omega})\overline{m_j(\boldsymbol{\omega}+\boldsymbol{\pi})} = 0, & \boldsymbol{\pi}\in\Gamma_1\setminus\Gamma_0
\end{cases}
\end{equation}
%where  $\Lambda = (QD\mathbb{Z}^2)^*/(\mathbb{Z}^2)^*,\,\Gamma = (D\mathbb{Z}^2)^*/(\mathbb{Z}^2)^*.$ %$\{(\frac{\pi}{2},\frac{\pi}{2}),(\frac{3\pi}{2},\frac{\pi}{2}),(\frac{\pi}{2},\frac{3\pi}{2}),(\frac{3\pi}{2},\frac{3\pi}{2}),$ $(0,0),(0,\pi),(\pi,0),(\pi,\pi)\}$
\end{thm}

Theorem \ref{thm: conds} is a corollary of Prop. 1 and Prop. 2 in \cite{durand2007}. We give an alternate proof in Appendix \ref{app: cond-thm}.
In Theorem \ref{thm: conds}, Eq. \eqref{eq: id-sum} is the {\it identity summation} condition, guaranteeing conservation of $l_2$ energy; Eq. \eqref{eq: shift-cancel} is the {\it shift cancellation} condition such that aliasing is canceled correctly in reconstruction from wavelet coefficients. %, such that downsampling of scaling and shearlet coefficients is valid. 
 %$\Lambda = \{(\frac{\pi}{2},\frac{\pi}{2}),(\frac{3\pi}{2},\frac{\pi}{2}),(\frac{\pi}{2},\frac{3\pi}{2}),(\frac{3\pi}{2},\frac{3\pi}{2}),$ $(0,0),(0,\pi),(\pi,0),(\pi,\pi)\},\Gamma = \{(0,0),(0,\pi),(\pi,0),(\pi,\pi)\}$ are the sets of shifts associated to quincunx-dyadic dilation $QD$ and dyadic dilation $D$ respectively.
%Each $M_j$ contributes a term $M_j(\boldsymbol{omega})\overline{M_j(\boldsymbol{omega} + \boldsymbol{\nu})}$ in the cancellation condition of any shift $\boldsymbol{\nu}$ corresponding to the downsampling scheme of $M_j$.

%\subsection{Extra condition for basis}
%By Theorem \ref{thm: conds}, the system \eqref{eq: MRA} is a Parseval frame ; 
For  \eqref{eq: MRA} to be an orthonormal basis,  $\{\phi_{\boldsymbol{k}}\}_{\boldsymbol{k}\in\mathbb{Z}^2}$ need to be an orthonormal basis, which is determined by $m_0$ in \eqref{eq: phi-m0}. In 1D MRA, Cohen's theorem in \cite{cohen1992biorthogonal} provides a necessary and sufficient condition on $m_0$ such that \eqref{eq: MRA} is an orthonormal basis. %Such a condition in 1D wavelet MRA is given by Cohen \cite{cohen1992biorthogonal}. 
We generalized this theorem to 2D in \cite{yin2014orthshear}.
\begin{thm}\label{thm: basis cond}
Assume that $m_0$ is a trigonometric polynomial with $m_0(0)=1$, and define $\hat{\phi}(\boldsymbol{\omega})$ as in \eqref{eq: phi-m0}.\\
If $\phi(\cdot - \boldsymbol{k}),\boldsymbol{k}\in\mathbb{Z}^2$ are orthonormal, then $\exists K$ containing a neighborhood of 0, s.t. $\forall\boldsymbol{\omega}\in S_0,\,\boldsymbol{\omega}+2\pi\mathbf{n}\in K$ for some $\mathbf{n}\in\mathbb{Z}^2, $ and $\inf_{k>0,\,\boldsymbol{\omega}\in K}|m_0(\mathbf{D_2}^{-k}\boldsymbol{\omega})| >0$. 
 Further, if $\sum_{\boldsymbol{\V{\pi}}\in \Gamma_0} |m_0(\boldsymbol{\omega}+\boldsymbol{\pi})|^2 = 1$, then the inverse is true.
\end{thm}
%Theorem \ref{thm: basis cond} can be proved similarly to Cohen's theorem (\cite{cohen1992biorthogonal}).
%Because it is difficult to directly design $M_0$ that satisfies the conditions in Theorem \ref{thm: basis cond}, 
%Below, we construct $m-$functions imposing only \eqref{eq: id-sum} and \eqref{eq: shift-cancel} and then check if the resulting Parseval frame is an orthonormal basis by applying Theorem \ref{thm: basis cond} to $m_0$.

\subsection{$m$-function Design and Boundary Regularity}\label{sec: design}
%In this section, we define Shannon-type directional orthonormal basis same as in \cite{durand2007} and \cite{nuDFB05}. Then, we apply direct smoothing to its $m-$functions to improve spatial localization, this leads to a critical analysis of the boundary regularity of $S_1$ and the $C_j$'s.

%\subsection{Shannon-type wavelets and smoothing}
We begin with the Shannon-type wavelet construction.
Let each $m_j$ be an indicator function, $ m_j = \mathbbm{1}_{C_j},\,0\leq j \leq 6,\,$ and we use the boundary assignment of $C_j$ in Fig.\ref{fig: partition},
then the identity summation follows from the partition, and the shift cancellation follows from \eqref{eq: tiling}. % which implies $m_j(\boldsymbol{\omega})\overline{m_j(\boldsymbol{\omega} + \boldsymbol{\pi}_i)}\equiv 0,\,\forall j,\, i\neq 0.$ %Let $\partial\mathbf{C}_j = \overline{\mathbf{C}_j}-\mathbf{C}_j^{\circ}$ be the boundary of $\mathbf{C}_j$,  
Applying Theorem \ref{thm: basis cond} to $m_0$, we check that the Shannon-type wavelets generated from these $m-$functions form an orthonormal basis.

%However, because of the discontinuity of $m_j$ across the boundary of its support, the corresponding wavelet has slow decay in the time domain. 
Such $m_j$'s are not regularized due to the discontinuity on $\partial C_j$'s, the boundaries of $C_j$'s, hence the corresponding wavelets are not spatially localized. $m_j$'s can be regularized by direct smoothing on $\partial C_j$'s.
%We take a different regularization approach from Durand's \cite{durand2007}, where three regular quincunx filter banks are constructed and then composed to obtain the desired regular quincunx dyadic filter banks. Here, we smooth the discontinuous boundaries of $m-$functions directly. 
However, as shown in Proposition 3 in \cite{durand2007}, it is not possible to smooth all the boundaries with discontinuity if $m_j$'s have to satisfy the perfect reconstruction condition.
In \cite{yin2014orthshear}, $\partial C_j$'s are segmented into {\it singular} and {\it regular} pieces. On regular boundaries, pairs of $(m_j,\, m_{j'})$ that share the boundary can be smoothed in a coherent way such that all the constraints in Theorem \ref{thm: basis cond} are satisfied. Yet, on those singular pieces, it is shown that $m_j$ cannot be smoothed without violating the shift cancellation condition. Fig. \ref{fig: boundary} shows the boundary classification, where the corners of $S_0$ and $C_0$ are singular, hence $m_0$ and the diagonal $m-$ functions of an orthonormal bases are discontinuous there. A mechanism of constructing orthonormal bases by smoothing $m_j$ on regular boundaries from indicator functions is provided in \cite{yin2014orthshear}.%However, it remains unclear if some of the discontinuity can be removed by direct smoothing.

%Next, we analyze the limitation of direct smoothing in detail and show that there are regular boundaries which can be smoothed without violating \eqref{eq: id-sum} and \eqref{eq: shift-cancel}, and singular boundaries which cannot be smoothed.

%\textcolor{red}{In the remainder of this paper, we explore how this can be done for the first dyadic layer (without cutting further at higher frequencies); we call the corresponding basis {\it quasi-Shearlets}.}

\begin{figure}[!t]
\centering
%\hspace*{-5mm}
%\begin{minipage}{.6\textwidth}
%\includegraphics[width=\textwidth]{fig_overlap.png}\hspace*{-2mm}
%\end{minipage}
\begin{minipage}{.3\textwidth}
\includegraphics[width=\textwidth]{bdyclass.jpg}
\end{minipage}
%\includegraphics[width=.2\textwidth]{fig/sublattice-2.jpg}
\caption[caption]{
%\textcolor{red}{
%{\it Left top}: the supports of $m_j$ (green) and $m_j(\cdot+\V{\pi}_2$ (red) for $j = 5,6$ after smoothing, overlap on the vertical boundary at $\omega_1 =  \pm \pi/2$ of $C_5$ (green) and its shift (red) by $\boldsymbol{\pi}_2 = (\pi,0)$. Note that two copies of shifted $S_0$ (red) overlap the un-shifted $S_0$(green) due to the $(2\pi,2\pi)$ periodicity of $m_j$. Only $m_0$ and $m_5$ have overlapping smoothed boundaries by $\boldsymbol{\pi}_2$. 
%\\\hspace{\textwidth}
%{\it Left bottom}: intersection of $\mathcal{B}(0,\boldsymbol{\pi}_2)$ and $\mathcal{B}(5,\boldsymbol{\pi}_2)$ in yellow and $\mathcal{C}(0,\boldsymbol{\pi}_2) = \mathcal{B}(0,\boldsymbol{\pi}_2)\setminus\mathcal{B}(5,\boldsymbol{\pi}_2)$ in red. Smoothing $m_0$ in the red (singular) region is impossible without violating \eqref{eq: shift-cancel}. %This leads to the distinction between regular (yellow) and singular (red) boundaries at $omega_1 = \pm\pi/2$.
%\\\hspace{\textwidth}
%{\it Right}: 
Boundary classification, singular (red) and regular (yellow) }%after similar arguments for all shifts $\boldsymbol{\pi}_i$.}
 %}
\label{fig: boundary}
\end{figure}

\begin{comment}
%\subsection{Boundary classification}
 After smoothing the $m_j$'s, the shift cancellation \eqref{eq: shift-cancel} may fail to hold as $supp(m_j)$ and $(supp(m_j)-\boldsymbol{\pi})$ may overlap near the smoothed boundaries, see Fig. \ref{fig: boundary}, illustrating $m_5(\boldsymbol{\omega})\overline{m_5(\boldsymbol{\omega} + \boldsymbol{\pi}_2)}\not\equiv 0.$
For simplicity, we introduce the following notations: let 
$\mathcal{B}(j,\boldsymbol{\pi}) =  supp(m_j)\cap\,(supp(m_j)-\boldsymbol{\pi})$ be the support of  $m_j(\boldsymbol{\omega})\overline{m_j(\boldsymbol{\omega} + \boldsymbol{\pi})}$ associated to $m_j$ and shift $\boldsymbol{\pi}$;
let $\mathcal{C}(j,\boldsymbol{\pi}) = \mathcal{B}(j,\boldsymbol{\pi}) \setminus \bigcup_{j'\neq j}\mathcal{B}(j',\boldsymbol{\pi})$. 
%In addition, we may introduce the (slight abuse of) notation $C_0 = S_1$.
\begin{lem}\label{lem: singular-bdy}
Shift cancellation \eqref{eq: shift-cancel} can hold for shift $\boldsymbol{\pi}\in\Gamma_0\setminus\{\boldsymbol{0}\}$, only if $\mathcal{C}(j,\boldsymbol{\pi})=\varnothing,\, \forall\, 0\leq j\leq 6$; 
it can hold for shift $\boldsymbol{\pi}\in\Gamma_1\setminus\Gamma_0$, only if $\mathcal{C}(j,\boldsymbol{\pi})=\varnothing,\, \forall\, 1\leq j\leq 6$. 
\end{lem}
\noindent{\it Proof.} Observe that, on $\mathcal{C}(j,\boldsymbol{\pi})$, $m_j(\boldsymbol{\omega})\overline{m_{j}(\boldsymbol{\omega}+\boldsymbol{\pi})} \not\equiv 0$ but $m_{j'}(\boldsymbol{\omega})\overline{m_{j'}(\boldsymbol{\omega}+\boldsymbol{\pi})} \equiv 0,\, \forall j' \neq j$, hence \eqref{eq: shift-cancel} doesn't hold. $\hfill\square$
%Furthermore, the identity summation constraint \eqref{eq: id-sum} implies that $|M_j(\boldsymbol{omega})| = 1$ on $\mathcal{C}(j,\boldsymbol{\nu}),\,\forall\boldsymbol{\nu}$, hence the discontinuity of $M$-functions across these boundaries is unavoidable.
 
Therefore, boundaries that after smoothing make $\mathcal{C}(j,\boldsymbol{\pi})$ non-empty are called {\it singular}; the rest are {\it regular}. We next provide an explicit boundary classification method:
\begin{prop}\label{prop: class-bdy}
$\forall\, 1\leq j\leq 6$, let $supp(m_j) = \ov{$C_j$}$, then the boundary of $C_j$ is $\partial C_j=\bigcup_{i \neq 0}\mathcal{B}(j,\boldsymbol{\pi}_i)$. The set of singular boundaries of $\partial C_j$ is $\bigcup_{i\neq 0}\mathcal{C}(j,\boldsymbol{\pi}_i)$, whereas its compliment set is the regular boundary set.
 \end{prop}
\noindent{\it Proof.} Since $\bigcup_i(C_j+\boldsymbol{\pi}_i) = S_0$, $\partial C_j\subset \bigcup_{i\neq 0} (\ov{$C_j$}+\boldsymbol{\pi}_i) $. Therefore, $\partial C_j\subset\bigcup_{i\neq 0}\mathcal{B}(j,\boldsymbol{\pi}_i)$. On the other hand, $\mathcal{B}(j,\boldsymbol{\pi}_i)\subset \partial\mathbf{C}_j, \forall \, i\neq 0$, hence the union of them is a subset of $\partial C_j$. It follows that $\mathcal{B}(j,\boldsymbol{\pi}_i)$ form a partition of the boundary $\partial C_j$. The partition of $\partial C_j$ into singular and regular boundaries follows from Lemma \ref{lem: singular-bdy}.$\hfill\square$

 The case of $\partial S_1$ is similar where $\mathcal{B}(0,\boldsymbol{\pi}),\,\boldsymbol{\pi}\in\Gamma_0\setminus\{0\}$ are considered. We use the notation $\mathcal{B}_s(j,\boldsymbol{\pi}),\mathcal{C}_s(j,\boldsymbol{\pi})$ for the special case $supp(m_j) = \ov{$C_j$}$ hereafter.
The boundary classification based on Proposition \ref{prop: class-bdy} is shown in the right of Fig. \ref{fig: boundary}, where the boundaries on the four corners of both $S_0$ and $S_1$ are singular: smoothing is then not allowed there.

\begin{prop}\label{prop: jump}
Let $\mathcal{C} = \mathcal{C}_s(j_1,\boldsymbol{\pi}_{i_1})\cap \mathcal{C}_s(j_2,\boldsymbol{\pi}_{i_2})$, if $m_{j_1}, m_{j_2}$ satisfy \eqref{eq: id-sum}, \eqref{eq: shift-cancel}, then $|m_{j_1}|=\mathbbm{1}_{\mathbf{C}_{j_1}},\,|m_{j_2}|=\mathbbm{1}_{\mathbf{C}_{j_2}}$ on $\mathcal{C}$.
\end{prop}
\noindent{\it Proof.} Suppose the common singular boundary $\mathcal{C}$ is non-empty and observe that $\mathcal{C}\subset(C_{j_1})\cap(C_{j_2})$. Since $m_{j_1}$ cannot be smoothed on $\mathcal{C}_s(j_1,\boldsymbol{\pi}_{i_1})$, $|m_{j_1}| = 0$ on $\mathcal{C}_s(j_1,\boldsymbol{\pi}_{i_1})\setminus\mathbf{C}_{j_1}$, 
%Because $\mathcal{C}\subset\mathcal{C}_s(j_1,\boldsymbol{\nu}_1)\cap\partial\mathbf{C}_{j_2}$, 
and \eqref{eq: id-sum} implies that $|m_{j_2}| = 1$ there, or equivalently $|m_{j_2}|=\mathbbm{1}_{\mathbf{C}_{j_2}}$ on $\mathcal{C}$. Similarly, $|m_{j_1}|=\mathbbm{1}_{\mathbf{C}_{j_1}}$ on $\mathcal{C}.\hfill\square$

Prop. \ref{prop: jump} shows that if $m_j$ and $m_{j'}$ have common singular boundaries, then both will have a discontinuity across those boundaries. For example, $\mathcal{C}_s(0,(\pi,0))\cap\mathcal{C}_s(4,(\pi/2,\pi/2)) = (\pi/2,(\pi/6,\pi/2))$, hence $m_0$ and $m_4$ both are discontinuous at $(\pi/2,(\pi/6,\pi/2))$. All the singular boundaries related to \eqref{eq: MRA} are such "double" singular boundaries.

\subsection{Pairwise smoothing of regular boundary}
%Despite of the singular boundaries, better spatial localization can be achieved by carefully smoothing the regular boundaries. 
The regular boundaries of both $C_{j_1}$ and $C_{j_2}$ with adjacent supports consist of $\mathcal{B}_s(j_1,\boldsymbol{\pi})\cap\mathcal{B}_s(j_2,\boldsymbol{\pi})$, which we denote by the triple $(j_1,j_2,\boldsymbol{\pi})$. The following proposition shows that the regular boundaries $(j_1,j_2,\boldsymbol{\pi})$ can be paired according to shift pairs $(\boldsymbol{\pi},-\boldsymbol{\pi})$, and the boundaries must be smoothed pairwise within their $\epsilon-$neighborhood, $\mathcal{B}_{\epsilon}(j_1,j_2,\boldsymbol{\pi})$ and $\mathcal{B}_{\epsilon}(j_1,j_2,-\boldsymbol{\pi})$.

\begin{prop}\label{prop: pair-smooth}
Given $(j_1,j_2,\boldsymbol{\pi})\neq \varnothing$, %let $\boldsymbol{\nu}'= -\boldsymbol{\nu}\in \mathbb{R}^2/(\mathbb{Z}^2)^*$, 
then $(j_1,j_2,-\boldsymbol{\pi})\neq\varnothing$. In addition, let $\mathcal{B}=\mathcal{B}_{\epsilon}(j_1,j_2,\boldsymbol{\pi})\cup\mathcal{B}_{\epsilon}(j_1,j_2,-\boldsymbol{\pi})$. Then the identity summation and shift cancellation conditions hold if
\begin{itemize}
\item[{\it (i)}] $m_j = \mathbbm{1}_{C_j},\quad\text{on }S_0,\; j\neq j_1,j_2$
\item[{\it (ii)}] $m_{j_1} =\mathbbm{1}_{C_{j_1}},\, m_{j_2} =\mathbbm{1}_{C_{j_2}},\quad \text{on }\mathcal{B}^c$\\[.5em]
\hspace*{-2em} and on $\mathcal{B}$ the following hold%\vspace*{.1em}
\item[{\it(iii)}] $|m_{j_1}|^2 + |m_{j_2}|^2 = 1,$ %\hfill on $ \mathcal{B}_{\epsilon}(j_1,j_2,\nu)\cup\mathcal{B}_{\epsilon}(j_1,j_2,\nu')$
\item[{\it (iv)}] $\sum_{j_1,j_2} m_j(\cdot)\overline{m_j(\cdot+\widetilde{\boldsymbol{\pi}})} = 0,\, \widetilde{\boldsymbol{\pi}} = \pm\boldsymbol{\pi}$
%\hspace*{8em} on $\mathcal{B}_{\epsilon}(j_1,j_2,\nu)$ and $\mathcal{B}_{\epsilon}(j_1,j_2,\nu)-\nu$
%\item[{\it (v)}] $\sum_{j_1,j_2} M_j(\cdot)\overline{M_j(\cdot+\boldsymbol{\nu}')} = 0.$ 
%\hspace*{7em} on $\mathcal{B}_{\epsilon}(j_1,j_2,\boldsymbol{\nu}')$ and $\mathcal{B}_{\epsilon}(j_1,j_2,\nu')-\nu'$
\end{itemize}
\end{prop}
\noindent{\it Proof}.
 We first show that $(j_1,j_2,-\boldsymbol{\pi})\neq\varnothing.$ By definition, $\mathcal{B}_s(j_1,\boldsymbol{\pi}) = supp(m_{j_1})\cap (supp(m_{j_1})-\boldsymbol{\pi}) = \left((supp(m_{j_1})+\boldsymbol{\pi})\cap supp(m_{j_1})\right)-\boldsymbol{\pi} = \mathcal{B}_s(j_1,-\boldsymbol{\pi})-\boldsymbol{\pi}.$ Rewrite $(j_1,j_2,\boldsymbol{\pi})$ by $\mathcal{B}_s(j_1,-\boldsymbol{\pi})$ and $\mathcal{B}_s(j_2,-\boldsymbol{\pi})$, we have $(j_1,j_2,-\boldsymbol{\pi}) = (j_1,j_2,\boldsymbol{\pi})+\boldsymbol{\pi}$, hence it's non-empty.
 
Because $(j_1,j_2,\pm\boldsymbol{\pi})\subset (\partial C_{j_1} \bigcap \partial C_{j_2})$ and $(\bigcup_{j}C_j)=S_0$, $(\bigcup_{j\neq j_1,j_2}C_j)\cap \mathcal{B} = \varnothing.$ Therefore, smoothing of $m_{j_1}$ and $m_{j_2}$ in $\mathcal{B}$ doesn't impact the region where other $m_j$'s are supported.

We then show that the cancellation conditions \eqref{eq: shift-cancel} hold for all shifts. Condition (i) and (ii) imply that $\mathcal{B}(j,\widetilde{\boldsymbol{\pi}})= \varnothing,\,\forall j,\widetilde{\boldsymbol{\pi}}\neq \pm \boldsymbol{\pi}$, hence \eqref{eq: shift-cancel} hold for $\widetilde{\boldsymbol{\pi}}\neq \pm \boldsymbol{\pi}$. %For $\boldsymbol{\nu}$, 
(i) implies $\mathcal{B}(j,\widetilde{\boldsymbol{\pi}}) = \varnothing,\,\forall j\neq j_1,j_2$, so then \eqref{eq: shift-cancel} is equivalent to (iv). %Similarly, \eqref{eq: shift-cancel} is equivalent to (v) under (i).
The identity summation \eqref{eq: id-sum} holds due to (i), (ii) and (iii).$\hfill\square$.

By Prop. \ref{prop: pair-smooth} we can smooth some pairs of regular boundaries starting from the Shannon-type directional wavelets with the simplified conditions (iii), (iv) and (v); (i) and (ii) can be removed as long as the initial $m_j$ satisfy \eqref{eq: id-sum} and \eqref{eq: shift-cancel} and every $\boldsymbol{\omega}\in S_0$ is not covered by more than two $m$ functions. We can thus smooth regular boundaries pairwise, one by one.

The next proposition gives an explicit design of $(m_{j_1}, m_{j_2})$ satisfying the simplified conditions (iii),(iv) in Proposition \ref{prop: pair-smooth}. %as well as a necessary condition for any valid design.
\begin{prop}\label{prop: m-design}
Let $C\subset S_0$, given $m_{j_1},m_{j_2}\neq 0$ continuous on $ C\cup(C+\boldsymbol{\pi})$, satisfying the following conditions
\begin{itemize}
\item[(i)] $\sum_{j_1,j_2}m_{j}(\boldsymbol{\omega})\overline{m_{j}(\boldsymbol{\omega}+\boldsymbol{\pi})} = 0$ \hspace*{2em} on $C$
\item[(ii)] $\sum_{j_1,j_2}|m_{j}(\boldsymbol{\omega})|^2= 1 $ \hspace*{5em} on $C\cup (C+\boldsymbol{\pi})$
\item[(iii)] $m_{j_1}(\boldsymbol{\omega})m_{j_2}(\boldsymbol{\omega}) = 0$\hspace*{5em} on $\partial C$ ;
\end{itemize}
then
$\quad|m_{j_1}(\boldsymbol{\omega})| = |m_{j_2}(\boldsymbol{\omega}+\boldsymbol{\pi})|,\quad |m_{j_2}(\boldsymbol{\omega})| = |m_{j_1}(\boldsymbol{\omega}+\boldsymbol{\pi})|.$\\[1mm]
Furthermore, if $m_{j} = e^{i\boldsymbol{\omega}^{T}\boldsymbol{\eta}_{j}}\mathrm{m}_{j},\quad j=j_1,j_2, \text{ on }C,$ where $\mathrm{m}_j$ is a real-valued function, % $\mathcal{M}_{j_1}$ and $\mathcal{M}_{j_2}$,  phase $\eta_1,\eta_2$ s.t.
$e^{i\boldsymbol{\pi}^T(\boldsymbol{\eta}_{j1}-\boldsymbol{\eta}_{j2})} = -1,$ and 
\[\mathrm{m}_{j_1}(\boldsymbol{\omega}) = \mathrm{m}_{j_2}(\boldsymbol{\omega}-\boldsymbol{\pi}),\;\mathrm{m}_{j_2}(\boldsymbol{\omega}) = \mathrm{m}_{j_1}(\boldsymbol{\omega}-\boldsymbol{\pi}),\text{ on }C+\boldsymbol{\pi},\] 
%where $\mathcal{M}_{j1},\,\mathcal{M}_{j2}$ are 
then $(i)$ holds.
\end{prop}
\noindent{\it Proof.} 
To prove the necessary condition, note that (i) implies $|m_{j_1}(\boldsymbol{\omega})|^2|m_{j_1}(\boldsymbol{\omega+\pi})|^2 = |m_{j_2}(\boldsymbol{\omega})|^2|m_{j_2}(\boldsymbol{\omega+\pi})|^2$; the condition then follows from (ii).
For the sufficient construction, check by directly substituting the construction into (i). $\hfill\square$
%For the necessary condition, we prove in two cases. Suppose $|M_{j_1}|  = |M_{j_2}|$ on $\omega$, then (ii) implies $|M_{j_1}| = |M_{j_2}| = \frac{1}{2}$ on $\omega$. From (i), it's necessary $|M_{j_1}(\boldsymbol{omega}+\boldsymbol{\nu})|=|M_{j_2}(\boldsymbol{omega}+\boldsymbol{\nu})|$ on $\omega$, or equivalently, $|M_{j_1}| = |M_{j_2}|$ on $\omega + \boldsymbol{\nu}$. Apply (ii) again, we have $|M_{j_1}| = |M_{j_2}| = \frac{1}{2}$ on $\omega+\boldsymbol{\nu}$, therefore, $|M_{j_1}(\boldsymbol{omega})| = |M_{j_2}(\boldsymbol{omega}+\boldsymbol{\nu})|=|M_{j_2}(\boldsymbol{omega})| = |M_{j_1}(\boldsymbol{omega}+\boldsymbol{\nu})|=\frac{1}{2}.$%However, (iii) implies that either $M_{j_1}$ or $M_{j_2}$ decays to zero on the boundary, therefore, $|M_{j_1}$ and $|M_{j_2}|$ cannot be constant.

%Suppose now $|M_{j_1}(\boldsymbol{omega})| = |M_{j_2}(\boldsymbol{omega}+\boldsymbol{\nu})|$ on $\omega$, then (i) implies $|M_{j_2}(\boldsymbol{omega})| = |M_{j_1}(\boldsymbol{omega}+\boldsymbol{\nu})|$ on $\omega$.


Proposition \ref{prop: m-design} breaks down the design of $(m_{j_1},m_{j_2})$ into a pair of real functions $(\mathrm{m}_{j_1}, \mathrm{m}_{j_2})$ on $\mathcal{B}_{\epsilon}(j_1,j_2,\boldsymbol{\pi})$ and two vectors $\boldsymbol{\eta}_1,\boldsymbol{\eta}_2$; then $(\mathrm{m}_{j_1}, \mathrm{m}_{j_2})$ on $\mathcal{B}_{\epsilon}(j_1,j_2,-\boldsymbol{\pi})$ are automatically determined. %When all regular boundaries adopt this smoothing scheme,  each $\mathcal{M}_j$ has to locally match up with every $\mathcal{M}_{j'}$'s that shares a regular boundary $(j,j',\nu)$. Therefore, we may focus on 
The only constraint on $(\mathrm{m}_{j_1},\mathrm{m}_{j_2})$ for (ii) in Proposition \ref{prop: m-design} to hold is that on $\mathcal{B}_{\epsilon}(j_1,j_2,\boldsymbol{\pi})$, $\sum_{j_1,j_2}|\mathrm{m}_{j}(\boldsymbol{\omega})|^2= 1$, which is easy to be satisfied.
We may construct all local pairs of $(\mathrm{m}_{j_1},\mathrm{m}_{j_2})$ separately, and put together afterwards different pieces of each $\mathrm{m}_j$ located in different regular boundary neighborhoods $\mathcal{B}_{\epsilon}(j,j',\boldsymbol{\pi})$. 

\vspace*{.2em}
%The phase term $e^{i\boldsymbol{omega}^T\boldsymbol{\eta}_j}$ is preferably defined on the full frequency domain, hence $\boldsymbol{\eta}_j$'s need to be solved globally. This global phase problem is stated precisely in t
The next proposition gives % conditions for the $\boldsymbol{\eta}_j$ as well as 
one solution, easy to verify.
\begin{prop}\label{prop: phase}
Applying Proposition \ref{prop: m-design} to all regular boundaries requires a set of phases $\{\boldsymbol{\eta}_j\}_{j = 0}^6,$ s.t.\[\textstyle e^{i\boldsymbol{\nu}^T(\boldsymbol{\eta}_{j1}-\boldsymbol{\eta}_{j2})} = -1, \quad\forall (j_1,j_2,\boldsymbol{\pi})\in\Delta,\]
{\small\begin{multline*}
\Delta = \Bigl\{\big(0,2,(0,\pi)\big),\, \big(0,5,(\pi,0)\big),\,\big(1,3,(\pi,0)\big),\,\big(4,6,(0,\pi)\big),\\
\big(1,6,(\pi/2, 3\pi/2)\big),\,\big(2,3,(\pi/2, 3\pi/2)\big),\,\big(4,5,(\pi/2, 3\pi/2)\big), \\
\big(3,4,(\pi/2, \pi/2)\big),\big(1,2,(\pi/2, \pi/2)\big),\,\big(5,6,(\pi/2, \pi/2)\big)\Bigr\}
\end{multline*}}
The following is a (non-unique) solution: 
{\small\begin{multline*}\boldsymbol{\eta}_0 = (0,0),\,\boldsymbol{\eta}_1 = (0,0),\,\boldsymbol{\eta}_2 = (1,1),\,\boldsymbol{\eta}_3 = (1,-1),\\
\boldsymbol{\eta}_4 = (0,2),\,\boldsymbol{\eta}_5=(1,1),\,\boldsymbol{\eta}_6 = (-1,1).
\end{multline*}}
\end{prop}

To summarize, Proposition \ref{prop: m-design} and \ref{prop: phase} introduce the following regular boundary smoothing scheme for the $m$ functions:
\begin{description}% prevent items from splitting
\item[construction of orthonormal basis]\
\begin{itemize}
\item[1.] First, set $\mathrm{m}_j = \mathbbm{1}_{C_j}$; then smoothen these across a pair of regular boundaries $(j_1,j_2,\pm\boldsymbol{\pi})$ following steps 2, 3.
\item[2.]  On $\mathcal{B}_{\epsilon}(j_1,j_2,\boldsymbol{\pi})$,\\
\hspace*{2em} design $(\mathrm{m}_{j_1},\mathrm{m}_{j_2}),\quad$ s.t.
$\sum_{j_1,j_2}|\mathrm{m}_{j}(\boldsymbol{\omega})|^2= 1$.
\item[3.] On $\mathcal{B}_{\epsilon}(j_1,j_2,-\boldsymbol{\pi})$, \\
\hspace*{2em}let $\mathrm{m}_{j_1}(\boldsymbol{\omega}) = \mathrm{m}_{j_2}(\boldsymbol{\omega}-\boldsymbol{\pi})$, $\mathrm{m}_{j_2}(\boldsymbol{\omega}) = \mathrm{m}_{j_1}(\boldsymbol{\omega}-\boldsymbol{\pi})$\vspace*{.1em}
\item[4.] Repeat step 2 and 3 for all $(j_1,j_2,\boldsymbol{\pi})\in\Delta$. 
\item[5.]$m_j(\boldsymbol{\omega}) =e^{i\boldsymbol{\omega}^T\boldsymbol{\eta}_j} \mathrm{m}_j(\boldsymbol{\omega}),$ on $S_0$, with the $\boldsymbol{\eta}_j$ of Prop. \ref{prop: phase}.
\end{itemize}
\end{description}

\begin{figure}[!t]
\centering
\includegraphics[height=.3\textwidth]{contour_design.pdf}\hspace*{2mm}
\includegraphics[height=.3\textwidth]{smsh-sh.jpg}
\caption{ Left: contour design of $supp(m_5)$, Right: frequency support $|\hat{\psi}^j|$}
\label{fig: design}
\end{figure}

%\section{Quasi-shearlet bases construction}\label{sec: bases construction}
%In this section, we present a family of quasi-shearlet bases constructed based on the $M$-function design discussed in section \ref{sec: design} using our proposed MRA framework.

We apply this to smooth all the regular boundaries except those on the boundary of $S_0$. Near a regular boundary $\mathcal{B}_{\epsilon}(j,j',\boldsymbol{\pi})$, the discontinuity of $|m_j|$ from 0 to 1 depends on $\mathrm{m}_j$; the contour of stop-band(pass-band) is the boundary of level set $\{\mathrm{m}_j(\boldsymbol{\omega}) = 0\}\,(\{\mathrm{m}_j(\boldsymbol{\omega}) = 1\}$). Fig. \ref{fig: design} shows our design of the stop-band/pass-band contours of regular boundaries {\small $\big(5,6,(\frac{\pi}{2},\frac{\pi}{2})\big)$} and {\small $\big(0,5,(\pi,0)\big)$}. The contours intersect only at the vertices of $C_5$, e.g. $supp(m_5)\cap supp(m_6)\cap supp(m_0)$ contains just one point. % as the relaxed condition (1) in Proposition \ref{prop: pair-smooth}. 
Moreover, we set $\mathrm{m}_5$ to be symmetric with respect to the origin near both regular boundaries. 

The contours related to other regular boundaries are designed likewise to achieve the best symmetry; the corresponding wavelets are real. Fig.\ref{fig: design} (right) shows the frequency support of directional wavelets generated by such design; Fig.\ref{fig: many-squares}(a) shows the wavelets and scaling function in space domain. One easily checks (using Theorem \ref{thm: basis cond}) that this is an orthonormal basis.

\begin{figure}[!t]
\centering
\hspace*{-5mm}\vspace*{-2mm}
\begin{minipage}[t]{\textwidth}
\includegraphics[width=\textwidth]{many_squares_new.png}
\end{minipage}\hspace*{1mm}
\caption{ directional wavelets $\psi^1$, $\psi^2$ and scaling function $\phi$ in different constructions (a) our directional wavelet orthonormal basis, whose frequency support is shown in Fig. \ref{fig: design};(b) Durand's directional wavelet; (c) $m-$ functions of wavelets in (b). (d) our directional wavelet frame; (e) zoom in on (d); (f) $m-$ functions of wavelets in (d);  Our basis construction in (a) has good frequency localization, but slowly decaying spatial oscillation; Durand's construction in (b) has good spatial localization but non-localized frequency support; our frame construction in (d) has both good frequency localization and spatial localization. Note that plots (a),(b),(e) are at the same resolution.}
\label{fig: many-squares}
\vspace*{-3mm}
\end{figure}

Although the wavelets orient in six directions, they are not very well localized spatially, due to the singular boundaries on the corners of the low-frequency square $S_1$, where the discontinuity in the frequency domain is inevitable. The lack of smoothness at the vertices of $m_2$ and $m_5$ could possibly be avoided by using a more delicate (but more complicated) design around the vertices $(\pm\frac{\pi}{2},\pm\frac{\pi}{6})$ allowing triple overlapping of $m-$functions.% yet Proposition \ref{prop: pair-smooth} and consequently Proposition \ref{prop: M-design} are nolonger helpful.

Allowing a bit of redundancy (abandoning critical downsampling), we show next how to construct a frame with low redundancy that has much better spatial localization.
\end{comment}

%\section{Low-redundancy frame construction}\label{sec: frame}
\subsection{Extension to low-redundancy tight frame}\label{sec: frame}
%Consider the $L-$level directional wavelet MRA system
The irregularity of orthonormal bases can be overcome in the following low-redundancy tight frame construction,
 \begin{align}\label{eq: MRA-frame}
 \{\phi_{L,\boldsymbol{k}}\,,\psi^j_{l,\boldsymbol{k'}}\,, \, {\small 1\leq l \leq L,\, \boldsymbol{k},\,\boldsymbol{k'}\in \mathbb{Z}^2,\,1\leq j \leq J\}}.
\end{align}  
All wavelet coefficients are sub-sampled on dyadic sub-lattice and the redundancy of any $L-$level MRA frame doesn't exceed $\frac{J/|D|}{1-1/|D|} = \frac{6/4}{1-1/4} = 2$.
%where $\phi,\psi^j$ satisfy \eqref{eq: m0} and \eqref{eq: mj} as before. Instead of taking the dilated quincunx subsampling of directional wavelet coefficients of  \eqref{eq: MRA}, a dyadic subsampling is taken instead. A 1-level MRA frame \eqref{eq: MRA-frame} has redundancy $\frac{1}{|D|} + \frac{J}{|D|} = 1/4 + 6/4 = 7/4$, and the redundancy for any $L-$level MRA frame doesn't exceed $\frac{J/|D|}{1-1/|D|} = \frac{6/4}{1-1/4} = 2$. 
Similar to Theorem \ref{thm: conds}, we have
\begin{thm}\label{thm: frame-conds}
%Set $\Gamma = (D\mathbb{Z}^2)^*/(\mathbb{Z}^2)^*.$ 
The perfect reconstruction condition holds for \eqref{eq: MRA-frame} iff the following both hold
\begin{align}
\textstyle |m_0(\boldsymbol{\omega})|^2 + \sum_{j = 1}^6|m_j(\boldsymbol{\omega})|^2 &= 1 \\
\textstyle\sum_{j = 0}^6\,m_j(\boldsymbol{\omega})\overline{m_j}(\boldsymbol{\omega} + \boldsymbol{\pi}) &= 0,\quad  \boldsymbol{\pi}\in \Gamma_0\setminus\{\boldsymbol{0}\} \label{eq: reduced-shift-cancel}
\end{align}
\end{thm}
Theorem \ref{thm: frame-conds} can be proved analogously to Theorem \ref{thm: conds}, 
but with fewer shift cancellation constraints. Following the same analysis of boundary regularity as before, we show in \cite{yin2014orthshear} that all boundaries are regular and can be smoothed properly. Hence, we were able to obtain directional wavelets with much better spatial and frequency localization than those constructed by Durand in \cite{durand2007}. 
%We can define {\it singular} boundaries as before, %and Lemma \ref{lem: singular-bdy} holds for \eqref{eq: reduced-shift-cancel} as well.
%but only $\{\mathcal{B}(j,\boldsymbol{\pi})\}_{\boldsymbol{\pi}\in\Gamma_0\setminus\{\boldsymbol{0}\}}$ need to be considered, which results in fewer singular boundaries $\{\mathcal{C}_s(j,\boldsymbol{\pi})\}_{\boldsymbol{\pi}\in\Gamma\setminus\{\boldsymbol{0}\}}$; 
% In particular, we construct a directional wavelet tight frame with redundancy of 2 by using the classical dyadic downsampling $D_2$, with shift cancellaiton constraints \eqref{eq: shift-cancel} only on set $\Gamma\setminus\{\boldsymbol{0}\}$. 
%We check that within these singular boundaries, 
%and no "double" singular boundaries now.

%This means that even though $supp(m_0)$ still cannot be extended outside of the four corners of $S_1$ due to $\mathcal{C}_s(0,(\pi,0))$ and $\mathcal{C}_s(0,(0,\pi))$, $m_1$ can penetrate into the inside of $S_1$ because $\mathcal{C}_s(1,(\pi/2,3\pi/2))$ is not a singular boundary in \eqref{eq: MRA-frame}. The same is true for $m_3,m_4$ and $m_6$. This makes smoothing the boundaries of $m_0$ inwards possible without violating \eqref{eq: id-sum}, see Fig. \ref{fig: many-squares}(c). At the price of double redundancy, we obtain directional wavelets with much better spatial localization; see Fig. \ref{fig: many-squares}(d)(e):
%the discontinuities of a directional wavelets basis in the frequency domain around the singular boundaries can be removed in a low redundant directional wavelet tight frame.

So far, we have considered two directional wavelet MRA systems \eqref{eq: MRA} and \eqref{eq: MRA-frame} such that the directional wavelets characterize 2D signals in six equi-angled directions. 
%The orthonormal basis we construct has better frequency localization than the one constructed by Durand in \cite{durand2007} ( see Fig. \ref{fig: design} and \ref{fig: many-squares}(b)(c)), but has long tails in certain spatial directions, unavoidable because of "double" singular boundaries. 
%By doubling the redundancy we obtain spatially well localized directional wavelets.
Furthermore, these wavelets are well localized in the frequency domain such that $supp(m_j)$ is convex and $\exists\,\epsilon\; s.t.$
\begin{align}\label{eq: no-alians}
 \sup_{\boldsymbol{\omega}'\in supp(m_j)}\inf_{\boldsymbol{\omega}\in C_j}\Vert\boldsymbol{\omega'} - \boldsymbol{\omega}\Vert < \epsilon,\quad  0\leq j\leq 6.
\end{align}
This desirable condition is hard to obtain by multi-directional filter bank assembly of several elementary filter banks.

In the next section, we analyze the more general case of directional bi-orthorgonal filters constructed with respect to the same frequency partition. 
%%\section{Low-redundancy frame construction}\label{sec: frame}
\subsection{Extension to low-redundancy tight frame}\label{sec: frame}
%Consider the $L-$level directional wavelet MRA system
The irregularity of orthonormal bases can be overcome in the following low-redundancy tight frame construction,
 \begin{align}\label{eq: MRA-frame}
 \{\phi_{L,\boldsymbol{k}}\,,\psi^j_{l,\boldsymbol{k'}}\,, \, {\small 1\leq l \leq L,\, \boldsymbol{k},\,\boldsymbol{k'}\in \mathbb{Z}^2,\,1\leq j \leq J\}}.
\end{align}  
All wavelet coefficients are sub-sampled on dyadic sub-lattice and the redundancy of any $L-$level MRA frame doesn't exceed $\frac{J/|D|}{1-1/|D|} = \frac{6/4}{1-1/4} = 2$.
%where $\phi,\psi^j$ satisfy \eqref{eq: m0} and \eqref{eq: mj} as before. Instead of taking the dilated quincunx subsampling of directional wavelet coefficients of  \eqref{eq: MRA}, a dyadic subsampling is taken instead. A 1-level MRA frame \eqref{eq: MRA-frame} has redundancy $\frac{1}{|D|} + \frac{J}{|D|} = 1/4 + 6/4 = 7/4$, and the redundancy for any $L-$level MRA frame doesn't exceed $\frac{J/|D|}{1-1/|D|} = \frac{6/4}{1-1/4} = 2$. 
Similar to Theorem \ref{thm: conds}, we have
\begin{thm}\label{thm: frame-conds}
%Set $\Gamma = (D\mathbb{Z}^2)^*/(\mathbb{Z}^2)^*.$ 
The perfect reconstruction condition holds for \eqref{eq: MRA-frame} iff the following both hold
\begin{align}
\textstyle |m_0(\boldsymbol{\omega})|^2 + \sum_{j = 1}^6|m_j(\boldsymbol{\omega})|^2 &= 1 \\
\textstyle\sum_{j = 0}^6\,m_j(\boldsymbol{\omega})\overline{m_j}(\boldsymbol{\omega} + \boldsymbol{\pi}) &= 0,\quad  \boldsymbol{\pi}\in \Gamma_0\setminus\{\boldsymbol{0}\} \label{eq: reduced-shift-cancel}
\end{align}
\end{thm}
Theorem \ref{thm: frame-conds} can be proved analogously to Theorem \ref{thm: conds}, 
but with fewer shift cancellation constraints. Following the same analysis of boundary regularity as before, we show in \cite{yin2014orthshear} that all boundaries are regular and can be smoothed properly. Hence, we were able to obtain directional wavelets with much better spatial and frequency localization than those constructed by Durand in \cite{durand2007}. 
%We can define {\it singular} boundaries as before, %and Lemma \ref{lem: singular-bdy} holds for \eqref{eq: reduced-shift-cancel} as well.
%but only $\{\mathcal{B}(j,\boldsymbol{\pi})\}_{\boldsymbol{\pi}\in\Gamma_0\setminus\{\boldsymbol{0}\}}$ need to be considered, which results in fewer singular boundaries $\{\mathcal{C}_s(j,\boldsymbol{\pi})\}_{\boldsymbol{\pi}\in\Gamma\setminus\{\boldsymbol{0}\}}$; 
% In particular, we construct a directional wavelet tight frame with redundancy of 2 by using the classical dyadic downsampling $D_2$, with shift cancellaiton constraints \eqref{eq: shift-cancel} only on set $\Gamma\setminus\{\boldsymbol{0}\}$. 
%We check that within these singular boundaries, 
%and no "double" singular boundaries now.

%This means that even though $supp(m_0)$ still cannot be extended outside of the four corners of $S_1$ due to $\mathcal{C}_s(0,(\pi,0))$ and $\mathcal{C}_s(0,(0,\pi))$, $m_1$ can penetrate into the inside of $S_1$ because $\mathcal{C}_s(1,(\pi/2,3\pi/2))$ is not a singular boundary in \eqref{eq: MRA-frame}. The same is true for $m_3,m_4$ and $m_6$. This makes smoothing the boundaries of $m_0$ inwards possible without violating \eqref{eq: id-sum}, see Fig. \ref{fig: many-squares}(c). At the price of double redundancy, we obtain directional wavelets with much better spatial localization; see Fig. \ref{fig: many-squares}(d)(e):
%the discontinuities of a directional wavelets basis in the frequency domain around the singular boundaries can be removed in a low redundant directional wavelet tight frame.

So far, we have considered two directional wavelet MRA systems \eqref{eq: MRA} and \eqref{eq: MRA-frame} such that the directional wavelets characterize 2D signals in six equi-angled directions. 
%The orthonormal basis we construct has better frequency localization than the one constructed by Durand in \cite{durand2007} ( see Fig. \ref{fig: design} and \ref{fig: many-squares}(b)(c)), but has long tails in certain spatial directions, unavoidable because of "double" singular boundaries. 
%By doubling the redundancy we obtain spatially well localized directional wavelets.
Furthermore, these wavelets are well localized in the frequency domain such that $supp(m_j)$ is convex and $\exists\,\epsilon\; s.t.$
\begin{align}\label{eq: no-alians}
 \sup_{\boldsymbol{\omega}'\in supp(m_j)}\inf_{\boldsymbol{\omega}\in C_j}\Vert\boldsymbol{\omega'} - \boldsymbol{\omega}\Vert < \epsilon,\quad  0\leq j\leq 6.
\end{align}
This desirable condition is hard to obtain by multi-directional filter bank assembly of several elementary filter banks.

In the next section, we analyze the more general case of directional bi-orthorgonal filters constructed with respect to the same frequency partition. 

\section{Bi-orthogonal Bases}\label{sec: bi-orth}
In this section, we analyze bi-orthogonal bases in the following form of MRA,
\begin{align}\label{eq: bi-orth MRA}
\{\phi_{L,\V{k}},\widetilde{\phi}_{L,\V{k}}, \psi_{l,\V{k}'}^j,\widetilde{\psi}_{l,\V{k}'}^j,\, 1\leq l\leq L,\,\V{k}\in\mathbb{Z}^2,\, \V{k}'\in\mathbf{Q}\mathbb{Z}^2,\,1\leq j\leq J \},
\end{align}
where $\phi$ and $\psi^j$ satisfy \eqref{eq: m0} and \eqref{eq: mj}, as well as $\widetilde{\phi}$ and $\widetilde{\psi^j}$, respectively,
$$\widehat{\widetilde{\phi}}(\V{D}^T\V{\omega}) = \widetilde{m_0}(\V{\omega})\widehat{\widetilde{\phi}}(\V{\omega}),\quad \widehat{\widetilde{\psi^j}}(\V{D}^T\V{\omega}) = \widetilde{m_j}(\V{\omega})\widehat{\widetilde{\phi}}(\V{\omega}).$$
For such bi-orthogonal bases, we have the similar identity summation and shift cancellation condition to those in Theorem \ref{thm: conds}.
\begin{thm}\label{thm: bi-orth conds}
The perfect reconstruction iff the following two conditions hold
\begin{align}\label{eq: id-sum 2}
m_0(\boldsymbol{\omega})\sbarm{0} + \sum_{j = 1}^6 m_j(\boldsymbol{\omega})\sbarm{j} = 1
\end{align}
\begin{equation}\label{eq: shift-cancel 2}
\begin{cases}
\sum_{j = 0}^6m_j(\boldsymbol{\omega})\overline{\widetilde{m_j}}(\boldsymbol{\omega} + \boldsymbol{\pi}) = 0, & \boldsymbol{\pi}\in \Gamma_0\setminus\{\boldsymbol{0}\}\\[.5em]
\sum_{j=1}^6m_j(\boldsymbol{\omega})\overline{\widetilde{m_j}}(\boldsymbol{\omega}+\boldsymbol{\pi}) = 0, & \boldsymbol{\pi}\in\Gamma_1\setminus\Gamma_0
\end{cases}
\end{equation}
\end{thm}
The conditions \eqref{eq: id-sum 2} and \eqref{eq: shift-cancel 2} can be combined into a linear system as follows,
\begin{align}\label{eq: LS-new}
%\overline{\M}(\V{\omega})\mathbf{m}_0(\V{\omega})=
\begin{bmatrix}
    \,\sbarm{0} & \sbarm{1} & \hdots & \sbarm{6}\;  \\
    \;0 & \sbarmp{1}{1}  & \hdots  & \sbarmp{6}{1}\; \\
    \,\sbarmp{0}{2} & \sbarmp{1}{2} & \hdots & \sbarmp{6}{2}\;\\
    \;\vdots & \vdots & \vdots & \vdots \; \\
    \;0 & \sbarmp{1}{7} & \hdots & \sbarmp{6}{7}\;
\end{bmatrix}
\begin{bmatrix}
\;\mo{0}\; \\
\;\mo{1}\; \\
\;\mo{2}\; \\
\; \vdots\; \\
\;\mo{6}\; 
\end{bmatrix} 
=
\begin{bmatrix}
1\\
0\\
0\\
\vdots \\
0
\end{bmatrix}
\end{align}
%where $\M\in\mathbb{C}^{8\times 7}$ and $\mathbf{m}_0\in\mathbb{C}^7$.
In addition, we have the following analogue of Theorem \ref{thm: basis cond}.
\begin{thm}\label{thm: basis cond 2}
Assume that $m_0, \widetilde{m_0}$ are trigonometric polynomials with $m_0(0)=\widetilde{m_0}(0) = 1$, which generate $\phi,\widetilde{\phi}$ respectively.\\
If $\phi(\cdot - \boldsymbol{k}),\widetilde{\phi}(\cdot - \boldsymbol{k}),\,\boldsymbol{k}\in\mathbb{Z}^2$ are bi-orthogonal, then $\exists K$ containing a neighborhood of 0, s.t. $\forall\boldsymbol{\omega}\in S_0,\,\boldsymbol{\omega}+2\pi\mathbf{n}\in K$ for some $\mathbf{n}\in\mathbb{Z}^2, $ and $\inf_{k>0,\,\boldsymbol{\omega}\in K}|m_0(\mathbf{D_2}^{-k}\boldsymbol{\omega})| >0$, $\inf_{k>0,\,\boldsymbol{\omega}\in K}|\widetilde{m_0}(\mathbf{D_2}^{-k}\boldsymbol{\omega})| >0$. 
 Further, if  $\sum_{\boldsymbol{\V{\pi}}\in \Gamma_0} m_0(\boldsymbol{\omega}+\boldsymbol{\pi})\sbarmp{0}{} = 1,$ then the inverse is true.
\end{thm}
By Theorem \ref{thm: basis cond 2}, $m_0$ and $\widetilde{m_0}$ need to satisfy the following identity constraint for the MRA \eqref{eq: bi-orth MRA} to be bi-orthogonal,
\begin{align}\label{eq: identity-cond}
m_0\sbarm{0} + m_0\sbarmp{0}{2} + m_0\sbarmp{0}{4} + m_0\sbarmp{0}{6} = 1.
\end{align}
In sum, the construction of a bi-orthogonal basis \eqref{eq: bi-orth MRA} is equivalent to find feasible solutions of \eqref{eq: LS-new} with constraint \eqref{eq: identity-cond}. To solve \eqref{eq: LS-new}, we use the same approach in \cite{cohen1993compactly}, which solves compactly supported symmetric bi-orthogonal filters on hexagon lattice. We next review the main scheme in \cite{cohen1993compactly} and extend it to our setup of directional wavelet filter.

\subsection{Summary of Cohen et al's construction}\label{subsec: cohen-summary}
We summerize the main setup and the approach in \cite{cohen1993compactly}. Consider a bi-orthogonal scheme consists of 3 high-pass filters $m_1,m_2$ and $m_3$ and a low-pass filter $m_0$ together with their bi-orthogonal duals $\widetilde{m_j}$, s.t.
$m_0$ is $\frac{2\pi}{3}$-rotation invariant and $m_1,\, m_2,\, m_3$ are $\frac{2\pi}{3}$-rotation co-variant.

This bi-orthogonal scheme satisfies the following linear system (
Lemma 2.2.2 in \cite{cohen1993compactly} )
\begin{align}\label{eq: LS}
\begin{bmatrix}
    \,\barm{0} & \barm{1} & \barm{2} & \barm{3}\;  \\
    \;\barmn{0}{1} & \barmn{1}{1}  & \barmn{2}{1}  & \barmn{3}{1}\; \\
    \;\vdots & \vdots & \vdots & \vdots \; \\
    \;\barmn{0}{3} & \barmn{1}{3} & \barmn{2}{3} & \barmn{3}{3}\;
\end{bmatrix}
\begin{bmatrix}
\;\mo{0}\; \\
\;\mo{1}\; \\
\;\mo{2}\; \\
\;\mo{3}\; 
\end{bmatrix} 
=
\begin{bmatrix}
1\\
0\\
0\\
0
\end{bmatrix}
\end{align}
 where $\V{\nu}_1 = (\pi,0),\V{\nu}_2 = (0,\pi),\V{\nu}_3=(\pi,\pi)$.
 Let $\widetilde{\mathbf{M}}(\V{\omega})\in\mathbb{C}^{4\times 4}$ be the matrix with $\barm{j}$ entries and $\mathbf{m}(\V{\omega})\in\mathbb{C}^4$ be the vector with $m_j$ entries in \eqref{eq: LS}, then \eqref{eq: LS} can be written as \(\widetilde{\mathbf{M}}\, \mathbf{m} (\V{\omega})= [1,0,0,0]^\top\).\\
Given $\m{1}$, $\m{2},\,\m{3}$ are determined by symmetry, and Lemma 2.2.2 in \cite{cohen1993compactly} shows that
\begin{align}\label{eq: m0-sol}
m_0(\V{\omega}) &= D^{-1}%\propto 
\left|
\begin{matrix}
    \; \sbarmn{1}{1}  & \sbarmn{2}{1}  & \sbarmn{3}{1}\; \\
    \; \sbarmn{1}{2}  & \sbarmn{2}{2}  & \sbarmn{3}{2}\; \\
    \; \sbarmn{1}{3} & \sbarmn{2}{3} & \sbarmn{3}{3}\;
\end{matrix}
\right| \notag\\
&= D^{-1}\det(\widetilde{\mathbf{M}}_{1,1}(\V{\omega})),
\end{align}
where $ D \equiv \det(\widetilde{\mathbf{M}})\in \mathbb{C}^* = \mathbb{C}\setminus\{0\}$.
%{\it Remark.} 
%For \eqref{eq: m0-sol} to hold, $m_0(\mathbf{\omega})$ and $\det(\widetilde{\mathbf{M}}_{1,1}(\V{\omega}))$ having the same phase suffices, which is implied by the symmetry of $m_0$ and $\widetilde{m_j} $'s.\\ % Both $\mo{0}$ and $\det(\widetilde{\mathbf{M}}_{1,1}(\V{\omega}))$ are $\frac{2\pi}{3}-$rotation invariant. \\
If $\widetilde{m_0}$ is solved, then $m_1,m_2$ and $m_3$ are obtained by solving the linear system \eqref{eq: LS}.
To get $\m{0}$, we solve 
\begin{align}\label{eq: bi-orth-eq}
m_0\sbarm{0} + m_0\sbarmn{0}{1} + m_0\sbarmn{0}{2} + m_0\sbarmn{0}{3} = 1
\end{align}
from expanding $det(\widetilde{\mathbf{M}})$ with respect to the first column.
According to Lemma 3.2.1 in \cite{cohen1993compactly} based on {\it Hilbert's Nullstellensatz}, \eqref{eq: bi-orth-eq} has a solution iff there does not exist $(z_1,z_2)\in (\mathbb{C}^*)^2,\, \mathbb{C}^* = \mathbb{C}\setminus\{0\}$\, s.t. $(\pm z_1,\pm z_2)$ are all 
vanishing points of the $z$-transform of $m_0$.

\subsubsection{Solving $\m{0}$}
In general, there is no efficient algorithm to solve {\it Hilbert's Nullstellensatz}, and how \eqref{eq: m0-sol} is solved exactly is not mentioned in \cite{cohen1993compactly}.

We propose an optimization approach, where \eqref{eq: m0-sol} is equivalent to a linear constraint and the objective function imposes regularity on $\widetilde{m_0}$.
On a $2N\times 2N$ grid $\G$ of $S_0 = [-\pi, \pi)\times[-\pi, \pi)$, s.t. $\forall \V{\omega}_j \in \G, \; \V{\omega}_j+\V{\nu}_1,\,\V{\omega}_j+\V{\nu}_2,\,\V{\omega}_j+\V{\nu}_3 \in \G$, \eqref{eq: m0-sol} is reformulated as
\begin{align}
\hspace*{10em} \V{A}\, \mathbf{\widetilde{m}_0}&= \mathbf{1}_{4N^2}, \label{eq: m0-A}\\ 
\mathbf{\widetilde{m}_0} = [\widetilde{m_0}(\V{\omega}_i)]_{\,i=1,\hdots,4N^2} \quad &\V{A}_{i,j} = m_0(\V{\omega}_j)\sum_{k=0}^3\delta(\V{\omega}_j-\V{\omega}_i-\V{\nu}_k) \notag
\end{align}
Because the set $\{\V{\omega},\, \V{\omega}+\V{\nu}_k,k=1,2,3\}$ is invariant under the shift $\V{\nu}_i,\, i = 1,2,3,$ the rows of $\V{A}$ corresponding to $\V{\omega}$ and $\V{\omega}+\V{\nu}_i$ are identical and we only need to consider rows corresponds to $\V{\omega}\in [-\pi,\pi)\times[-\pi,\pi)/\{\V{0},\,\V{\nu}_i,i=1,2,3\}$. Therefore, after removing the duplicate rows, $\V{A}\in \mathbb{C}^{N^2\times 4N^2}$ and \eqref{eq: m0-A} is under-determinant. \\
We thus use \eqref{eq: m0-A} as a linear constraint in quadratic optimization to solve $\mathbf{\widetilde{m}_0}$. Suppose that $\m{0}$ is smooth, then we build a differential operator $\V{D}$ and solve the following minimization problem:
\begin{align}
&\min_{\mvec{0}}\; \Vert \V{D}\mvec{0}\Vert^2,\quad s.t. \; \V{A}\mvec{0} = \mathbf{1} \label{eq: opt-diff}
\end{align}
%or
%\begin{align}
%&\min_{\mvec{0}}\; \Vert \V{D}\mvec{0}\Vert^2 + \lambda \Vert \V{A}\mvec{0} - \mathbf{1}\Vert^2 \label{eq: m0-smooth-relaxed}
%\end{align}
%The solution of \eqref{eq: m0-smooth-relaxed} is $\mvec{0} = \lambda(\lambda \V{A}^\top \V{A} + \V{D}^\top \V{D})^{-1}\V{A}^\top\mathbf{1}$.

Or suppose $\m{0}$ decays away from the origin, then we build a diagonal weighting operator $\V{W}$, and solve the following minimization problem:
\begin{align}\label{eq: opt-weight}
&\min_{\mvec{0}}\; \Vert \V{W}\mvec{0}\Vert^2,\quad s.t. \; \V{A}\mvec{0} = \mathbf{1}
\end{align}
Supplementary numerical results on solving $\m{0}$ by optimization are provided in Appendix \ref{app: supp-numerical}, where we test this optimization method on pre-designed bi-orthogonal filters $m_0$ and $\widetilde{m_0}$.

\section{Adaptation to dilated quincunx scheme}

\textcolor{red}{Replace by an overview of this section}

Following the same approach of Cohen et al, we focus on solving $m_i$'s and $\widetilde{m_0}$ in \eqref{eq: LS-new} given pre-designed $\m{i},\,i=1,\cdots,6$. %Assume $\m{i},\,i=1,\cdots,6$ satisfy weak constraints on the direction selectivity of their support.

For simplicity, in the rest of this paper, we reuse notations for \eqref{eq: LS} to express \eqref{eq: LS-new} in the same form as $\widetilde{\mathbf{M}}\mathbf{m}(\V{\omega}) = [1,0,\cdots,0]^T$, where $\widetilde{\mathbf{M}}(\V{\omega})\in\mathbb{C}^{8\times 7}$ and $\mathbf{m}(\V{\omega})\in\mathbb{C}^7$. In addition, for a matrix $\mathbf{A}$, we denote its sub-matrix containing all but the $k-$th row(column) as $\mathbf{A}[-k,:]\, (\mathbf{A}[:,-k])$.% and its minor with respect to its $(i,j)-$th entry as $A_{-i,-j} = \det(\mathbf{A}[-i,-j])$. 
In particular, we denote $\M[-1,-1]$ as $\Msub$.

As in Section \ref{subsec: cohen-summary}, we implicitly assume that $\M(\V{\omega})$ is full rank for any $\V{\omega}$ so that \eqref{eq: LS-new} has unique solution $\mathbf{m}$ and Cramer's rule can be applied to compute $m_0$ with respect to a non-singular sub-matrix of $\M$. That is, $\exists\, k_\omega\in\{2,\cdots,8\}$ such that $\M[-k_{\V{\omega}},:]$ is non-singular\footnote{Lemma \ref{lem: subM-singular} shows that $k_\omega\neq 1$}. Therefore, 
%there is a unique row $\M[k_{\V{\omega}},:],\,k_\omega\in\{2,\cdots,8\}$ such that removing it from $\M$ gives a non-singular square matrix $\M[-k_{\V{\omega}},:]$. By Cramer's rule, 
\begin{align}\label{eq: m0-cramer}
m_0(\V{\omega}) = \det(\Msub[-k_{\V{\omega}},:])/\det(\M[-k_{\V{\omega}},:]).
\end{align}
Instead of requiring strong symmetries of $\m{i}$'s as in Section \ref{subsec: cohen-summary}, we only ask for a minimum symmetry of $\m{i}$ such that $|\m{1}|$ and $|\m{6}|$ are symmetric with respect to the diagonal $\omega_1=\omega_2$, i.e.
$$ |\widetilde{m_1}(\V{\omega})| = |\widetilde{m_6}(\V{\omega}')|\quad \forall\, \omega_1=\omega_2', \,\omega_2=\omega_1',$$
and likewise for $\m{3}$ and $\m{4}$,
$$ |\widetilde{m_3}(\V{\omega})| = |\widetilde{m_4}(\V{\omega}')|\quad \forall\, \omega_1=-\omega_2', \,\omega_2=-\omega_1'.$$

In the following subsections, we first show our main result that for \eqref{eq: LS-new} to be solvable, the pre-designed $\widetilde{m_i}$'s are discontinuous. We then discuss how to design $\widetilde{m_i}$'s and solve the corresponding system \eqref{eq: LS-new}.

\subsection{Singularity of $\M[-1,:]$ and discontinuity of $\m{i}$}

%Same as in Section \ref{subsec: cohen-summary} , we first compute $m_0$ and assume that $\M$ is full rank, otherwise \eqref{eq: LS-new} has infinitely many solutions. Moreover, $\M[2:8,:]$ is singular. 
\begin{lemma}\label{lem: subM-singular}
If \eqref{eq: LS-new} is solvable, then $\M[-1,:](\V{\omega})$ is singular $\forall \V{\omega}$.
\end{lemma}
\noindent{\it Proof.}
If \eqref{eq: LS-new} has a solution, then $\forall \V{\omega}$,  $[1,0,\cdots,0]^\top\in \mathbb{R}^8$ is a linear combination of the columns of $\M$ hence the solution $\mathbf{m} \in Null(\M[-1,:])$ and it is non-zero. This implies that $\M[-1,:]$ is singular.\qed\\[1em]
Let $\mrow{i}(\V{\omega}) = [\widetilde{m_1}(\V{\omega}+\V{\pi}_i)\, \cdots,\,\widetilde{m_6}(\V{\omega}+\V{\pi}_i)]\in\mathbb{C}^6,\, i = 0,\cdots,7$ be the rows of $\M[:,-1]$, and define $$d_{i,j}(\V{\omega}) = \det([\mrow{k_1}(\V{\omega})^\top,\cdots,\mrow{k_6}(\V{\omega})^\top]),\;$$ where $0\leq k_1<\cdots<k_6\leq 7,\, s.t.\,z k_l\neq i,j.$
\begin{lemma}\label{lem: subM-singular-sys}
$\M[-1,:](\V{\omega})$ is singular $\forall \V{\omega}$ if and only if \vspace{.5em}
\begin{align}
\label{eq: singular-cond}
\mathfrak{D}(\omega)\begin{bmatrix}
\m{0}\\
\mp{0}{2}\\
\mp{0}{4}\\
\mp{0}{6}
\end{bmatrix}
\doteq
\begin{bmatrix}
0 & d_{0,2} & d_{0,4} & d_{0,6}\\
-d_{0,2} & 0 & d_{2,4} & d_{2,6}\\
-d_{0,4} & -d_{2,4} & 0 & d_{4,6}\\
-d_{0,6} & -d_{2,6} & -d_{4,6} & 0
\end{bmatrix}
\begin{bmatrix}
\m{0}\\
\mp{0}{2}\\
\mp{0}{4}\\
\mp{0}{6}
\end{bmatrix}
= \begin{bmatrix}
0\\0\\0\\0
\end{bmatrix}.
\end{align}
\end{lemma}
\noindent{\it Proof.}
The singularity condition on  $\M[-1,:](\V{\omega})$ can be rewritten as follows,
\begin{align}\label{eq: singular-omega}
0 &=\det(\M[-1,:]) \notag\\
&=  \widetilde{m_0}(\V{\omega}+\V{\pi}_2)\cdot\det(\Msub[-2,:])\notag\\
&\quad+ \,\widetilde{m_0}(\V{\omega}\,+\,\V{\pi}_4)\cdot\det(\Msub[-4,:])
+ \widetilde{m_0}(\V{\omega}+\V{\pi}_6)\cdot\det(\Msub[-6,:])\notag\\
&= 0\cdot\widetilde{m_0}(\V{\omega})\,+\,d_{0,2}\cdot\widetilde{m_0}(\V{\omega}+\V{\pi}_2) \notag\\
&\quad+\,d_{0,4}\cdot \widetilde{m_0}(\V{\omega} + \V{\pi}_4)\,+\, d_{0,6}\cdot\widetilde{m_0}(\V{\omega} + \V{\pi}_6) 
\end{align}
This is the first equation in the linear system \eqref{eq: singular-cond}. Substitute $\V{\omega}$ by $\V{\omega + \pi_2}$ in \eqref{eq: singular-omega} and use the $2\pi-$periodicity of $\V{\omega}$, we have the singularity condition on $\M[-1,:](\V{\omega+\pi_2})$ as follows
%then the above singularity condition on $\M[-1,:]$ at $\V{\omega}$ can be rewritten as follows,
%\begin{align*}
%[0,\, d_{0,2}(\V{\omega}),\, d_{0,4}(\V{\omega}),\, d_{0,6}(\V{\omega})]\,[\widetilde{m_0}(\V{\omega}),\,\widetilde{m_0}(\V{\omega}+\V{\pi}_2),\, \widetilde{m_0}(\V{\omega}+\V{\pi}_4),\,\widetilde{m_0}(\V{\omega}+\V{\pi}_6)]^\top = 0
%\end{align*}
%It is easy to verify that the above singular condition at $\V{\omega}+\V{\pi}_2$ is equivalent to 
\begin{align*}
-d_{0,2}(\V{\omega})\cdot \widetilde{m_0}(\V{\omega}) + d_{2,4}(\V{\omega})\cdot \widetilde{m_0}(\V{\omega}+\V{\pi}_4) + d_{2,6}(\V{\omega})\cdot\widetilde{m_0}(\V{\omega}+\V{\pi}_6) = 0,
%[-d_{0,2}(\V{\omega}),\, 0,\,d_{2,4}(\V{\omega}),\,d_{2,6}][\widetilde{m_0}(\V{\omega}),\,\widetilde{m_0}(\V{\omega}+\V{\pi}_2),\, \widetilde{m_0}(\V{\omega}+\V{\pi}_4),\,\widetilde{m_0}(\V{\omega}+\V{\pi}_6)]^\top = 0,
\end{align*}
which is the second linear equation in  \eqref{eq: singular-cond}.
Similarly, the third and fourth equations can be obtained by rewriting the singularity condition at $\V{\omega}+\V{\pi}_4$ and $\V{\omega}+\V{\pi}_6$ in the coordinate of $\V{\omega}$.\qed
%where $\mathfrak{D}(\V{\omega})$ is anti-symmetric. Because $\mathfrak{D}(\V{\omega})$ is independent of $m_0(\V{\omega})$, \eqref{eq: singular-cond} holds for $\mc{0}$ as well.

%On the other hand, given $m_0$, $\widetilde{m_0}$ has to satisfy the identity constraint \eqref{eq: identity-cond}.
The identity constraint \eqref{eq: identity-cond} on $m_0$ and the singularity condition \eqref{eq: singular-cond} together imply the following proposition,
%Due to the periodic wrapping of the frequency square $S_0$, we only need to consider \eqref{eq: singular-cond} and \eqref{eq: identity-cond} on $S_1$ and they imply the following proposition,
\begin{proposition}\label{prop: feasibility}
Given $\widetilde{m_i}, i = 1,\cdots,6$, \eqref{eq: LS-new} has no solution for $\widetilde{m_0}$, if $\exists\,\omega, \,s.t. \; [m_0(\omega), m_0(\omega+\pi_2),m_0(\omega+\pi_4),m_0(\omega+\pi_6)]$ is a linear combination of the conjugate of the rows of $\mathfrak{D}(\omega)$.% in \eqref{eq: singular-cond}.
\end{proposition}
%Proposition \ref{prop: feasibility} provides a necessary condition such that the numerical optimization solving $\widetilde{m_0}$ is feasible.
\begin{figure}
\centering
\includegraphics[width = .4\textwidth]{triangle-partition-new.png}
\caption{Partition of frequency square in six directions, where the essential support of $\m{i}$ is contained in each pair of triangles $T_i$. The pair of dark grey triangles is $T_1^-$ and the light grey pair is $T_1^+$.}
\label{fig: partition 2}
\end{figure}
%Let pairs of triangles $T_i$ in Fig.\ref{fig: partition 2} contain the essential support of $\widetilde{m_i},\,i=1,\cdots,6$.
%\eqref{eq: LS-new} takes a similar form to \eqref{eq: LS}, but with $\M\in\mathbb{C}^{8\times 7}$, which is an over-determinant linear system.

\noindent{\bf Definition.}
The {\it essential support} $\Omega_i$ of a function $\widetilde{m_i}$ is the set $\{\V{\omega}:\,|\widetilde{m_i}(\V{\omega})|> |\widetilde{m_j}(\V{\omega})|,\,\forall j\neq i\}$. \vspace{.5em}

Let $T_i$ be pairs of triangles shown in Figure \ref{fig: partition 2}, such that $C_i\subset T_i,\, i = 1,\cdots,6.$ Consider its decomposition, $T_i = T_i^-\bigcup T_i^+$, where $T_i^-, T_i^+$ are halves of $T_i$ adjacent  to $T_{i-1}$ and $T_{i+1}$ respectively.\\[.5em]
\noindent{\bf Definition.}  $\widetilde{m_i}$ {\it concentrates} within cone $T_i$ if 
\begin{itemize}
\item[(i)] $\Omega_i\subset T_i$;
\item[(ii)]$\text{supp}(\widetilde{m_i})\subset T_{i-1}^+\bigcup T_i\bigcup T_{i+1}^-$ and $\int_\Omega|\widetilde{m_i}| > \int_{\Omega'}|\widetilde{m_i}|, \forall\, \Omega\subset T_i\bigcap\text{supp}(\widetilde{m_i})$, where $\Omega' \subset T_{i-1}^+\bigcup T_{i+1}^-$ is symmetric to $\Omega$ with respect to the boundary of $T_i$.
\end{itemize}

\begin{figure}
\centering
\includegraphics[width = .5\textwidth]{S_shifts2.png}
\caption{$S_{\rho}$ and its shifts}
\label{fig: S-shifts}
\end{figure}
Given $\m{i}$ that concentrates in $T_i$, we study the feasibility condition in Proposition \ref{prop: feasibility} specifically on the domain $S_{\rho} = \{(\omega_1,\omega_2)|\;\Vert\omega\Vert < \rho, \omega_1 <0,\,\omega_2<0\}$, see Figure \ref{fig: S-shifts}. 

\begin{lemma}\label{lem: rank1}
$\exists\, \rho>0$ s.t. $\forall \omega\in S_\rho$, $rank(\mrow{1},\mrow{7})=1$ or $rank(\mrow{3},\mrow{5}) = 1$.
\end{lemma}
\noindent{\it Proof.}
When $\rho$ is small enough, due to the concentration property, $\m{i}$ is zero on all but a few sets $S_\rho + \V{\pi}_j$ (see Fig.\ref{fig: S-shifts} for reference of $S_\rho$ and its shifts), thus $\mrow{i}(\V{\omega})$ is sparse on $S_\rho$ and $\M[-1,:]$ takes the following form
\begin{align}
\label{eq: sparse-mat}
\M[-1,:](\V{\omega})=
\begin{bmatrix}
\mrow{0}\\
\mrow{1}\\
\mrow{2}\\
\mrow{3}\\
\mrow{4}\\
\mrow{5}\\
\mrow{6}\\
\mrow{7}
\end{bmatrix}
=
\begin{bmatrix}
0 & 0 & 0 & 0 & 0 & 0\\
* & 0 & 0 & 0 & 0 & *\\
0 & 0 & 0 & * & * & 0\\
0 & 0 & * & * & 0 & 0\\
0 & * & * & 0 & 0 & 0\\
0 & 0 & * & * & 0 & 0\\
* & 0 & * & * & 0 & *\\
%0 & * & * & 0 & 0 & 0\\
%0 & 0 & 0 & * & * & 0\\
%* & 0 & 0 & 0 & 0 & *\\
%* & 0 & 0 & 0 & 0 & *\\
%0 & 0 & * & * & 0 & 0\\
%0 & 0 & * & * & 0 & 0\\
* & 0 & 0 & 0 & 0 & *
\end{bmatrix}
%=\V{P}\,\widetilde{\mathbf{M}}[:,2:7],
\end{align}
where $*$s denote possible non-zero entries.
%where $\V{P}$ is a row permutation matrix. 
We make the following observation of $\mrow{i}$:
\begin{itemize}
\item[(i)] $\mrow{0}$ is a zero vector
\item[(ii)] $\mrow{2}$ and $\mrow{4}$ are linearly independent of each other and the rest of $\mrow{i}$
\item[(iii)] $span\{\mrow{1},\mrow{7}\} \perp span\{\mrow{3},\mrow{5}\}$ and $rank(\mrow{1},\mrow{7}) \leq 2$, \\$rank(\mrow{3},\mrow{5})\leq 2$
\item[(iv)] $span\{\mrow{1}, \mrow{7}, \mrow{3},\mrow{5},\mrow{6}\} \leq 4$
\end{itemize}
Since $S_\rho$ is in the low frequency domain, $m_0(\V{\omega})\neq 0$. \eqref{eq: m0-cramer} then implies that $\Msub$ is full rank, or equivantly, $rank(\M[-1,:]) = 6$. It follows from  (ii) and (iv) that $rank(\mrow{1},\mrow{6},\mrow{7},\mrow{3},\mrow{5})= 4$.\\
On the other hand, (ii) and (iv) imply that $$rank(\Msub(\V{\omega}+\V{\pi}_2))=rank(\mrow{0},\mrow{4},\mrow{6},\mrow{1},\mrow{3},\mrow{5},\mrow{7})= 5$$ and likewise $$rank(\Msub(\V{\omega}+\V{\pi}_4))=rank(\mrow{0},\mrow{2},\mrow{6},\mrow{1},\mrow{3},\mrow{5},\mrow{7})= 5.$$ Therefore, $\det(\Msub(\V{\omega} + \V{\pi}_2)) = \det(\Msub(\V{\omega} + \V{\pi}_4)) = 0$ and \eqref{eq: m0-cramer} implies $m_0(\V{\omega}+\V{\pi}_2) = m_0(\V{\omega}+\V{\pi}_4) = 0$.\\
If $\mrow{1}$ and $\mrow{7}$ are linearly independent and so are $\mrow{3}$ and $\mrow{5}$, then $$rank(\Msub(\V{\omega}+\V{\pi}_6))=rank(\mrow{2},\mrow{4},\mrow{1},\mrow{3},\mrow{5},\mrow{7}) = 6,$$ hence $m_0(\V{\omega}+\V{\pi}_6)\neq 0$. Therefore, $$[m_0(\V{\omega}),m_0(\V{\omega}+\V{\pi}_2),m_0(\V{\omega}+\V{\pi}_4),m_0(\V{\omega}+\V{\pi}_6)] = [*,0,0,*].$$ In addition, $d_{i,j} = 0,\, \forall(i,j)$ except $(0,6)$, so in \eqref{eq: singular-cond} $$\mathfrak{D}(\V{\omega}) = [d_{0,6}, 0, 0,0]^\top [0,0,0,1] + [0,0,0,d_{0,6}]^\top [-1,0,0,0].$$  By Proposition \ref{prop: feasibility}, the linear system \eqref{eq: LS-new} has no solution $\widetilde{m_0}$ and this proofs the lemma.\qed\\[.5em]
Without loss of generality, in the following analysis, we assume $rank(\mrow{1},\mrow{7}) = 1$ on $S_\rho$.
\begin{lemma}\label{lem: concentrate}
If $\m{1} (\m{6})$ concentrates in $T_1 (T_6)$, then $|\m{6}| > |\m{1}|\,$ a.e. on $T_6\bigcap \text{supp}(\widetilde{m_6})$ ($|\m{1}| > |\m{6}|$ 
a.e. on $T_1\bigcap\text{supp}(\widetilde{m_1})$).
\end{lemma}
\noindent{\it Proof}
Let $B_6=\{\V{\omega}: |\m{6}| \leq |\m{1}|\}\bigcap T_6\bigcap supp(\widetilde{m_1})$ and $B_1$ be its mirror set with respect to $\omega_1 = \omega_2$ and suppose $|B_6|>0$, then $\int_{B_6}|\m{6}|\leq \int_{B_6}|\m{1}|$. On the other hand, since $\m{1}$ concentrates in $T_1$, we know $\int_{B_1}|\m{1}| > \int_{B_6}|\m{1}|$. Moreover, due to the symmetry of $\m{1},\m{6}$ and $B_1,B_6$, $\int_{B_1}|\m{1}| = \int_{B_6}|\m{6}|$, hence $\int_{B_6}|\m{1}| \geq\int_{B_6}|\m{6}| = \int_{B_1}|\m{1}| $ which results in contradiction.\qed

\begin{proposition}
If  $m_1\,(m_6)$ concentrates within $T_1\,(T_6)$, then $\m{1} = \m{6} = 0,\, a.e. $ on $ S_\rho + \V{\pi}_1$.
\end{proposition}
\noindent{\it Proof.}
Consider frequency domain $S_\rho' = S_\rho\bigcap\{\omega_1<\omega_2\}.$ By Lemma \ref{lem: rank1}, $\exists\,\alpha_{\V{\omega}}\in\mathbb{C}, s.t.\,\mrow{1}(\V{\omega}) = \alpha_{\V{\omega}}\,\mrow{7}(\V{\omega}),\, \forall\, \V{\omega}\in S_\rho',$ i.e. $\widetilde{m_1}(\V{\omega} + \V{\pi}_1) = \alpha_{\V{\omega}}\cdot\widetilde{m_1}(\V{\omega} + \V{\pi}_7)$ and $\widetilde{m_6}(\V{\omega} + \V{\pi}_1) = \alpha_{\V{\omega}}\cdot\widetilde{m_6}(\V{\omega} + \V{\pi}_7)$. On the other hand, Lemma \ref{lem: concentrate} implies that $|\widetilde{m_1}(\V{\omega} + \V{\pi}_7)| \geq |\widetilde{m_6}(\V{\omega} + \V{\pi}_7)|$, hence $|\widetilde{m_1}(\V{\omega} + \V{\pi}_1)| \geq |\widetilde{m_6}(\V{\omega} + \V{\pi}_1)|$. Let $\Omega_6':= (S_\rho+\pi_1)\bigcap T_6$, then $\int_{\Omega_6'}|\m{1}| \geq\int_{\Omega_6'}|\m{6}|$, which will contradict Lemma \ref{lem: concentrate} unless $|\Omega_6'\bigcap\text{supp}(\widetilde{m_6})| = 0$, or equivalently $\alpha_{\V{\omega}}=0$ and so $\m{6} = \m{1} = 0,\,a.e.$ on $\Omega_6'$. By symmetry, $\m{6}=\m{1} = 0,\,a.e. $ on $(S_\rho+\V{\pi}_1)\setminus \Omega_6'$ as well.\qed

\begin{proposition}
$\m{1},\m{6}$ are not continuous at both $(\frac{\pi}{2},\frac{\pi}{2})$ and $(-\frac{\pi}{2},-\frac{\pi}{2})$.
\end{proposition}
\noindent{\it Proof}
If $\m{1}$ is continuous at $(\frac{\pi}{2},\frac{\pi}{2})$, then $\widetilde{m_1}(\frac{\pi}{2},\frac{\pi}{2}) = \lim_{\alpha\rightarrow 1^-}\widetilde{m_1}(\V{\omega}(\alpha)) = 0$, where $\{\V{\omega}(\alpha),\,0\leq \alpha<1\} \subset S_\rho + \V{\pi}_1$ and $\V{\omega}(1) = (\frac{\pi}{2},\frac{\pi}{2})$. By symmetry, we have $\widetilde{m_6}(\frac{\pi}{2},\frac{\pi}{2}) = 0$. Similarly, the continuity at $(-\frac{\pi}{2},-\frac{\pi}{2})$ implies $\widetilde{m_1}(-\frac{\pi}{2},-\frac{\pi}{2}) = \widetilde{m_6}(-\frac{\pi}{2},-\frac{\pi}{2}) = 0$. Therefore $\mrow{1}(0) = \mrow{7}(0) = \mathbf{0}$ which results in contradiction with Lemma \ref{lem: rank1}.\qed\\[1em]%and from \eqref{eq: m0C} $m_0^C(0)=0$ so that $m_0(0)=0$, %  On the other hand, Proposition\ref{prop: origin-det} implies that $m_0(0) = 0$ as $a = |\widetilde{m_1}(\pi_1)| = 0$, which results in contradiction.
The following theorem summarizes our main result.
\begin{theorem}\label{thm: thm}
If  $\m{i}$ concentrates in $T_i$ and $\m{1},\m{6}$ are symmetric to each other,  then  \eqref{eq: LS-new} doesn't have feasible solution given continuous $\m{1}$ and $\m{6}$.
\end{theorem}

\subsection{Design of input $\m{i}$}\label{sec: phase-design}
Following the orthonormal construction in \cite{yin2014orthshear}, we consider $\m{1},$ $\cdots,\m{6}$ in the form 
\begin{align}\label{eq: m-form}
\m{k} = e^{-i\V{\eta}_k^\top\V{\omega}}|\m{k}|,
\end{align}
 and $|\m{k}|$ have certain symmetry. We want to design the phase $\V{\eta}_k$ such that $m_0(\V{\omega}) > 0, \; \forall \omega\in S_1$. This is the same as requiring $\Msub$ to be full rank.
 We first show the necessary conditions on phases $\V{\eta}$ of the full rank requirement on $\Msub$.
 
\begin{lemma}\label{lem: phase-ineq}
If $\exists\,\V{\omega}\in D_1:=\{\omega_1=\omega_2,\,\omega_1\in(-\frac{\pi}{2},0)\},\,s.t. \,m_0(\V{\omega})>0,$ then $(\V{\eta}_1-\V{\eta}_6)^\top (\V{\pi}_6-\V{\pi}_7)\neq 0(\text{mod}\,2\pi)$. 
\end{lemma} 
\noindent {\it Proof}
  If $m_0(\V{\omega})>0, \,\V{\omega}\in D_1$ then $\Msub$ is full rank, hence its columns are linearly independent. Due to symmetry, $|\widetilde{m_1}(\V{\omega})| = |\widetilde{m_6}(\V{\omega})|$ on $\{\omega_1=\omega_2\}$. Let $A = |\widetilde{m_1}(\V{\omega}+\V{\pi}_7)| = |\widetilde{m_6}(\V{\omega}+\V{\pi}_7)|$ and $B=|\widetilde{m_1}(\V{\omega}+\V{\pi}_6)| = |\widetilde{m_6}(\V{\omega}+\V{\pi}_6)|$, then the first and the last columns of $\Msub$ are
  \begin{align*}
  \Msub[:,1] = 
 \begin{bmatrix}
 0\\
 \vdots\\
 0\\
 Ae^{i\V{\eta}_1^\top(\V{\omega}+\V{\pi}_6)}\\
 Be^{i\V{\eta}_1^\top(\V{\omega}+\V{\pi}_7)}
 \end{bmatrix}
 \quad\text{and}\quad
  \Msub[:,6] = 
 \begin{bmatrix}
 0\\
 \vdots\\
 0\\
 Ae^{i\V{\eta}_6^\top(\V{\omega}+\V{\pi}_6)}\\
 Be^{i\V{\eta}_6^\top(\V{\omega}+\V{\pi}_7)}
 \end{bmatrix} .
\end{align*}   
Therefore, $\Msub[:,1]$ and $\Msub[:,6]$ being linearly independent implies that \\$e^{i(\V{\eta}_1^\top\V{\pi}_6 + \V{\eta}_6^\top\V{\pi}_7)}\neq e^{i(\V{\eta}_6^\top\V{\pi}_6 + \V{\eta}_1^\top\V{\pi}_7)}$%$e^{i(\V{\eta}_1-\V{\eta}_6)^\top(\V{\omega}+\V{\pi}_6)}\neq e^{i(\V{\eta}_1-\V{\eta}_6)^\top(\V{\omega}+\V{\pi}_7)}$ 
or equivalently $(\V{\eta}_1-\V{\eta}_6)^\top(\V{\pi}_6-\V{\pi}_7)\neq 0(\text{mod}2\pi)$. \qed\\

Similarly, if $\exists\,\V{\omega}\in \{\omega_1 = \omega_2,\, \omega_1\in(0,\frac{\pi}{2})\},\, s.t.\, m_0(\V{\omega}) > 0$, then $(\V{\eta}_1-\V{\eta}_6)^\top (\V{\pi}_6-\V{\pi}_1)\neq 0(\text{mod}\,2\pi)$. These two conditions are equivalent to 
\begin{align*}
(\V{\eta}_1-\V{\eta}_6)^\top(\pi/2,\pi/2)\neq 0 (\text{mod}\,2\pi)\tag{\bf c1.1}
\end{align*}
given that $\V{\eta}_1$ and $\V{\eta}_6$ are integer phases in $\mathbb{Z}^2$.
Considering the other diagonal segment $\{\omega_2 = -\omega_1, |\omega_1| <\frac{\pi}{2}\}$, we have 
\begin{align*}
(\V{\eta}_3-\V{\eta}_4)^\top(-\pi/2,\pi/2)\neq 0 (\text{mod}\, 2\pi)\tag{\bf c1.2}
\end{align*}
%from the full rank condition.

Next, we investigate $\Msub$ at the origin, where the two diagonals meet.
\begin{proposition}\label{prop: origin-det}
If $|\widetilde{m_1}(\V{\pi}_1)| = |\widetilde{m_1}(\V{\pi}_7)| = |\widetilde{m_3}(\V{\pi}_3)| = |\widetilde{m_3}(\V{\pi}_5)|$ and $|\widetilde{m_1}(\V{\pi}_6)|= | \widetilde{m_3}(\V{\pi}_6)|$, then $\V{\pi}_1^\top(\V{\eta}_1-\V{\eta}_6)\neq \pi(\text{mod}\,2\pi)$ or $\V{\pi}_3^\top(\V{\eta}_3-\V{\eta}_4)\neq \pi(\text{mod}\,2\pi)$. 
\end{proposition}
\noindent{\it Proof.}
%$\Msub(\V{0})$ takes the following form
%$$\begin{bmatrix}
%* & 0 & 0 & 0 & 0 & *\\
%0 & * & 0 & 0 & 0 & 0\\
%0 & 0 & * & * & 0 & 0\\
%0 & 0 & 0 & 0 & * & 0\\
%0 & 0 & * & * & 0 & 0\\
%* & 0 & * & * & 0 & *\\
%* & 0 & 0 & 0 & 0 & *
%\end{bmatrix}$$
%The second and the fifth columns of $\Msub$ have single non-zero entry, $\widetilde{m_2}(\V{\pi}_2)$ and $\widetilde{m_5}(\V{\pi}_4)$ respectively, and are orthogonal to all the rest columns, hence the full-rank constraint of $\Msub$ is reduced to the full-rank constraint on its sub-matrix (with permutation of rows and columns)
 Since $\widetilde{m_0}(\V{0})\neq 0$, as shown in Lemma \ref{lem: rank1}, $rank(\mrow{1},\mrow{6},\mrow{7},\mrow{3},\mrow{5})= 4$ at $\V{\omega} = \V{0}$. This is equivalent to the matrix $\V{B}$ defined in \eqref{eq: matrix-B} to be full rank.
\begin{align}\label{eq: matrix-B}
%\mbox{\V{B}\strut}=
\V{B} = 
\begin{bmatrix}
& & & \\[-1em]
\widetilde{m_1}(\V{\pi}_6) & \widetilde{m_6}(\V{\pi}_6) & \widetilde{m_3}(\V{\pi}_6) & \widetilde{m_4}(\V{\pi}_6) \\
\widetilde{m_1}(\V{\pi}_1) & \widetilde{m_6}(\V{\pi}_1) & 0 & 0\\
\widetilde{m_1}(\V{\pi}_7) & \widetilde{m_6}(\V{\pi}_7) & 0 & 0\\
0 & 0 & \widetilde{m_3}(\V{\pi}_3) & \widetilde{m_4}(\V{\pi}_3)\\
0 & 0 & \widetilde{m_3}(\V{\pi}_5) & \widetilde{m_4}(\V{\pi}_5)\\
\end{bmatrix}
\end{align}
Let $|\widetilde{m_1}(\V{\pi}_1)| = |\widetilde{m_1}(\V{\pi}_7)| = |\widetilde{m_6}(\V{\pi}_1)| = |\widetilde{m_6}(\V{\pi}_7)| = |\widetilde{m_3}(\V{\pi}_3)| = |\widetilde{m_3}(\V{\pi}_5)| = |\widetilde{m_4}(\V{\pi}_3)| = |\widetilde{m_4}(\V{\pi}_5)|= a$ and $|\widetilde{m_1}(\V{\pi}_6)|=| \widetilde{m_6}(\V{\pi}_6)|= | \widetilde{m_3}(\V{\pi}_6)|=| \widetilde{m_4}(\V{\pi}_6)|=b$. Rewrite $\V{B}$ as follows,
$$\V{B}=
\begin{bmatrix}
b e^{-i\V{\pi}_6^\top\V{\eta}_1} & b e^{-i\V{\pi}_6^\top\V{\eta}_6} & b e^{-i\V{\pi}_6^\top\V{\eta}_3} & b e^{-i\V{\pi}_6^\top\V{\eta}_4}\\
a e^{-i\V{\pi}_1^\top\V{\eta}_1} & a e^{-i\V{\pi}_1^\top\V{\eta}_6} & 0						& 0 \\
a e^{i\V{\pi}_1^\top\V{\eta}_1} & a e^{i\V{\pi}_1^\top\V{\eta}_6} & 0						& 0 \\
0 					& 0 					& a e^{-i\V{\pi}_3^\top\V{\eta}_3} & a e^{-i\V{\pi}_3^\top\V{\eta}_4}\\
0 					& 0 					& a e^{i\V{\pi}_3^\top\V{\eta}_3} & a e^{i\V{\pi}_3^\top\V{\eta}_4}\\
\end{bmatrix}
$$
The product of singular values of $\V{B}$ is 
\begin{align}\label{eq: detB}
\sqrt{\text{det}(\V{B}^* \V{B})} = 4a^3\sqrt{a^2 K_1^2K_2^2 + b^2(Q_1K_2^2 + Q_2K_1^2)},
\end{align}
where $ Q_1 = 1 - \cos(\V{\pi}_6^\top(\V{\eta}_1-\V{\eta}_6))\cos(\V{\pi}_1^\top(\V{\eta}_1-\V{\eta}_6)), Q_2 = 1 - \cos(\V{\pi}_6^\top(\V{\eta}_3-\V{\eta}_4))\cos(\V{\pi}_3^\top(\V{\eta}_3-\V{\eta}_4)), K_1 = \sin(\V{\pi}_1^\top(\V{\eta}_1-\V{\eta}_6)), K_2 = \sin(\V{\pi}_3^\top(\V{\eta}_3-\V{\eta}_4)).$ If $\V{\pi}_1^\top(\V{\eta}_1-\V{\eta}_6) = \V{\pi}_3^\top(\V{\eta}_3-\V{\eta}_4) = \pi (mod\, 2\pi)$, then $K_1 = K_2 = 0$ and $\V{B}$ becomes singular.\qed\\
In Lemma \ref{lem: phase-ineq}, $\Msub[:,1]$ and $\Msub[:,6]$ being independent only guarantees $\det(\Msub[-k_{\V{\omega}},:])\neq 0$. However, \eqref{eq: m0-cramer} implies that $|m_0(\V{\omega})|\propto \det(\Msub[-k_{\V{\omega}},:])$ hence it is preferred to maximize the determinant. Since
\begin{align*}
\det(\Msub[-k_{\V{\omega}}, :]) = \det\big(\big[\, \Msub[-k_{\V{\omega}},-6], \;\Msub[-k_{\V{\omega}},6] + c \cdot\Msub[-k_{\V{\omega}},1]  \,\big]\big),\quad 
\end{align*}
$\forall c\in \mathbb{C}$, the angle between $\Msub[:,1]$ and $\Msub[:,6]$ should be maximized.
Therefore, a stronger condition than ({\bf c1.1}) is to require $\Msub[:,1]$ and $\Msub[:,6]$ be orthogonal, which is equivalent to 
\begin{align*}
(\V{\eta}_1-\V{\eta}_6)^\top(\pi/2, \pi/2) = \pi \,(\text{mod}\, 2\pi).\tag{\bf c2.1}
\end{align*}
The stronger condition corresponding to ({\bf c1.2}) is 
\begin{align*}
(\V{\eta}_3-\V{\eta}_4)^\top(-\pi/2,\pi/2)=\pi(\text{mod},\,2\pi).\tag{\bf c2.2}
\end{align*} %from the stronger orthogonal condition.

%{\it Remark}
%f $|\m{1}| = |\m{2}|$ on $\{\omega_y = 3\omega_x,\,|\omega_x| > \frac{\pi}{2}\}$ and $m_0(\V{\omega}) > 0$ on $\{\omega_y = 3\omega_x\pm \pi,\,|\omega_y| <\frac{\pi}{2}\}$, then the same conditions ({\bf c1}) and ({\bf c2}) can be derived from full rank and orthogonal conditions respectively for tuples $(\,\V{\eta}_1,\,\V{\eta}_2,(-\pi/2,\pi/2)\,),\,(\,\V{\eta}_2,\V{\eta}_3,(\pi/2,\pi/2)\,),\,(\V{\eta}_4,\V{\eta}_5,\,(\pi/2,\pi/2)\,)$ and $(\,\V{\eta}_5,\V{\eta}_6,\,(-\pi/2,\pi/2)\,)$. 

% If the previous strong orthogonal condition on $\V{\eta}_1, \V{\eta}_3, \V{\eta}_4,\V{\eta}_6$ holds, then $K_1 = K_2 = 0$ and $m_0(0)=m_0^C(0)= 0$. Therefore, the strong orthogonal conditions ({\bf c2}) cannot be satisfied at the same time. 
%In particular, we consider the following constraints on phase $\V{\eta}_k\in \mathbb{Z}^2,\, k = 1,\cdots,6$:
Unfortunately, Proposition \ref{prop: origin-det} prevents ({\bf c2.1}) and ({\bf c2.2}) from holding simultaneously.
We propose the following set of phases
\begin{align}\label{eq: phase-sol}
\V{\eta}_1 = (0,0),\; \V{\eta}_2 = (-1,1),\; \V{\eta}_3 = (0,2),\notag\\
\V{\eta}_4 = (1,0),\; \V{\eta}_5 = (0,-1),\; \V{\eta}_6 = (0,1).
\end{align}
where
\begin{align*}
%\label{eq: phase-constraint}
%&(\V{\eta}_1-\V{\eta}_2)^\top(-\pi/2, \pi/2) = (\V{\eta}_5-\V{\eta}_6)^\top(-\pi/2,\pi/2) = \pi\, (\text{mod}\, 2\pi)\notag\\
%&(\V{\eta}_2-\V{\eta}_3)^\top(\pi/2,\pi/2) = (\V{\eta}_4-\V{\eta}_5)^\top(\pi/2,\pi/2) = \pi\, (\text{mod}\, 2\pi)\\
%&
(\V{\eta}_3-\V{\eta}_4)^\top(-\pi/2, \pi/2) &=-\pi/2\,(\text{mod}\,2\pi)\notag\\
 (\V{\eta}_6 - \V{\eta}_1)^\top(\pi/2,\pi/2) &= \pi/2\, (\text{mod}\,2\pi)\notag
\end{align*}
%where we require strong orthogonal constraints on pair of shifts corresponding to $\widetilde{m}$ function with non-diagonal common boundary and weaker constraints on $(\V{\eta}_1,\V{\eta}_6)$ and $(\V{\eta}_3,\V{\eta}_4)$. A solution to \eqref{eq: phase-constraint} is 

\subsection{Computing $m_0$}\label{subsec: compute-m0}
\textcolor{red}{Remove this sub-section, or place it after the main result on discontinuity}.

Let $C_{\V{\omega}} = det(\M[-k_{\V{\omega}},:])$, then we have the following observation.
\begin{lemma}\label{lem: equal-det}
$C_{\V{\omega}} = C_{\V{\omega}+\V{\pi}_2} = C_{\V{\omega}+\V{\pi}_4} = C_{\V{\omega}+\V{\pi}_6}$
\end{lemma}
\noindent{\it Proof}
Because $\widetilde{M}(\V{\omega}+\V{\pi}_2) = P_{\V{\pi}_2}\M(\V{\omega})$ where $P_{\V{\pi}_2}$ is a row permutation matrix, it follows from the definition of $C_{\V{\omega}}$ that 
$C_{\V{\omega}} = det\big(\M[-k_{\V{\omega}},:](\V{\omega})\big) = det\big(\M[-k_{\V{\omega}+\V{\pi}_2},:](\V{\omega}+\V{\pi}_2) \big)= C_{\V{\omega}+\V{\pi}_2}$ where 
$\mathbf{1}_{k_{\V{\omega}+\V{\pi}_2}} = P_{\V{\pi}_2}\mathbf{1}_{k_{\V{\omega}}}$.
\qed\\[1em]
We assume that $m_0\in\mathbb{R}_{\geq 0}$ without phase. Let $m_0^C(\V{\omega}) = m_0(\V{\omega})|C_{\V{\omega}}|\in \mathbb{R}_{\geq 0}$ and $\mc{0} = \m{0}/|C_{\V{\omega}}|$, then Lemma \ref{lem: equal-det} implies the following.
\begin{proposition}\label{prop: mc}
$m_0(\V{\omega}),\,\m{0}, m_i(\V{\omega}),\,  i = 1,...,6$ satisfy \eqref{eq: LS-new} given $\m{i},\,i=1,...,6$ if and only if $m_0^C(\V{\omega}),$ $\,\mc{0}, m_i(\V{\omega}),\,i = 1,...,6$ do. More generally, $m_0^C(\V{\omega})c(\V{\omega}),\,\mc{0}c(\V{\omega})^{-1}, m_i(\V{\omega}),\,i=1,...,6$ satisfy \eqref{eq: LS-new} if $c(\V{\omega}) = c(\V{\omega}+\V{\pi}_2)=c(\V{\omega}+\V{\pi}_4) = c(\V{\omega}+\V{\pi}_6) \neq 0$.
\end{proposition}
According to Proposition \ref{prop: mc}, we can first solve $\mc{0}$ and $m_0^C(\V{\omega})$ and then construct $c(\V{\omega})$ for optimal $\m{0}$ and $m_0(\V{\omega})$. 
In particular, $m_0^C$ can be computed without knowing $k_{\V{\omega}}$,
\begin{align}\label{eq: m0C}
m_0^C(\V{\omega}) = m_0(\V{\omega})|C_{\V{\omega}}| = |det(\M_{1,1}[-k_{\V{\omega}},:])| = \prod_{i=1}^6\sigma_i(\M_{1,1}[-k_{\V{\omega}},:]) = \prod_{i=1}^6\sigma_i(\M_{1,1}).
\end{align}
In practice, we first perform QR decomposition on $\Msub:=\M_{1,1}$ and then take the absolute value of the product of the diagonal entries of the upper triangular matrix, $diag(R)$. 
We propose the following algorithm for bi-orthogonal directional filter construction with dilated quincunx downsampling scheme:
\begin{description}% prevent items from splitting
\item[construction of bi-orthogonal basis]\
\begin{itemize}
\item[Input:] $\m{i},\,i=1,...,6$
\item[1.] compute $m_0^C(\V{\omega}) = \left|det(\M_{1,1}[-k_{\V{\omega}},:])\right|$
\item[2.] compute $\mc{0}$, such that \eqref{eq: LS-new} is solvable and \eqref{eq: identity-cond} holds
\item[3.] solve $m_i(\V{\omega}),\, i=1,...,6$ according to \eqref{eq: LS-new}
\item[4.] design $c(\V{\omega})$ and let $m_0(\V{\omega}) = m_0^C(\V{\omega})c(\V{\omega}),\,\m{0} = \mc{0}\overline{c}(\V{\omega})^{-1}$
\end{itemize}
\end{description}



\subsection{solving $m_i$}
In the final step, we substitute $\mc{0}$ and $m_0^C(\V{\omega})$ into \eqref{eq: LS-new} and rewrite it into the following linear system,
\begin{align}\label{eq: mi}
\overline{\M}[:,2:7]\,\mathbf{m}[2:7](\V{\omega}) = 
\begin{bmatrix}
1-m_0^C\overline{\widetilde{m_0}^C}(\V{\omega})\\
0\\
-m_0^C\overline{\widetilde{m_0}^C}(\V{\omega}+\V{\pi}_2)\\
\vdots \\
0
\end{bmatrix}
=:\mathbf{b}(\V{\omega}).
\end{align}
The solution of \eqref{eq: mi} depends only on $m_0^C\overline{\widetilde{m_0}^C}$, or equivalently $m_0\overline{\widetilde{m_0}}$. 


\section{Numerical Experiments}
\subsection{solving $m_0^C$}
A set of $\m{i}$ that satisfy the conditions of Theorem \ref{thm: thm} with phase terms in \eqref{eq: phase-sol} is used as the input of \eqref{eq: LS-new}.
The left figure in Fig.\ref{fig: tm_i_m_0} shows the absolute value of $\m{i}$. In particular, $\m{i} =0,\,\forall \omega\in S_1$. 
We follow the construction process in Section \ref{subsec: compute-m0} and obtain $m_0^C$ shown in the right of Fig.\ref{fig: tm_i_m_0}, in both normal scale and log scale.  
We perform a numerical sanity check on the necessary condition in Proposition \ref{prop: feasibility}, that is $\forall\,\V{\omega}, \,s.t. [m_0(\V{\omega}), m_0(\V{\omega}+\V{\pi}_2),m_0(\V{\omega}+\V{\pi}_4),m_0(\V{\omega}+\V{\pi}_6)]$ is not a linear combination of the rows of $\mathfrak{D}(\V{\omega})$ in \eqref{eq: singular-cond}. Equivalently, we compute the following quantity $$\vartheta = 1 - \Vert V^\top\mathfrak{m}_0 \Vert/\Vert \mathfrak{m}_0\Vert ,$$ where $\mathfrak{m}_0(\V{\omega})=[m_0(\V{\omega}), m_0(\V{\omega}+\V{\pi}_2),m_0(\V{\omega}+\V{\pi}_4),m_0(\V{\omega}+\V{\pi}_6)]^\top$ and $V$ are the left singular vectors of $\mathfrak{D}(\omega)$ whose corresponding singular values are non-zero. If $\mathfrak{m}_0\in span(V)$, then $\vartheta = 0$. If $\mathfrak{m}_0\bot span(V)$, then $\vartheta = 1$.
Fig.\ref{fig: feasible} shows the feasibility check $\vartheta$ of input $\m{i}$, and $\mathfrak{m}_0$ is orthogonal to $span(V)$ everywhere.

\begin{figure}
\centering
\begin{minipage}[c]{.48\textwidth}
\includegraphics[width=\textwidth]{feasible_mi.pdf}
\end{minipage}
\begin{minipage}[c]{.22\textwidth}
\centering
\includegraphics[width=.8\textwidth]{feasible_m0.pdf}
\end{minipage}
\begin{minipage}[c]{.28\textwidth}
\centering
\includegraphics[width=.8\textwidth]{feasible_m0_log.pdf}
\end{minipage}
\caption{Left:  $|\m{i}|$, middle: computed $m_0^C$, right: $\log(m_0^C)$}
\label{fig: tm_i_m_0}
\end{figure}


\subsection{solving $\mc{0}$ and $m_i$}
We compute $\mc{0}$ by solving the following optimization problem similar to \eqref{eq: opt-diff} for the dyadic scheme,
\begin{align}
\min_{\xvec}\; \Vert \V{D}(\mathbf{m}_0^C\circ\xvec)\Vert^2 + \lambda\Vert \wvec\circ\mathbf{m}_0^C\circ\xvec\Vert^2,\quad 
s.t. \; A\xvec = \mathbf{1},\, \mathfrak{D}\xvec = \mathbf{0}
\label{eq: opt-2d}
\end{align}
where $\circ$ is Hadamard product and $\wvec$ is a weight vector and we consider real solution $\xvec$ here.
$A$ in the constraint is the matrix generated from the identity condition \eqref{eq: identity-cond} and $\mathfrak{D}$ is generated from the singularity condition \eqref{eq: singular-cond}. Since $A$ and $\mathfrak{D}$ are linearly independent, \eqref{eq: opt-2d} is feasible. Here, instead of optimizing the properties of $\xvec$ as in \eqref{eq: opt-diff}, we optimize those of $\widetilde{\mathbf{m}_0}^C\circ \xvec$ since $m_0^C \cdot\widetilde{m_0}^C$ will be later re-decomposed into $m_0$ and $\widetilde{m_0}$. In addition, if $m_0^C$ is symmetric with respect to the two coordinates $\omega_x$ and $\omega_y$, then we impose the same symmetry on $\widetilde{m_0}^C$ by solving \eqref{eq: opt-2d} on $[0,\pi)\times[0,\pi)$ and then extend the solution to $[-\pi,\pi)\times[-\pi,\pi)$ by symmetry.

\begin{figure}
\centering
\begin{minipage}[c]{.3\textwidth}
\includegraphics[width = .8\textwidth]{feasible_check.pdf}
\caption{$\vartheta$}\label{fig: feasible}
\end{minipage}
\begin{minipage}[c]{.63\textwidth}%{.28\textwidth}
\centering
\includegraphics[width = .38\textwidth]{feasible_tm0.pdf}\hspace*{2em}
\includegraphics[width = .42\textwidth]{feasible_m0tm0.pdf}
\caption{Left: : computed $\widetilde{m_0}^C$, right: $\widetilde{m_0}^C \cdot m_0^C $}
\label{fig: tm0}
\end{minipage}
\end{figure}

Fig.\ref{fig: tm0} shows $\mc{0}$ obtained from \eqref{eq: opt-2d} and $\widetilde{m_0}^C \cdot m_0^C$ which is $\mathbf{1}_{S_1}$.

In particular, given $\widetilde{m_0}^C \cdot m_0^C = 1$, $\mathbf{b}(\V{\omega}) = \mathbf{0}, \, \forall\,\V{\omega}\in S_1$, hence $\mathbf{m}[2:7] = \mathbf{0}$. 
When $\mathbf{b}(\V{\omega})\neq \mathbf{0}$, \eqref{eq: mi} is a degenerated over-determinant linear system (we also do a sanity check here for the linearity between $\overline{\M}[:,2:7]$ and $\mathbf{b}$ by computing $\vartheta$) and 
$$\mathbf{m}[2:7](\V{\omega}) = \Big(\overline{\M}[:,2:7]\Big)^\dagger\,\mathbf{b}(\V{\omega}),$$
where $\dagger$ is the pseudo-inverse of a matrix. Fig.\ref{fig: m_i} shows the solution $m_i$ of \eqref{eq: mi} and the corresponding spatial filters $\mathcal{F}^{-1}\widetilde{m_0}$.
As shown in Fig.\ref{fig: m_i}, the energy of $m_i$ concentrates on $\{|\omega_x| = \frac{\pi}{2},\, |\omega_y| = \frac{\pi}{2}\}$ where $|\widetilde{m_i}|$ is small, and the filters decay slowly in time domain.

The bi-orthogonal bases constructed is not ideal, despite the regularization on $m_0$ in the optimization \eqref{eq: opt-2d}. Since no explicit regularization is put on $m_i$, it's difficult to control the regularity of the output $m_i$ from the input $\widetilde{m_i}$.

\begin{figure}
\centering
\begin{minipage}[c]{.5\textwidth}
\includegraphics[width = .9\textwidth]{feasible_m.pdf}
\end{minipage}
\begin{minipage}[c]{.48\textwidth}
\includegraphics[width = .9\textwidth]{feasible_mi_time.pdf}
\end{minipage}
\caption{Left: $|m_i|,\, i = 1,\cdots,6$, right:$|\mathcal{F}^{-1}m_i|$}
\label{fig: m_i}
\end{figure}
%\section{Bi-orthogonal Bases}\label{sec: bi-orth}
In this section, we analyze bi-orthogonal bases in the following form of MRA,
\begin{align}\label{eq: bi-orth MRA}
\{\phi_{L,\V{k}},\widetilde{\phi}_{L,\V{k}}, \psi_{l,\V{k}'}^j,\widetilde{\psi}_{l,\V{k}'}^j,\, 1\leq l\leq L,\,\V{k}\in\mathbb{Z}^2,\, \V{k}'\in\mathbf{Q}\mathbb{Z}^2,\,1\leq j\leq J \},
\end{align}
where $\phi$ and $\psi^j$ satisfy \eqref{eq: m0} and \eqref{eq: mj}, as well as $\widetilde{\phi}$ and $\widetilde{\psi^j}$, respectively,
$$\widehat{\widetilde{\phi}}(\V{D}^T\V{\omega}) = \widetilde{m_0}(\V{\omega})\widehat{\widetilde{\phi}}(\V{\omega}),\quad \widehat{\widetilde{\psi^j}}(\V{D}^T\V{\omega}) = \widetilde{m_j}(\V{\omega})\widehat{\widetilde{\phi}}(\V{\omega}).$$
For such bi-orthogonal bases, we have the similar identity summation and shift cancellation condition to those in Theorem \ref{thm: conds}.
\begin{thm}\label{thm: bi-orth conds}
The perfect reconstruction iff the following two conditions hold
\begin{align}\label{eq: id-sum 2}
m_0(\boldsymbol{\omega})\sbarm{0} + \sum_{j = 1}^6 m_j(\boldsymbol{\omega})\sbarm{j} = 1
\end{align}
\begin{equation}\label{eq: shift-cancel 2}
\begin{cases}
\sum_{j = 0}^6m_j(\boldsymbol{\omega})\overline{\widetilde{m_j}}(\boldsymbol{\omega} + \boldsymbol{\pi}) = 0, & \boldsymbol{\pi}\in \Gamma_0\setminus\{\boldsymbol{0}\}\\[.5em]
\sum_{j=1}^6m_j(\boldsymbol{\omega})\overline{\widetilde{m_j}}(\boldsymbol{\omega}+\boldsymbol{\pi}) = 0, & \boldsymbol{\pi}\in\Gamma_1\setminus\Gamma_0
\end{cases}
\end{equation}
\end{thm}
The conditions \eqref{eq: id-sum 2} and \eqref{eq: shift-cancel 2} can be combined into a linear system as follows,
\begin{align}\label{eq: LS-new}
%\overline{\M}(\V{\omega})\mathbf{m}_0(\V{\omega})=
\begin{bmatrix}
    \,\sbarm{0} & \sbarm{1} & \hdots & \sbarm{6}\;  \\
    \;0 & \sbarmp{1}{1}  & \hdots  & \sbarmp{6}{1}\; \\
    \,\sbarmp{0}{2} & \sbarmp{1}{2} & \hdots & \sbarmp{6}{2}\;\\
    \;\vdots & \vdots & \vdots & \vdots \; \\
    \;0 & \sbarmp{1}{7} & \hdots & \sbarmp{6}{7}\;
\end{bmatrix}
\begin{bmatrix}
\;\mo{0}\; \\
\;\mo{1}\; \\
\;\mo{2}\; \\
\; \vdots\; \\
\;\mo{6}\; 
\end{bmatrix} 
=
\begin{bmatrix}
1\\
0\\
0\\
\vdots \\
0
\end{bmatrix}
\end{align}
%where $\M\in\mathbb{C}^{8\times 7}$ and $\mathbf{m}_0\in\mathbb{C}^7$.
In addition, we have the following analogue of Theorem \ref{thm: basis cond}.
\begin{thm}\label{thm: basis cond 2}
Assume that $m_0, \widetilde{m_0}$ are trigonometric polynomials with $m_0(0)=\widetilde{m_0}(0) = 1$, which generate $\phi,\widetilde{\phi}$ respectively.\\
If $\phi(\cdot - \boldsymbol{k}),\widetilde{\phi}(\cdot - \boldsymbol{k}),\,\boldsymbol{k}\in\mathbb{Z}^2$ are bi-orthogonal, then $\exists K$ containing a neighborhood of 0, s.t. $\forall\boldsymbol{\omega}\in S_0,\,\boldsymbol{\omega}+2\pi\mathbf{n}\in K$ for some $\mathbf{n}\in\mathbb{Z}^2, $ and $\inf_{k>0,\,\boldsymbol{\omega}\in K}|m_0(\mathbf{D_2}^{-k}\boldsymbol{\omega})| >0$, $\inf_{k>0,\,\boldsymbol{\omega}\in K}|\widetilde{m_0}(\mathbf{D_2}^{-k}\boldsymbol{\omega})| >0$. 
 Further, if  $\sum_{\boldsymbol{\V{\pi}}\in \Gamma_0} m_0(\boldsymbol{\omega}+\boldsymbol{\pi})\sbarmp{0}{} = 1,$ then the inverse is true.
\end{thm}
By Theorem \ref{thm: basis cond 2}, $m_0$ and $\widetilde{m_0}$ need to satisfy the following identity constraint for the MRA \eqref{eq: bi-orth MRA} to be bi-orthogonal,
\begin{align}\label{eq: identity-cond}
m_0\sbarm{0} + m_0\sbarmp{0}{2} + m_0\sbarmp{0}{4} + m_0\sbarmp{0}{6} = 1.
\end{align}
In sum, the construction of a bi-orthogonal basis \eqref{eq: bi-orth MRA} is equivalent to find feasible solutions of \eqref{eq: LS-new} with constraint \eqref{eq: identity-cond}. To solve \eqref{eq: LS-new}, we use the same approach in \cite{cohen1993compactly}, which solves compactly supported symmetric bi-orthogonal filters on hexagon lattice. We next review the main scheme in \cite{cohen1993compactly} and extend it to our setup of directional wavelet filter.

\subsection{Summary of Cohen et al's construction}\label{subsec: cohen-summary}
We summerize the main setup and the approach in \cite{cohen1993compactly}. Consider a bi-orthogonal scheme consists of 3 high-pass filters $m_1,m_2$ and $m_3$ and a low-pass filter $m_0$ together with their bi-orthogonal duals $\widetilde{m_j}$, s.t.
$m_0$ is $\frac{2\pi}{3}$-rotation invariant and $m_1,\, m_2,\, m_3$ are $\frac{2\pi}{3}$-rotation co-variant.

This bi-orthogonal scheme satisfies the following linear system (
Lemma 2.2.2 in \cite{cohen1993compactly} )
\begin{align}\label{eq: LS}
\begin{bmatrix}
    \,\barm{0} & \barm{1} & \barm{2} & \barm{3}\;  \\
    \;\barmn{0}{1} & \barmn{1}{1}  & \barmn{2}{1}  & \barmn{3}{1}\; \\
    \;\vdots & \vdots & \vdots & \vdots \; \\
    \;\barmn{0}{3} & \barmn{1}{3} & \barmn{2}{3} & \barmn{3}{3}\;
\end{bmatrix}
\begin{bmatrix}
\;\mo{0}\; \\
\;\mo{1}\; \\
\;\mo{2}\; \\
\;\mo{3}\; 
\end{bmatrix} 
=
\begin{bmatrix}
1\\
0\\
0\\
0
\end{bmatrix}
\end{align}
 where $\V{\nu}_1 = (\pi,0),\V{\nu}_2 = (0,\pi),\V{\nu}_3=(\pi,\pi)$.
 Let $\widetilde{\mathbf{M}}(\V{\omega})\in\mathbb{C}^{4\times 4}$ be the matrix with $\barm{j}$ entries and $\mathbf{m}(\V{\omega})\in\mathbb{C}^4$ be the vector with $m_j$ entries in \eqref{eq: LS}, then \eqref{eq: LS} can be written as \(\widetilde{\mathbf{M}}\, \mathbf{m} (\V{\omega})= [1,0,0,0]^\top\).\\
Given $\m{1}$, $\m{2},\,\m{3}$ are determined by symmetry, and Lemma 2.2.2 in \cite{cohen1993compactly} shows that
\begin{align}\label{eq: m0-sol}
m_0(\V{\omega}) &= D^{-1}%\propto 
\left|
\begin{matrix}
    \; \sbarmn{1}{1}  & \sbarmn{2}{1}  & \sbarmn{3}{1}\; \\
    \; \sbarmn{1}{2}  & \sbarmn{2}{2}  & \sbarmn{3}{2}\; \\
    \; \sbarmn{1}{3} & \sbarmn{2}{3} & \sbarmn{3}{3}\;
\end{matrix}
\right| \notag\\
&= D^{-1}\det(\widetilde{\mathbf{M}}_{1,1}(\V{\omega})),
\end{align}
where $ D \equiv \det(\widetilde{\mathbf{M}})\in \mathbb{C}^* = \mathbb{C}\setminus\{0\}$.
%{\it Remark.} 
%For \eqref{eq: m0-sol} to hold, $m_0(\mathbf{\omega})$ and $\det(\widetilde{\mathbf{M}}_{1,1}(\V{\omega}))$ having the same phase suffices, which is implied by the symmetry of $m_0$ and $\widetilde{m_j} $'s.\\ % Both $\mo{0}$ and $\det(\widetilde{\mathbf{M}}_{1,1}(\V{\omega}))$ are $\frac{2\pi}{3}-$rotation invariant. \\
If $\widetilde{m_0}$ is solved, then $m_1,m_2$ and $m_3$ are obtained by solving the linear system \eqref{eq: LS}.
To get $\m{0}$, we solve 
\begin{align}\label{eq: bi-orth-eq}
m_0\sbarm{0} + m_0\sbarmn{0}{1} + m_0\sbarmn{0}{2} + m_0\sbarmn{0}{3} = 1
\end{align}
from expanding $det(\widetilde{\mathbf{M}})$ with respect to the first column.
According to Lemma 3.2.1 in \cite{cohen1993compactly} based on {\it Hilbert's Nullstellensatz}, \eqref{eq: bi-orth-eq} has a solution iff there does not exist $(z_1,z_2)\in (\mathbb{C}^*)^2,\, \mathbb{C}^* = \mathbb{C}\setminus\{0\}$\, s.t. $(\pm z_1,\pm z_2)$ are all 
vanishing points of the $z$-transform of $m_0$.

\subsubsection{Solving $\m{0}$}
In general, there is no efficient algorithm to solve {\it Hilbert's Nullstellensatz}, and how \eqref{eq: m0-sol} is solved exactly is not mentioned in \cite{cohen1993compactly}.

We propose an optimization approach, where \eqref{eq: m0-sol} is equivalent to a linear constraint and the objective function imposes regularity on $\widetilde{m_0}$.
On a $2N\times 2N$ grid $\G$ of $S_0 = [-\pi, \pi)\times[-\pi, \pi)$, s.t. $\forall \V{\omega}_j \in \G, \; \V{\omega}_j+\V{\nu}_1,\,\V{\omega}_j+\V{\nu}_2,\,\V{\omega}_j+\V{\nu}_3 \in \G$, \eqref{eq: m0-sol} is reformulated as
\begin{align}
\hspace*{10em} \V{A}\, \mathbf{\widetilde{m}_0}&= \mathbf{1}_{4N^2}, \label{eq: m0-A}\\ 
\mathbf{\widetilde{m}_0} = [\widetilde{m_0}(\V{\omega}_i)]_{\,i=1,\hdots,4N^2} \quad &\V{A}_{i,j} = m_0(\V{\omega}_j)\sum_{k=0}^3\delta(\V{\omega}_j-\V{\omega}_i-\V{\nu}_k) \notag
\end{align}
Because the set $\{\V{\omega},\, \V{\omega}+\V{\nu}_k,k=1,2,3\}$ is invariant under the shift $\V{\nu}_i,\, i = 1,2,3,$ the rows of $\V{A}$ corresponding to $\V{\omega}$ and $\V{\omega}+\V{\nu}_i$ are identical and we only need to consider rows corresponds to $\V{\omega}\in [-\pi,\pi)\times[-\pi,\pi)/\{\V{0},\,\V{\nu}_i,i=1,2,3\}$. Therefore, after removing the duplicate rows, $\V{A}\in \mathbb{C}^{N^2\times 4N^2}$ and \eqref{eq: m0-A} is under-determinant. \\
We thus use \eqref{eq: m0-A} as a linear constraint in quadratic optimization to solve $\mathbf{\widetilde{m}_0}$. Suppose that $\m{0}$ is smooth, then we build a differential operator $\V{D}$ and solve the following minimization problem:
\begin{align}
&\min_{\mvec{0}}\; \Vert \V{D}\mvec{0}\Vert^2,\quad s.t. \; \V{A}\mvec{0} = \mathbf{1} \label{eq: opt-diff}
\end{align}
%or
%\begin{align}
%&\min_{\mvec{0}}\; \Vert \V{D}\mvec{0}\Vert^2 + \lambda \Vert \V{A}\mvec{0} - \mathbf{1}\Vert^2 \label{eq: m0-smooth-relaxed}
%\end{align}
%The solution of \eqref{eq: m0-smooth-relaxed} is $\mvec{0} = \lambda(\lambda \V{A}^\top \V{A} + \V{D}^\top \V{D})^{-1}\V{A}^\top\mathbf{1}$.

Or suppose $\m{0}$ decays away from the origin, then we build a diagonal weighting operator $\V{W}$, and solve the following minimization problem:
\begin{align}\label{eq: opt-weight}
&\min_{\mvec{0}}\; \Vert \V{W}\mvec{0}\Vert^2,\quad s.t. \; \V{A}\mvec{0} = \mathbf{1}
\end{align}
Supplementary numerical results on solving $\m{0}$ by optimization are provided in Appendix \ref{app: supp-numerical}, where we test this optimization method on pre-designed bi-orthogonal filters $m_0$ and $\widetilde{m_0}$.

\section{Adaptation to dilated quincunx scheme}

\textcolor{red}{Replace by an overview of this section}

Following the same approach of Cohen et al, we focus on solving $m_i$'s and $\widetilde{m_0}$ in \eqref{eq: LS-new} given pre-designed $\m{i},\,i=1,\cdots,6$. %Assume $\m{i},\,i=1,\cdots,6$ satisfy weak constraints on the direction selectivity of their support.

For simplicity, in the rest of this paper, we reuse notations for \eqref{eq: LS} to express \eqref{eq: LS-new} in the same form as $\widetilde{\mathbf{M}}\mathbf{m}(\V{\omega}) = [1,0,\cdots,0]^T$, where $\widetilde{\mathbf{M}}(\V{\omega})\in\mathbb{C}^{8\times 7}$ and $\mathbf{m}(\V{\omega})\in\mathbb{C}^7$. In addition, for a matrix $\mathbf{A}$, we denote its sub-matrix containing all but the $k-$th row(column) as $\mathbf{A}[-k,:]\, (\mathbf{A}[:,-k])$.% and its minor with respect to its $(i,j)-$th entry as $A_{-i,-j} = \det(\mathbf{A}[-i,-j])$. 
In particular, we denote $\M[-1,-1]$ as $\Msub$.

As in Section \ref{subsec: cohen-summary}, we implicitly assume that $\M(\V{\omega})$ is full rank for any $\V{\omega}$ so that \eqref{eq: LS-new} has unique solution $\mathbf{m}$ and Cramer's rule can be applied to compute $m_0$ with respect to a non-singular sub-matrix of $\M$. That is, $\exists\, k_\omega\in\{2,\cdots,8\}$ such that $\M[-k_{\V{\omega}},:]$ is non-singular\footnote{Lemma \ref{lem: subM-singular} shows that $k_\omega\neq 1$}. Therefore, 
%there is a unique row $\M[k_{\V{\omega}},:],\,k_\omega\in\{2,\cdots,8\}$ such that removing it from $\M$ gives a non-singular square matrix $\M[-k_{\V{\omega}},:]$. By Cramer's rule, 
\begin{align}\label{eq: m0-cramer}
m_0(\V{\omega}) = \det(\Msub[-k_{\V{\omega}},:])/\det(\M[-k_{\V{\omega}},:]).
\end{align}
Instead of requiring strong symmetries of $\m{i}$'s as in Section \ref{subsec: cohen-summary}, we only ask for a minimum symmetry of $\m{i}$ such that $|\m{1}|$ and $|\m{6}|$ are symmetric with respect to the diagonal $\omega_1=\omega_2$, i.e.
$$ |\widetilde{m_1}(\V{\omega})| = |\widetilde{m_6}(\V{\omega}')|\quad \forall\, \omega_1=\omega_2', \,\omega_2=\omega_1',$$
and likewise for $\m{3}$ and $\m{4}$,
$$ |\widetilde{m_3}(\V{\omega})| = |\widetilde{m_4}(\V{\omega}')|\quad \forall\, \omega_1=-\omega_2', \,\omega_2=-\omega_1'.$$

In the following subsections, we first show our main result that for \eqref{eq: LS-new} to be solvable, the pre-designed $\widetilde{m_i}$'s are discontinuous. We then discuss how to design $\widetilde{m_i}$'s and solve the corresponding system \eqref{eq: LS-new}.

\subsection{Singularity of $\M[-1,:]$ and discontinuity of $\m{i}$}

%Same as in Section \ref{subsec: cohen-summary} , we first compute $m_0$ and assume that $\M$ is full rank, otherwise \eqref{eq: LS-new} has infinitely many solutions. Moreover, $\M[2:8,:]$ is singular. 
\begin{lemma}\label{lem: subM-singular}
If \eqref{eq: LS-new} is solvable, then $\M[-1,:](\V{\omega})$ is singular $\forall \V{\omega}$.
\end{lemma}
\noindent{\it Proof.}
If \eqref{eq: LS-new} has a solution, then $\forall \V{\omega}$,  $[1,0,\cdots,0]^\top\in \mathbb{R}^8$ is a linear combination of the columns of $\M$ hence the solution $\mathbf{m} \in Null(\M[-1,:])$ and it is non-zero. This implies that $\M[-1,:]$ is singular.\qed\\[1em]
Let $\mrow{i}(\V{\omega}) = [\widetilde{m_1}(\V{\omega}+\V{\pi}_i)\, \cdots,\,\widetilde{m_6}(\V{\omega}+\V{\pi}_i)]\in\mathbb{C}^6,\, i = 0,\cdots,7$ be the rows of $\M[:,-1]$, and define $$d_{i,j}(\V{\omega}) = \det([\mrow{k_1}(\V{\omega})^\top,\cdots,\mrow{k_6}(\V{\omega})^\top]),\;$$ where $0\leq k_1<\cdots<k_6\leq 7,\, s.t.\,z k_l\neq i,j.$
\begin{lemma}\label{lem: subM-singular-sys}
$\M[-1,:](\V{\omega})$ is singular $\forall \V{\omega}$ if and only if \vspace{.5em}
\begin{align}
\label{eq: singular-cond}
\mathfrak{D}(\omega)\begin{bmatrix}
\m{0}\\
\mp{0}{2}\\
\mp{0}{4}\\
\mp{0}{6}
\end{bmatrix}
\doteq
\begin{bmatrix}
0 & d_{0,2} & d_{0,4} & d_{0,6}\\
-d_{0,2} & 0 & d_{2,4} & d_{2,6}\\
-d_{0,4} & -d_{2,4} & 0 & d_{4,6}\\
-d_{0,6} & -d_{2,6} & -d_{4,6} & 0
\end{bmatrix}
\begin{bmatrix}
\m{0}\\
\mp{0}{2}\\
\mp{0}{4}\\
\mp{0}{6}
\end{bmatrix}
= \begin{bmatrix}
0\\0\\0\\0
\end{bmatrix}.
\end{align}
\end{lemma}
\noindent{\it Proof.}
The singularity condition on  $\M[-1,:](\V{\omega})$ can be rewritten as follows,
\begin{align}\label{eq: singular-omega}
0 &=\det(\M[-1,:]) \notag\\
&=  \widetilde{m_0}(\V{\omega}+\V{\pi}_2)\cdot\det(\Msub[-2,:])\notag\\
&\quad+ \,\widetilde{m_0}(\V{\omega}\,+\,\V{\pi}_4)\cdot\det(\Msub[-4,:])
+ \widetilde{m_0}(\V{\omega}+\V{\pi}_6)\cdot\det(\Msub[-6,:])\notag\\
&= 0\cdot\widetilde{m_0}(\V{\omega})\,+\,d_{0,2}\cdot\widetilde{m_0}(\V{\omega}+\V{\pi}_2) \notag\\
&\quad+\,d_{0,4}\cdot \widetilde{m_0}(\V{\omega} + \V{\pi}_4)\,+\, d_{0,6}\cdot\widetilde{m_0}(\V{\omega} + \V{\pi}_6) 
\end{align}
This is the first equation in the linear system \eqref{eq: singular-cond}. Substitute $\V{\omega}$ by $\V{\omega + \pi_2}$ in \eqref{eq: singular-omega} and use the $2\pi-$periodicity of $\V{\omega}$, we have the singularity condition on $\M[-1,:](\V{\omega+\pi_2})$ as follows
%then the above singularity condition on $\M[-1,:]$ at $\V{\omega}$ can be rewritten as follows,
%\begin{align*}
%[0,\, d_{0,2}(\V{\omega}),\, d_{0,4}(\V{\omega}),\, d_{0,6}(\V{\omega})]\,[\widetilde{m_0}(\V{\omega}),\,\widetilde{m_0}(\V{\omega}+\V{\pi}_2),\, \widetilde{m_0}(\V{\omega}+\V{\pi}_4),\,\widetilde{m_0}(\V{\omega}+\V{\pi}_6)]^\top = 0
%\end{align*}
%It is easy to verify that the above singular condition at $\V{\omega}+\V{\pi}_2$ is equivalent to 
\begin{align*}
-d_{0,2}(\V{\omega})\cdot \widetilde{m_0}(\V{\omega}) + d_{2,4}(\V{\omega})\cdot \widetilde{m_0}(\V{\omega}+\V{\pi}_4) + d_{2,6}(\V{\omega})\cdot\widetilde{m_0}(\V{\omega}+\V{\pi}_6) = 0,
%[-d_{0,2}(\V{\omega}),\, 0,\,d_{2,4}(\V{\omega}),\,d_{2,6}][\widetilde{m_0}(\V{\omega}),\,\widetilde{m_0}(\V{\omega}+\V{\pi}_2),\, \widetilde{m_0}(\V{\omega}+\V{\pi}_4),\,\widetilde{m_0}(\V{\omega}+\V{\pi}_6)]^\top = 0,
\end{align*}
which is the second linear equation in  \eqref{eq: singular-cond}.
Similarly, the third and fourth equations can be obtained by rewriting the singularity condition at $\V{\omega}+\V{\pi}_4$ and $\V{\omega}+\V{\pi}_6$ in the coordinate of $\V{\omega}$.\qed
%where $\mathfrak{D}(\V{\omega})$ is anti-symmetric. Because $\mathfrak{D}(\V{\omega})$ is independent of $m_0(\V{\omega})$, \eqref{eq: singular-cond} holds for $\mc{0}$ as well.

%On the other hand, given $m_0$, $\widetilde{m_0}$ has to satisfy the identity constraint \eqref{eq: identity-cond}.
The identity constraint \eqref{eq: identity-cond} on $m_0$ and the singularity condition \eqref{eq: singular-cond} together imply the following proposition,
%Due to the periodic wrapping of the frequency square $S_0$, we only need to consider \eqref{eq: singular-cond} and \eqref{eq: identity-cond} on $S_1$ and they imply the following proposition,
\begin{proposition}\label{prop: feasibility}
Given $\widetilde{m_i}, i = 1,\cdots,6$, \eqref{eq: LS-new} has no solution for $\widetilde{m_0}$, if $\exists\,\omega, \,s.t. \; [m_0(\omega), m_0(\omega+\pi_2),m_0(\omega+\pi_4),m_0(\omega+\pi_6)]$ is a linear combination of the conjugate of the rows of $\mathfrak{D}(\omega)$.% in \eqref{eq: singular-cond}.
\end{proposition}
%Proposition \ref{prop: feasibility} provides a necessary condition such that the numerical optimization solving $\widetilde{m_0}$ is feasible.
\begin{figure}
\centering
\includegraphics[width = .4\textwidth]{triangle-partition-new.png}
\caption{Partition of frequency square in six directions, where the essential support of $\m{i}$ is contained in each pair of triangles $T_i$. The pair of dark grey triangles is $T_1^-$ and the light grey pair is $T_1^+$.}
\label{fig: partition 2}
\end{figure}
%Let pairs of triangles $T_i$ in Fig.\ref{fig: partition 2} contain the essential support of $\widetilde{m_i},\,i=1,\cdots,6$.
%\eqref{eq: LS-new} takes a similar form to \eqref{eq: LS}, but with $\M\in\mathbb{C}^{8\times 7}$, which is an over-determinant linear system.

\noindent{\bf Definition.}
The {\it essential support} $\Omega_i$ of a function $\widetilde{m_i}$ is the set $\{\V{\omega}:\,|\widetilde{m_i}(\V{\omega})|> |\widetilde{m_j}(\V{\omega})|,\,\forall j\neq i\}$. \vspace{.5em}

Let $T_i$ be pairs of triangles shown in Figure \ref{fig: partition 2}, such that $C_i\subset T_i,\, i = 1,\cdots,6.$ Consider its decomposition, $T_i = T_i^-\bigcup T_i^+$, where $T_i^-, T_i^+$ are halves of $T_i$ adjacent  to $T_{i-1}$ and $T_{i+1}$ respectively.\\[.5em]
\noindent{\bf Definition.}  $\widetilde{m_i}$ {\it concentrates} within cone $T_i$ if 
\begin{itemize}
\item[(i)] $\Omega_i\subset T_i$;
\item[(ii)]$\text{supp}(\widetilde{m_i})\subset T_{i-1}^+\bigcup T_i\bigcup T_{i+1}^-$ and $\int_\Omega|\widetilde{m_i}| > \int_{\Omega'}|\widetilde{m_i}|, \forall\, \Omega\subset T_i\bigcap\text{supp}(\widetilde{m_i})$, where $\Omega' \subset T_{i-1}^+\bigcup T_{i+1}^-$ is symmetric to $\Omega$ with respect to the boundary of $T_i$.
\end{itemize}

\begin{figure}
\centering
\includegraphics[width = .5\textwidth]{S_shifts2.png}
\caption{$S_{\rho}$ and its shifts}
\label{fig: S-shifts}
\end{figure}
Given $\m{i}$ that concentrates in $T_i$, we study the feasibility condition in Proposition \ref{prop: feasibility} specifically on the domain $S_{\rho} = \{(\omega_1,\omega_2)|\;\Vert\omega\Vert < \rho, \omega_1 <0,\,\omega_2<0\}$, see Figure \ref{fig: S-shifts}. 

\begin{lemma}\label{lem: rank1}
$\exists\, \rho>0$ s.t. $\forall \omega\in S_\rho$, $rank(\mrow{1},\mrow{7})=1$ or $rank(\mrow{3},\mrow{5}) = 1$.
\end{lemma}
\noindent{\it Proof.}
When $\rho$ is small enough, due to the concentration property, $\m{i}$ is zero on all but a few sets $S_\rho + \V{\pi}_j$ (see Fig.\ref{fig: S-shifts} for reference of $S_\rho$ and its shifts), thus $\mrow{i}(\V{\omega})$ is sparse on $S_\rho$ and $\M[-1,:]$ takes the following form
\begin{align}
\label{eq: sparse-mat}
\M[-1,:](\V{\omega})=
\begin{bmatrix}
\mrow{0}\\
\mrow{1}\\
\mrow{2}\\
\mrow{3}\\
\mrow{4}\\
\mrow{5}\\
\mrow{6}\\
\mrow{7}
\end{bmatrix}
=
\begin{bmatrix}
0 & 0 & 0 & 0 & 0 & 0\\
* & 0 & 0 & 0 & 0 & *\\
0 & 0 & 0 & * & * & 0\\
0 & 0 & * & * & 0 & 0\\
0 & * & * & 0 & 0 & 0\\
0 & 0 & * & * & 0 & 0\\
* & 0 & * & * & 0 & *\\
%0 & * & * & 0 & 0 & 0\\
%0 & 0 & 0 & * & * & 0\\
%* & 0 & 0 & 0 & 0 & *\\
%* & 0 & 0 & 0 & 0 & *\\
%0 & 0 & * & * & 0 & 0\\
%0 & 0 & * & * & 0 & 0\\
* & 0 & 0 & 0 & 0 & *
\end{bmatrix}
%=\V{P}\,\widetilde{\mathbf{M}}[:,2:7],
\end{align}
where $*$s denote possible non-zero entries.
%where $\V{P}$ is a row permutation matrix. 
We make the following observation of $\mrow{i}$:
\begin{itemize}
\item[(i)] $\mrow{0}$ is a zero vector
\item[(ii)] $\mrow{2}$ and $\mrow{4}$ are linearly independent of each other and the rest of $\mrow{i}$
\item[(iii)] $span\{\mrow{1},\mrow{7}\} \perp span\{\mrow{3},\mrow{5}\}$ and $rank(\mrow{1},\mrow{7}) \leq 2$, \\$rank(\mrow{3},\mrow{5})\leq 2$
\item[(iv)] $span\{\mrow{1}, \mrow{7}, \mrow{3},\mrow{5},\mrow{6}\} \leq 4$
\end{itemize}
Since $S_\rho$ is in the low frequency domain, $m_0(\V{\omega})\neq 0$. \eqref{eq: m0-cramer} then implies that $\Msub$ is full rank, or equivantly, $rank(\M[-1,:]) = 6$. It follows from  (ii) and (iv) that $rank(\mrow{1},\mrow{6},\mrow{7},\mrow{3},\mrow{5})= 4$.\\
On the other hand, (ii) and (iv) imply that $$rank(\Msub(\V{\omega}+\V{\pi}_2))=rank(\mrow{0},\mrow{4},\mrow{6},\mrow{1},\mrow{3},\mrow{5},\mrow{7})= 5$$ and likewise $$rank(\Msub(\V{\omega}+\V{\pi}_4))=rank(\mrow{0},\mrow{2},\mrow{6},\mrow{1},\mrow{3},\mrow{5},\mrow{7})= 5.$$ Therefore, $\det(\Msub(\V{\omega} + \V{\pi}_2)) = \det(\Msub(\V{\omega} + \V{\pi}_4)) = 0$ and \eqref{eq: m0-cramer} implies $m_0(\V{\omega}+\V{\pi}_2) = m_0(\V{\omega}+\V{\pi}_4) = 0$.\\
If $\mrow{1}$ and $\mrow{7}$ are linearly independent and so are $\mrow{3}$ and $\mrow{5}$, then $$rank(\Msub(\V{\omega}+\V{\pi}_6))=rank(\mrow{2},\mrow{4},\mrow{1},\mrow{3},\mrow{5},\mrow{7}) = 6,$$ hence $m_0(\V{\omega}+\V{\pi}_6)\neq 0$. Therefore, $$[m_0(\V{\omega}),m_0(\V{\omega}+\V{\pi}_2),m_0(\V{\omega}+\V{\pi}_4),m_0(\V{\omega}+\V{\pi}_6)] = [*,0,0,*].$$ In addition, $d_{i,j} = 0,\, \forall(i,j)$ except $(0,6)$, so in \eqref{eq: singular-cond} $$\mathfrak{D}(\V{\omega}) = [d_{0,6}, 0, 0,0]^\top [0,0,0,1] + [0,0,0,d_{0,6}]^\top [-1,0,0,0].$$  By Proposition \ref{prop: feasibility}, the linear system \eqref{eq: LS-new} has no solution $\widetilde{m_0}$ and this proofs the lemma.\qed\\[.5em]
Without loss of generality, in the following analysis, we assume $rank(\mrow{1},\mrow{7}) = 1$ on $S_\rho$.
\begin{lemma}\label{lem: concentrate}
If $\m{1} (\m{6})$ concentrates in $T_1 (T_6)$, then $|\m{6}| > |\m{1}|\,$ a.e. on $T_6\bigcap \text{supp}(\widetilde{m_6})$ ($|\m{1}| > |\m{6}|$ 
a.e. on $T_1\bigcap\text{supp}(\widetilde{m_1})$).
\end{lemma}
\noindent{\it Proof}
Let $B_6=\{\V{\omega}: |\m{6}| \leq |\m{1}|\}\bigcap T_6\bigcap supp(\widetilde{m_1})$ and $B_1$ be its mirror set with respect to $\omega_1 = \omega_2$ and suppose $|B_6|>0$, then $\int_{B_6}|\m{6}|\leq \int_{B_6}|\m{1}|$. On the other hand, since $\m{1}$ concentrates in $T_1$, we know $\int_{B_1}|\m{1}| > \int_{B_6}|\m{1}|$. Moreover, due to the symmetry of $\m{1},\m{6}$ and $B_1,B_6$, $\int_{B_1}|\m{1}| = \int_{B_6}|\m{6}|$, hence $\int_{B_6}|\m{1}| \geq\int_{B_6}|\m{6}| = \int_{B_1}|\m{1}| $ which results in contradiction.\qed

\begin{proposition}
If  $m_1\,(m_6)$ concentrates within $T_1\,(T_6)$, then $\m{1} = \m{6} = 0,\, a.e. $ on $ S_\rho + \V{\pi}_1$.
\end{proposition}
\noindent{\it Proof.}
Consider frequency domain $S_\rho' = S_\rho\bigcap\{\omega_1<\omega_2\}.$ By Lemma \ref{lem: rank1}, $\exists\,\alpha_{\V{\omega}}\in\mathbb{C}, s.t.\,\mrow{1}(\V{\omega}) = \alpha_{\V{\omega}}\,\mrow{7}(\V{\omega}),\, \forall\, \V{\omega}\in S_\rho',$ i.e. $\widetilde{m_1}(\V{\omega} + \V{\pi}_1) = \alpha_{\V{\omega}}\cdot\widetilde{m_1}(\V{\omega} + \V{\pi}_7)$ and $\widetilde{m_6}(\V{\omega} + \V{\pi}_1) = \alpha_{\V{\omega}}\cdot\widetilde{m_6}(\V{\omega} + \V{\pi}_7)$. On the other hand, Lemma \ref{lem: concentrate} implies that $|\widetilde{m_1}(\V{\omega} + \V{\pi}_7)| \geq |\widetilde{m_6}(\V{\omega} + \V{\pi}_7)|$, hence $|\widetilde{m_1}(\V{\omega} + \V{\pi}_1)| \geq |\widetilde{m_6}(\V{\omega} + \V{\pi}_1)|$. Let $\Omega_6':= (S_\rho+\pi_1)\bigcap T_6$, then $\int_{\Omega_6'}|\m{1}| \geq\int_{\Omega_6'}|\m{6}|$, which will contradict Lemma \ref{lem: concentrate} unless $|\Omega_6'\bigcap\text{supp}(\widetilde{m_6})| = 0$, or equivalently $\alpha_{\V{\omega}}=0$ and so $\m{6} = \m{1} = 0,\,a.e.$ on $\Omega_6'$. By symmetry, $\m{6}=\m{1} = 0,\,a.e. $ on $(S_\rho+\V{\pi}_1)\setminus \Omega_6'$ as well.\qed

\begin{proposition}
$\m{1},\m{6}$ are not continuous at both $(\frac{\pi}{2},\frac{\pi}{2})$ and $(-\frac{\pi}{2},-\frac{\pi}{2})$.
\end{proposition}
\noindent{\it Proof}
If $\m{1}$ is continuous at $(\frac{\pi}{2},\frac{\pi}{2})$, then $\widetilde{m_1}(\frac{\pi}{2},\frac{\pi}{2}) = \lim_{\alpha\rightarrow 1^-}\widetilde{m_1}(\V{\omega}(\alpha)) = 0$, where $\{\V{\omega}(\alpha),\,0\leq \alpha<1\} \subset S_\rho + \V{\pi}_1$ and $\V{\omega}(1) = (\frac{\pi}{2},\frac{\pi}{2})$. By symmetry, we have $\widetilde{m_6}(\frac{\pi}{2},\frac{\pi}{2}) = 0$. Similarly, the continuity at $(-\frac{\pi}{2},-\frac{\pi}{2})$ implies $\widetilde{m_1}(-\frac{\pi}{2},-\frac{\pi}{2}) = \widetilde{m_6}(-\frac{\pi}{2},-\frac{\pi}{2}) = 0$. Therefore $\mrow{1}(0) = \mrow{7}(0) = \mathbf{0}$ which results in contradiction with Lemma \ref{lem: rank1}.\qed\\[1em]%and from \eqref{eq: m0C} $m_0^C(0)=0$ so that $m_0(0)=0$, %  On the other hand, Proposition\ref{prop: origin-det} implies that $m_0(0) = 0$ as $a = |\widetilde{m_1}(\pi_1)| = 0$, which results in contradiction.
The following theorem summarizes our main result.
\begin{theorem}\label{thm: thm}
If  $\m{i}$ concentrates in $T_i$ and $\m{1},\m{6}$ are symmetric to each other,  then  \eqref{eq: LS-new} doesn't have feasible solution given continuous $\m{1}$ and $\m{6}$.
\end{theorem}

\subsection{Design of input $\m{i}$}\label{sec: phase-design}
Following the orthonormal construction in \cite{yin2014orthshear}, we consider $\m{1},$ $\cdots,\m{6}$ in the form 
\begin{align}\label{eq: m-form}
\m{k} = e^{-i\V{\eta}_k^\top\V{\omega}}|\m{k}|,
\end{align}
 and $|\m{k}|$ have certain symmetry. We want to design the phase $\V{\eta}_k$ such that $m_0(\V{\omega}) > 0, \; \forall \omega\in S_1$. This is the same as requiring $\Msub$ to be full rank.
 We first show the necessary conditions on phases $\V{\eta}$ of the full rank requirement on $\Msub$.
 
\begin{lemma}\label{lem: phase-ineq}
If $\exists\,\V{\omega}\in D_1:=\{\omega_1=\omega_2,\,\omega_1\in(-\frac{\pi}{2},0)\},\,s.t. \,m_0(\V{\omega})>0,$ then $(\V{\eta}_1-\V{\eta}_6)^\top (\V{\pi}_6-\V{\pi}_7)\neq 0(\text{mod}\,2\pi)$. 
\end{lemma} 
\noindent {\it Proof}
  If $m_0(\V{\omega})>0, \,\V{\omega}\in D_1$ then $\Msub$ is full rank, hence its columns are linearly independent. Due to symmetry, $|\widetilde{m_1}(\V{\omega})| = |\widetilde{m_6}(\V{\omega})|$ on $\{\omega_1=\omega_2\}$. Let $A = |\widetilde{m_1}(\V{\omega}+\V{\pi}_7)| = |\widetilde{m_6}(\V{\omega}+\V{\pi}_7)|$ and $B=|\widetilde{m_1}(\V{\omega}+\V{\pi}_6)| = |\widetilde{m_6}(\V{\omega}+\V{\pi}_6)|$, then the first and the last columns of $\Msub$ are
  \begin{align*}
  \Msub[:,1] = 
 \begin{bmatrix}
 0\\
 \vdots\\
 0\\
 Ae^{i\V{\eta}_1^\top(\V{\omega}+\V{\pi}_6)}\\
 Be^{i\V{\eta}_1^\top(\V{\omega}+\V{\pi}_7)}
 \end{bmatrix}
 \quad\text{and}\quad
  \Msub[:,6] = 
 \begin{bmatrix}
 0\\
 \vdots\\
 0\\
 Ae^{i\V{\eta}_6^\top(\V{\omega}+\V{\pi}_6)}\\
 Be^{i\V{\eta}_6^\top(\V{\omega}+\V{\pi}_7)}
 \end{bmatrix} .
\end{align*}   
Therefore, $\Msub[:,1]$ and $\Msub[:,6]$ being linearly independent implies that \\$e^{i(\V{\eta}_1^\top\V{\pi}_6 + \V{\eta}_6^\top\V{\pi}_7)}\neq e^{i(\V{\eta}_6^\top\V{\pi}_6 + \V{\eta}_1^\top\V{\pi}_7)}$%$e^{i(\V{\eta}_1-\V{\eta}_6)^\top(\V{\omega}+\V{\pi}_6)}\neq e^{i(\V{\eta}_1-\V{\eta}_6)^\top(\V{\omega}+\V{\pi}_7)}$ 
or equivalently $(\V{\eta}_1-\V{\eta}_6)^\top(\V{\pi}_6-\V{\pi}_7)\neq 0(\text{mod}2\pi)$. \qed\\

Similarly, if $\exists\,\V{\omega}\in \{\omega_1 = \omega_2,\, \omega_1\in(0,\frac{\pi}{2})\},\, s.t.\, m_0(\V{\omega}) > 0$, then $(\V{\eta}_1-\V{\eta}_6)^\top (\V{\pi}_6-\V{\pi}_1)\neq 0(\text{mod}\,2\pi)$. These two conditions are equivalent to 
\begin{align*}
(\V{\eta}_1-\V{\eta}_6)^\top(\pi/2,\pi/2)\neq 0 (\text{mod}\,2\pi)\tag{\bf c1.1}
\end{align*}
given that $\V{\eta}_1$ and $\V{\eta}_6$ are integer phases in $\mathbb{Z}^2$.
Considering the other diagonal segment $\{\omega_2 = -\omega_1, |\omega_1| <\frac{\pi}{2}\}$, we have 
\begin{align*}
(\V{\eta}_3-\V{\eta}_4)^\top(-\pi/2,\pi/2)\neq 0 (\text{mod}\, 2\pi)\tag{\bf c1.2}
\end{align*}
%from the full rank condition.

Next, we investigate $\Msub$ at the origin, where the two diagonals meet.
\begin{proposition}\label{prop: origin-det}
If $|\widetilde{m_1}(\V{\pi}_1)| = |\widetilde{m_1}(\V{\pi}_7)| = |\widetilde{m_3}(\V{\pi}_3)| = |\widetilde{m_3}(\V{\pi}_5)|$ and $|\widetilde{m_1}(\V{\pi}_6)|= | \widetilde{m_3}(\V{\pi}_6)|$, then $\V{\pi}_1^\top(\V{\eta}_1-\V{\eta}_6)\neq \pi(\text{mod}\,2\pi)$ or $\V{\pi}_3^\top(\V{\eta}_3-\V{\eta}_4)\neq \pi(\text{mod}\,2\pi)$. 
\end{proposition}
\noindent{\it Proof.}
%$\Msub(\V{0})$ takes the following form
%$$\begin{bmatrix}
%* & 0 & 0 & 0 & 0 & *\\
%0 & * & 0 & 0 & 0 & 0\\
%0 & 0 & * & * & 0 & 0\\
%0 & 0 & 0 & 0 & * & 0\\
%0 & 0 & * & * & 0 & 0\\
%* & 0 & * & * & 0 & *\\
%* & 0 & 0 & 0 & 0 & *
%\end{bmatrix}$$
%The second and the fifth columns of $\Msub$ have single non-zero entry, $\widetilde{m_2}(\V{\pi}_2)$ and $\widetilde{m_5}(\V{\pi}_4)$ respectively, and are orthogonal to all the rest columns, hence the full-rank constraint of $\Msub$ is reduced to the full-rank constraint on its sub-matrix (with permutation of rows and columns)
 Since $\widetilde{m_0}(\V{0})\neq 0$, as shown in Lemma \ref{lem: rank1}, $rank(\mrow{1},\mrow{6},\mrow{7},\mrow{3},\mrow{5})= 4$ at $\V{\omega} = \V{0}$. This is equivalent to the matrix $\V{B}$ defined in \eqref{eq: matrix-B} to be full rank.
\begin{align}\label{eq: matrix-B}
%\mbox{\V{B}\strut}=
\V{B} = 
\begin{bmatrix}
& & & \\[-1em]
\widetilde{m_1}(\V{\pi}_6) & \widetilde{m_6}(\V{\pi}_6) & \widetilde{m_3}(\V{\pi}_6) & \widetilde{m_4}(\V{\pi}_6) \\
\widetilde{m_1}(\V{\pi}_1) & \widetilde{m_6}(\V{\pi}_1) & 0 & 0\\
\widetilde{m_1}(\V{\pi}_7) & \widetilde{m_6}(\V{\pi}_7) & 0 & 0\\
0 & 0 & \widetilde{m_3}(\V{\pi}_3) & \widetilde{m_4}(\V{\pi}_3)\\
0 & 0 & \widetilde{m_3}(\V{\pi}_5) & \widetilde{m_4}(\V{\pi}_5)\\
\end{bmatrix}
\end{align}
Let $|\widetilde{m_1}(\V{\pi}_1)| = |\widetilde{m_1}(\V{\pi}_7)| = |\widetilde{m_6}(\V{\pi}_1)| = |\widetilde{m_6}(\V{\pi}_7)| = |\widetilde{m_3}(\V{\pi}_3)| = |\widetilde{m_3}(\V{\pi}_5)| = |\widetilde{m_4}(\V{\pi}_3)| = |\widetilde{m_4}(\V{\pi}_5)|= a$ and $|\widetilde{m_1}(\V{\pi}_6)|=| \widetilde{m_6}(\V{\pi}_6)|= | \widetilde{m_3}(\V{\pi}_6)|=| \widetilde{m_4}(\V{\pi}_6)|=b$. Rewrite $\V{B}$ as follows,
$$\V{B}=
\begin{bmatrix}
b e^{-i\V{\pi}_6^\top\V{\eta}_1} & b e^{-i\V{\pi}_6^\top\V{\eta}_6} & b e^{-i\V{\pi}_6^\top\V{\eta}_3} & b e^{-i\V{\pi}_6^\top\V{\eta}_4}\\
a e^{-i\V{\pi}_1^\top\V{\eta}_1} & a e^{-i\V{\pi}_1^\top\V{\eta}_6} & 0						& 0 \\
a e^{i\V{\pi}_1^\top\V{\eta}_1} & a e^{i\V{\pi}_1^\top\V{\eta}_6} & 0						& 0 \\
0 					& 0 					& a e^{-i\V{\pi}_3^\top\V{\eta}_3} & a e^{-i\V{\pi}_3^\top\V{\eta}_4}\\
0 					& 0 					& a e^{i\V{\pi}_3^\top\V{\eta}_3} & a e^{i\V{\pi}_3^\top\V{\eta}_4}\\
\end{bmatrix}
$$
The product of singular values of $\V{B}$ is 
\begin{align}\label{eq: detB}
\sqrt{\text{det}(\V{B}^* \V{B})} = 4a^3\sqrt{a^2 K_1^2K_2^2 + b^2(Q_1K_2^2 + Q_2K_1^2)},
\end{align}
where $ Q_1 = 1 - \cos(\V{\pi}_6^\top(\V{\eta}_1-\V{\eta}_6))\cos(\V{\pi}_1^\top(\V{\eta}_1-\V{\eta}_6)), Q_2 = 1 - \cos(\V{\pi}_6^\top(\V{\eta}_3-\V{\eta}_4))\cos(\V{\pi}_3^\top(\V{\eta}_3-\V{\eta}_4)), K_1 = \sin(\V{\pi}_1^\top(\V{\eta}_1-\V{\eta}_6)), K_2 = \sin(\V{\pi}_3^\top(\V{\eta}_3-\V{\eta}_4)).$ If $\V{\pi}_1^\top(\V{\eta}_1-\V{\eta}_6) = \V{\pi}_3^\top(\V{\eta}_3-\V{\eta}_4) = \pi (mod\, 2\pi)$, then $K_1 = K_2 = 0$ and $\V{B}$ becomes singular.\qed\\
In Lemma \ref{lem: phase-ineq}, $\Msub[:,1]$ and $\Msub[:,6]$ being independent only guarantees $\det(\Msub[-k_{\V{\omega}},:])\neq 0$. However, \eqref{eq: m0-cramer} implies that $|m_0(\V{\omega})|\propto \det(\Msub[-k_{\V{\omega}},:])$ hence it is preferred to maximize the determinant. Since
\begin{align*}
\det(\Msub[-k_{\V{\omega}}, :]) = \det\big(\big[\, \Msub[-k_{\V{\omega}},-6], \;\Msub[-k_{\V{\omega}},6] + c \cdot\Msub[-k_{\V{\omega}},1]  \,\big]\big),\quad 
\end{align*}
$\forall c\in \mathbb{C}$, the angle between $\Msub[:,1]$ and $\Msub[:,6]$ should be maximized.
Therefore, a stronger condition than ({\bf c1.1}) is to require $\Msub[:,1]$ and $\Msub[:,6]$ be orthogonal, which is equivalent to 
\begin{align*}
(\V{\eta}_1-\V{\eta}_6)^\top(\pi/2, \pi/2) = \pi \,(\text{mod}\, 2\pi).\tag{\bf c2.1}
\end{align*}
The stronger condition corresponding to ({\bf c1.2}) is 
\begin{align*}
(\V{\eta}_3-\V{\eta}_4)^\top(-\pi/2,\pi/2)=\pi(\text{mod},\,2\pi).\tag{\bf c2.2}
\end{align*} %from the stronger orthogonal condition.

%{\it Remark}
%f $|\m{1}| = |\m{2}|$ on $\{\omega_y = 3\omega_x,\,|\omega_x| > \frac{\pi}{2}\}$ and $m_0(\V{\omega}) > 0$ on $\{\omega_y = 3\omega_x\pm \pi,\,|\omega_y| <\frac{\pi}{2}\}$, then the same conditions ({\bf c1}) and ({\bf c2}) can be derived from full rank and orthogonal conditions respectively for tuples $(\,\V{\eta}_1,\,\V{\eta}_2,(-\pi/2,\pi/2)\,),\,(\,\V{\eta}_2,\V{\eta}_3,(\pi/2,\pi/2)\,),\,(\V{\eta}_4,\V{\eta}_5,\,(\pi/2,\pi/2)\,)$ and $(\,\V{\eta}_5,\V{\eta}_6,\,(-\pi/2,\pi/2)\,)$. 

% If the previous strong orthogonal condition on $\V{\eta}_1, \V{\eta}_3, \V{\eta}_4,\V{\eta}_6$ holds, then $K_1 = K_2 = 0$ and $m_0(0)=m_0^C(0)= 0$. Therefore, the strong orthogonal conditions ({\bf c2}) cannot be satisfied at the same time. 
%In particular, we consider the following constraints on phase $\V{\eta}_k\in \mathbb{Z}^2,\, k = 1,\cdots,6$:
Unfortunately, Proposition \ref{prop: origin-det} prevents ({\bf c2.1}) and ({\bf c2.2}) from holding simultaneously.
We propose the following set of phases
\begin{align}\label{eq: phase-sol}
\V{\eta}_1 = (0,0),\; \V{\eta}_2 = (-1,1),\; \V{\eta}_3 = (0,2),\notag\\
\V{\eta}_4 = (1,0),\; \V{\eta}_5 = (0,-1),\; \V{\eta}_6 = (0,1).
\end{align}
where
\begin{align*}
%\label{eq: phase-constraint}
%&(\V{\eta}_1-\V{\eta}_2)^\top(-\pi/2, \pi/2) = (\V{\eta}_5-\V{\eta}_6)^\top(-\pi/2,\pi/2) = \pi\, (\text{mod}\, 2\pi)\notag\\
%&(\V{\eta}_2-\V{\eta}_3)^\top(\pi/2,\pi/2) = (\V{\eta}_4-\V{\eta}_5)^\top(\pi/2,\pi/2) = \pi\, (\text{mod}\, 2\pi)\\
%&
(\V{\eta}_3-\V{\eta}_4)^\top(-\pi/2, \pi/2) &=-\pi/2\,(\text{mod}\,2\pi)\notag\\
 (\V{\eta}_6 - \V{\eta}_1)^\top(\pi/2,\pi/2) &= \pi/2\, (\text{mod}\,2\pi)\notag
\end{align*}
%where we require strong orthogonal constraints on pair of shifts corresponding to $\widetilde{m}$ function with non-diagonal common boundary and weaker constraints on $(\V{\eta}_1,\V{\eta}_6)$ and $(\V{\eta}_3,\V{\eta}_4)$. A solution to \eqref{eq: phase-constraint} is 

\subsection{Computing $m_0$}\label{subsec: compute-m0}
\textcolor{red}{Remove this sub-section, or place it after the main result on discontinuity}.

Let $C_{\V{\omega}} = det(\M[-k_{\V{\omega}},:])$, then we have the following observation.
\begin{lemma}\label{lem: equal-det}
$C_{\V{\omega}} = C_{\V{\omega}+\V{\pi}_2} = C_{\V{\omega}+\V{\pi}_4} = C_{\V{\omega}+\V{\pi}_6}$
\end{lemma}
\noindent{\it Proof}
Because $\widetilde{M}(\V{\omega}+\V{\pi}_2) = P_{\V{\pi}_2}\M(\V{\omega})$ where $P_{\V{\pi}_2}$ is a row permutation matrix, it follows from the definition of $C_{\V{\omega}}$ that 
$C_{\V{\omega}} = det\big(\M[-k_{\V{\omega}},:](\V{\omega})\big) = det\big(\M[-k_{\V{\omega}+\V{\pi}_2},:](\V{\omega}+\V{\pi}_2) \big)= C_{\V{\omega}+\V{\pi}_2}$ where 
$\mathbf{1}_{k_{\V{\omega}+\V{\pi}_2}} = P_{\V{\pi}_2}\mathbf{1}_{k_{\V{\omega}}}$.
\qed\\[1em]
We assume that $m_0\in\mathbb{R}_{\geq 0}$ without phase. Let $m_0^C(\V{\omega}) = m_0(\V{\omega})|C_{\V{\omega}}|\in \mathbb{R}_{\geq 0}$ and $\mc{0} = \m{0}/|C_{\V{\omega}}|$, then Lemma \ref{lem: equal-det} implies the following.
\begin{proposition}\label{prop: mc}
$m_0(\V{\omega}),\,\m{0}, m_i(\V{\omega}),\,  i = 1,...,6$ satisfy \eqref{eq: LS-new} given $\m{i},\,i=1,...,6$ if and only if $m_0^C(\V{\omega}),$ $\,\mc{0}, m_i(\V{\omega}),\,i = 1,...,6$ do. More generally, $m_0^C(\V{\omega})c(\V{\omega}),\,\mc{0}c(\V{\omega})^{-1}, m_i(\V{\omega}),\,i=1,...,6$ satisfy \eqref{eq: LS-new} if $c(\V{\omega}) = c(\V{\omega}+\V{\pi}_2)=c(\V{\omega}+\V{\pi}_4) = c(\V{\omega}+\V{\pi}_6) \neq 0$.
\end{proposition}
According to Proposition \ref{prop: mc}, we can first solve $\mc{0}$ and $m_0^C(\V{\omega})$ and then construct $c(\V{\omega})$ for optimal $\m{0}$ and $m_0(\V{\omega})$. 
In particular, $m_0^C$ can be computed without knowing $k_{\V{\omega}}$,
\begin{align}\label{eq: m0C}
m_0^C(\V{\omega}) = m_0(\V{\omega})|C_{\V{\omega}}| = |det(\M_{1,1}[-k_{\V{\omega}},:])| = \prod_{i=1}^6\sigma_i(\M_{1,1}[-k_{\V{\omega}},:]) = \prod_{i=1}^6\sigma_i(\M_{1,1}).
\end{align}
In practice, we first perform QR decomposition on $\Msub:=\M_{1,1}$ and then take the absolute value of the product of the diagonal entries of the upper triangular matrix, $diag(R)$. 
We propose the following algorithm for bi-orthogonal directional filter construction with dilated quincunx downsampling scheme:
\begin{description}% prevent items from splitting
\item[construction of bi-orthogonal basis]\
\begin{itemize}
\item[Input:] $\m{i},\,i=1,...,6$
\item[1.] compute $m_0^C(\V{\omega}) = \left|det(\M_{1,1}[-k_{\V{\omega}},:])\right|$
\item[2.] compute $\mc{0}$, such that \eqref{eq: LS-new} is solvable and \eqref{eq: identity-cond} holds
\item[3.] solve $m_i(\V{\omega}),\, i=1,...,6$ according to \eqref{eq: LS-new}
\item[4.] design $c(\V{\omega})$ and let $m_0(\V{\omega}) = m_0^C(\V{\omega})c(\V{\omega}),\,\m{0} = \mc{0}\overline{c}(\V{\omega})^{-1}$
\end{itemize}
\end{description}



\subsection{solving $m_i$}
In the final step, we substitute $\mc{0}$ and $m_0^C(\V{\omega})$ into \eqref{eq: LS-new} and rewrite it into the following linear system,
\begin{align}\label{eq: mi}
\overline{\M}[:,2:7]\,\mathbf{m}[2:7](\V{\omega}) = 
\begin{bmatrix}
1-m_0^C\overline{\widetilde{m_0}^C}(\V{\omega})\\
0\\
-m_0^C\overline{\widetilde{m_0}^C}(\V{\omega}+\V{\pi}_2)\\
\vdots \\
0
\end{bmatrix}
=:\mathbf{b}(\V{\omega}).
\end{align}
The solution of \eqref{eq: mi} depends only on $m_0^C\overline{\widetilde{m_0}^C}$, or equivalently $m_0\overline{\widetilde{m_0}}$. 


\section{Numerical Experiments}
\subsection{solving $m_0^C$}
A set of $\m{i}$ that satisfy the conditions of Theorem \ref{thm: thm} with phase terms in \eqref{eq: phase-sol} is used as the input of \eqref{eq: LS-new}.
The left figure in Fig.\ref{fig: tm_i_m_0} shows the absolute value of $\m{i}$. In particular, $\m{i} =0,\,\forall \omega\in S_1$. 
We follow the construction process in Section \ref{subsec: compute-m0} and obtain $m_0^C$ shown in the right of Fig.\ref{fig: tm_i_m_0}, in both normal scale and log scale.  
We perform a numerical sanity check on the necessary condition in Proposition \ref{prop: feasibility}, that is $\forall\,\V{\omega}, \,s.t. [m_0(\V{\omega}), m_0(\V{\omega}+\V{\pi}_2),m_0(\V{\omega}+\V{\pi}_4),m_0(\V{\omega}+\V{\pi}_6)]$ is not a linear combination of the rows of $\mathfrak{D}(\V{\omega})$ in \eqref{eq: singular-cond}. Equivalently, we compute the following quantity $$\vartheta = 1 - \Vert V^\top\mathfrak{m}_0 \Vert/\Vert \mathfrak{m}_0\Vert ,$$ where $\mathfrak{m}_0(\V{\omega})=[m_0(\V{\omega}), m_0(\V{\omega}+\V{\pi}_2),m_0(\V{\omega}+\V{\pi}_4),m_0(\V{\omega}+\V{\pi}_6)]^\top$ and $V$ are the left singular vectors of $\mathfrak{D}(\omega)$ whose corresponding singular values are non-zero. If $\mathfrak{m}_0\in span(V)$, then $\vartheta = 0$. If $\mathfrak{m}_0\bot span(V)$, then $\vartheta = 1$.
Fig.\ref{fig: feasible} shows the feasibility check $\vartheta$ of input $\m{i}$, and $\mathfrak{m}_0$ is orthogonal to $span(V)$ everywhere.

\begin{figure}
\centering
\begin{minipage}[c]{.48\textwidth}
\includegraphics[width=\textwidth]{feasible_mi.pdf}
\end{minipage}
\begin{minipage}[c]{.22\textwidth}
\centering
\includegraphics[width=.8\textwidth]{feasible_m0.pdf}
\end{minipage}
\begin{minipage}[c]{.28\textwidth}
\centering
\includegraphics[width=.8\textwidth]{feasible_m0_log.pdf}
\end{minipage}
\caption{Left:  $|\m{i}|$, middle: computed $m_0^C$, right: $\log(m_0^C)$}
\label{fig: tm_i_m_0}
\end{figure}


\subsection{solving $\mc{0}$ and $m_i$}
We compute $\mc{0}$ by solving the following optimization problem similar to \eqref{eq: opt-diff} for the dyadic scheme,
\begin{align}
\min_{\xvec}\; \Vert \V{D}(\mathbf{m}_0^C\circ\xvec)\Vert^2 + \lambda\Vert \wvec\circ\mathbf{m}_0^C\circ\xvec\Vert^2,\quad 
s.t. \; A\xvec = \mathbf{1},\, \mathfrak{D}\xvec = \mathbf{0}
\label{eq: opt-2d}
\end{align}
where $\circ$ is Hadamard product and $\wvec$ is a weight vector and we consider real solution $\xvec$ here.
$A$ in the constraint is the matrix generated from the identity condition \eqref{eq: identity-cond} and $\mathfrak{D}$ is generated from the singularity condition \eqref{eq: singular-cond}. Since $A$ and $\mathfrak{D}$ are linearly independent, \eqref{eq: opt-2d} is feasible. Here, instead of optimizing the properties of $\xvec$ as in \eqref{eq: opt-diff}, we optimize those of $\widetilde{\mathbf{m}_0}^C\circ \xvec$ since $m_0^C \cdot\widetilde{m_0}^C$ will be later re-decomposed into $m_0$ and $\widetilde{m_0}$. In addition, if $m_0^C$ is symmetric with respect to the two coordinates $\omega_x$ and $\omega_y$, then we impose the same symmetry on $\widetilde{m_0}^C$ by solving \eqref{eq: opt-2d} on $[0,\pi)\times[0,\pi)$ and then extend the solution to $[-\pi,\pi)\times[-\pi,\pi)$ by symmetry.

\begin{figure}
\centering
\begin{minipage}[c]{.3\textwidth}
\includegraphics[width = .8\textwidth]{feasible_check.pdf}
\caption{$\vartheta$}\label{fig: feasible}
\end{minipage}
\begin{minipage}[c]{.63\textwidth}%{.28\textwidth}
\centering
\includegraphics[width = .38\textwidth]{feasible_tm0.pdf}\hspace*{2em}
\includegraphics[width = .42\textwidth]{feasible_m0tm0.pdf}
\caption{Left: : computed $\widetilde{m_0}^C$, right: $\widetilde{m_0}^C \cdot m_0^C $}
\label{fig: tm0}
\end{minipage}
\end{figure}

Fig.\ref{fig: tm0} shows $\mc{0}$ obtained from \eqref{eq: opt-2d} and $\widetilde{m_0}^C \cdot m_0^C$ which is $\mathbf{1}_{S_1}$.

In particular, given $\widetilde{m_0}^C \cdot m_0^C = 1$, $\mathbf{b}(\V{\omega}) = \mathbf{0}, \, \forall\,\V{\omega}\in S_1$, hence $\mathbf{m}[2:7] = \mathbf{0}$. 
When $\mathbf{b}(\V{\omega})\neq \mathbf{0}$, \eqref{eq: mi} is a degenerated over-determinant linear system (we also do a sanity check here for the linearity between $\overline{\M}[:,2:7]$ and $\mathbf{b}$ by computing $\vartheta$) and 
$$\mathbf{m}[2:7](\V{\omega}) = \Big(\overline{\M}[:,2:7]\Big)^\dagger\,\mathbf{b}(\V{\omega}),$$
where $\dagger$ is the pseudo-inverse of a matrix. Fig.\ref{fig: m_i} shows the solution $m_i$ of \eqref{eq: mi} and the corresponding spatial filters $\mathcal{F}^{-1}\widetilde{m_0}$.
As shown in Fig.\ref{fig: m_i}, the energy of $m_i$ concentrates on $\{|\omega_x| = \frac{\pi}{2},\, |\omega_y| = \frac{\pi}{2}\}$ where $|\widetilde{m_i}|$ is small, and the filters decay slowly in time domain.

The bi-orthogonal bases constructed is not ideal, despite the regularization on $m_0$ in the optimization \eqref{eq: opt-2d}. Since no explicit regularization is put on $m_i$, it's difficult to control the regularity of the output $m_i$ from the input $\widetilde{m_i}$.

\begin{figure}
\centering
\begin{minipage}[c]{.5\textwidth}
\includegraphics[width = .9\textwidth]{feasible_m.pdf}
\end{minipage}
\begin{minipage}[c]{.48\textwidth}
\includegraphics[width = .9\textwidth]{feasible_mi_time.pdf}
\end{minipage}
\caption{Left: $|m_i|,\, i = 1,\cdots,6$, right:$|\mathcal{F}^{-1}m_i|$}
\label{fig: m_i}
\end{figure}

\section{Conclusion and future work}\label{sec: end}
In this paper, we consider directional wavelet schemes on dilated quincunx sub-lattice and analyze their regularity. We show that filters in bi-orthogonal bases have the same discontinuity in the frequency domain as that of the orthonormal bases at the corners of $S_1 = [-\pi/2,\pi/2)\times[-\pi/2,\pi/2)$, hence they cannot be not well localized in the time domain. 

%Our analysis is closely related to our proposed bases construction algorithms, 
%and we show that the construction method of orthonormal bases can be easily extended to build frames construction of redundancy 2, which achieve much better time frequency localization and thus practically useful.
 We construct the bi-orthogonal taking a different approach from the orthonormal case, where directional dual filters are first designed such that they can be completed to a bi-orthogonal frame and the remaining filters are obtained by solving feasible linear systems or quadratic optimization. 
 Its extension to low-redundancy dual frame construction is not studied here and will be our future focus.
%\section{Conclusion and future work}\label{sec: end}
In this paper, we consider directional wavelet schemes on dilated quincunx sub-lattice and analyze their regularity. We show that filters in bi-orthogonal bases have the same discontinuity in the frequency domain as that of the orthonormal bases at the corners of $S_1 = [-\pi/2,\pi/2)\times[-\pi/2,\pi/2)$, hence they cannot be not well localized in the time domain. 

%Our analysis is closely related to our proposed bases construction algorithms, 
%and we show that the construction method of orthonormal bases can be easily extended to build frames construction of redundancy 2, which achieve much better time frequency localization and thus practically useful.
 We construct the bi-orthogonal taking a different approach from the orthonormal case, where directional dual filters are first designed such that they can be completed to a bi-orthogonal frame and the remaining filters are obtained by solving feasible linear systems or quadratic optimization. 
 Its extension to low-redundancy dual frame construction is not studied here and will be our future focus.
%\section{Conclusion}\label{sec: end}

\begin{appendices}
\section{Proof of Theorem \ref{thm: conds}}\label{app: cond-thm}
Take the Fourier transform of both sides of \eqref{eq: PR}, we have 
\begin{align*}
\sum_{\V{k}}\langle f,\phi_{\V{k}}\rangle\hat{\phi}(\V{\omega})e^{-i\V{\omega}^T\V{k}} = \sum_{\V{k}}&\langle f,\phi_{1,\V{k}}\rangle e^{-i\V{\omega}^T\V{D_2k}}|\V{D_2}|^{1/2}\hat{\phi}(\V{D_2}^T\V{\omega}) \\
&+ \sum_{j=1}^J\sum_{\V{k}}\langle f,\psi^j_{1,\V{k}}\rangle e^{-i\V{\omega}^T\V{Dk}}|\V{D}|^{1/2}\hat{\phi}(\V{D}^T\V{\omega})
\end{align*}
Suppose $m_j$ are trigonometric series
\begin{align}\label{eq: mra1}
m_0(\V{\omega}) = \sum_{\V{k}} c_{\V{k}}e^{-i \V{\omega}^T\V{k}} \quad
m_j(\V{\omega}) = \sum_{\V{k}} g_{\V{k}}e^{-i \V{\omega}^T\V{k}},\quad j=1,\cdots,J
\end{align}
The first term on the right hand side can be represented by $\hat{\phi}(\V{\omega})$ and $\langle f,\phi_k\rangle$ using \eqref{eq: m0} and \eqref{eq: mra1}.

\begin{align*}
\text{the first term on R.H.S. } = \sum_{\V{k}}\langle f,\phi_{1,\V{k}}\rangle e^{-i\V{\omega}^T\V{D_2k}}|\V{D_2}|^{1/2}m_0(\V{\omega})\hat{\phi}(\V{\omega}) \\= \sum_{\V{k}}\Big(\sum_{\V{k}'}\langle f,\phi_{\V{k}'}\rangle\overline{c_{\V{k'-D_2k}}}|\V{D_2}|^{1/2}\Big)e^{-i\V{\omega}^T\V{D_2k}}|\V{D_2}|^{1/2}m_0(\V{\omega})\hat{\phi}(\V{\omega})\\
=\sum_{\V{k}'}\langle f,\phi_{\V{k}'}\rangle\Big(|\V{D_2}|\sum_{\V{k}}\overline{c_{\V{k'-D_2k}}}e^{i\V{\omega}^T(\V{k'-D_2k})}\Big)e^{-i\V{\omega} ^T\V{k}'} m_0(\V{\omega})\hat{\phi}(\V{\omega}).
\end{align*}
{\it Remark}.
If we have a shift $\V{k}_0$ in the down-sample scheme, i.e. $\V{D_2}\mathbb{Z}^2 - \V{k}_0$ instead of $\V{D_2}\mathbb{Z}^2$, so that we obtain coefficient of $\tilde{\phi}_{1,\V{k}} = \phi_{1,\V{k}+\V{k}_0}$ instead of $\phi_{1,\V{k}}$, and $\tilde{\phi}_1(\V{x}) =\phi_1(\V{x}-\V{k}_0)= |\V{D_2}|^{1/2}\sum_{\V{k}}c_{\V{k}}\phi(\V{x-k-k}_0) = |\V{D_2}|^{1/2}\sum_{\V{k}}c_{\V{k}-\V{k}_0}\phi(\V{x-k})$. This change of down-sample scheme results in an extra phase term $e^{-i\V{\omega}^T \V{k}_0}$ in $m_0$. Here, we use the down-sample scheme without translation.

Since $\bigcup_{\V{\beta}\in B} \{\V{\beta}\} :=\bigcup_{\V{\beta}\in B}(\V{D_2}\mathbb{Z}^2+\V{\beta}) = \mathbb{Z}^2$, where $B = \{ (0,0),\,(1,0),\,(0,1),\,(1,1)\}$, the summation over $\V{k}'\in \mathbb{Z}^2$ can be written as a double sum $\sum_{\V{\beta}\in B}\sum_{\V{k}'\in \{\V{\beta}\}}$,
\begin{align*}
\sum_{\V{\beta}\in B}\sum_{\V{k}'\in\{\V{\beta}\}} \langle f,\phi_{\V{k}'}\rangle\sum_{\V{k}}\overline{c_{\V{k'-D_2k}}}e^{i\V{\omega}^T(\V{k}'-\V{D_2k})}e^{-i\V{\omega}^T\V{k}'}|\V{D_2}|m_0(\V{\omega})\hat{\phi}(\V{\omega})\\
=\sum_{\V{\beta}\in B}\sum_{\V{k}'\in\{\V{\beta}\}} \langle f,\phi_{\V{k}'}\rangle\sum_{\V{k}\in\{\V{\beta}\}}\overline{c_{\V{k}}}e^{i\V{\omega}^T\V{k}}e^{-i\V{\omega}^T\V{k}'}|\V{D_2}|m_0(\V{\omega})\hat{\phi}(\V{\omega})
\end{align*}
The summation over $\V{k}$ in the middle is similar to the trigonometric form of $m_0$ in \eqref{eq: mra1}, but $\V{k}$ takes value on the shifted sub-lattice $\{\V{\beta}\}$ instead of $\mathbb{Z}^2$. Therefore, the summation equals to instead a linear combination of $m_0$ with shifts $\Gamma_0$,
\begin{align}\label{eq:eq1}
\sum_{\V{\pi}\in\Gamma_0}m_0(\V{\omega}+\V{\pi})\;e^{i\V{\beta}^T\V{\pi}} = \sum_{\V{k}\in \{\V{\beta}\}}c_{\V{k}}e^{-i\V{\omega}^T\V{k}}
\end{align}
Substitute \eqref{eq:eq1} into the previous expression,
\begin{align*}
\sum_{\V{\beta}\in B}\sum_{\V{k}'\in \{\V{\beta}\}}\langle f,\phi_{\V{k}'}\rangle\sum_{\V{\pi}\in\Gamma_0}\overline{m_0(\V{\omega}+\V{\pi})\;}e^{-i\V{\beta}^T\V{\pi}}\,e^{-i\V{\omega}^T\V{k}'}m_0(\V{\omega})\hat{\phi}(\V{\omega})
\end{align*}
Since $e^{i \V{\pi}^{T}\V{\beta}}=e^{i\V{\pi}^T\V{k}'},\; \forall \V{k}'\in \{\V{\beta}\} $, after rewriting the double sum over $\V{k}'$ back to a unit sum on $\mathbb{Z}^2$, we get
\begin{align*}
\sum_{\V{k}'}\langle f,\phi_{\V{k}'}\rangle e^{-i\V{\omega} ^T\V{k}'}\hat{\phi}(\V{\omega})\Big(\sum_{\V{\pi}\in\Gamma_0}\overline{m_0(\V{\omega}+\V{\pi})}m_0(\V{\omega})e^{-i\V{\pi}^T\V{k}'} \Big)
\end{align*}

Similarly, the second term on the R.H.S. of \eqref{eq: PR} equals to 
\begin{align*}
\sum_{j=1}^J\sum_{\V{k}'}\langle f,\phi_{\V{k}'}\rangle e^{-i\V{\omega}^T \V{k}'}\hat{\phi}(\V{\omega})\Big(\sum_{\V{\pi}\in\Gamma_1} \overline{m_j(\V{\omega}+\V{\pi})}m_j(\V{\omega})e^{-i\V{\pi}^T\V{k}'} \Big)
\end{align*}
(For Theorem \ref{thm: frame-conds} on frame construction, the summation of shifts $\V{\pi}$ is over $\Gamma_0$ instead of $\Gamma_1$.) 
Combining the two terms on the R.H.S. of \eqref{eq: PR}, and compare the coefficients of $\langle f,\phi_{\V{k}'}\rangle e^{-i\V{\omega}^T \V{k}'}\hat{\phi}(\V{\omega})$ on both sides, the perfect reconstruction condition is then equivalent to $\forall \V{k}'$,
\begin{align*}
\sum_{\V{\pi}\in\Gamma_0}e^{-i\V{\pi}^T\V{k}'}\overline{m_0(\V{\omega}+\V{\pi})}m_0(\V{\omega}) + \sum_j\sum_{\V{\pi}\in\Gamma_1} e^{-i\V{\pi}^T\V{k}'}\overline{m_j(\V{\omega}+\V{\pi})}m_j(\V{\omega}) = 1. 
%\sum_{l=0}^{3}e^{-i\gamma_l^T(k'-k_0)}\overline{M_0(\xi+\gamma_l)}M_0(\xi) + \sum_j\sum_{s=0}^7 e^{-i\nu_s^T(k'-k_j)}\overline{M_j(\xi+\nu_s)}M_j(\xi) = 1. 
\end{align*} 
This is equivalent to 
\begin{align*}
&|m_0(\V{\omega})|^2 + \sum_j|m_j(\V{\omega})|^2 = 1
\end{align*}
and
\begin{align*}
\sum_{j=0}^J\overline{m_j(\V{\omega}+\V{\pi})}m_j(\V{\omega}) = 0, 
%+ \overline{m_0(\V{\omega}+\V{\pi})}m_0(\V{\omega}) = 0, 
\,\V{\pi}\in \Gamma_0\setminus\{\V{0}\}\\
\sum_{j=1}^J\overline{m_j(\V{\omega}+\V{\pi})}m_j(\V{\omega}) = 0,\,\V{\pi}\in \Gamma_1\setminus \Gamma_0
\end{align*}

{\it Remark}.
Because each $m_j$ is $(2\pi,2\pi)$ periodic, we only need to check the above equality $\forall \V{\omega}\in S_0$.
If we downsample $\psi_1^j$ on a shifted sub-lattice $\V{D}\mathbb{Z}^2-\V{k}_j$, we then have an extra phase $e^{i\V{\pi}^T\V{k}_j}$ before $\overline{m_j(\V{\omega}+\V{\pi})}m_j(\V{\omega})$ in shift cancellation condition. This provides additional freedom in the construction yet it is not substantial.

\section{Supplementary Numerical Results}\label{app: supp-numerical}
\subsection{Numerical optimization of $\m{0}$ in 1D}\label{subsec: 1D-opt}
To test whether numerical optimization is a practical way to solve \eqref{eq: identity-cond}, we experiment on $m_0$ and $\widetilde{m_0}$ of existing real biorthogonal wavelets. We consider a pair of low frequency filters corresponding to biorthogonal scaling functions $\phi,\, \tilde{\phi}$ with vanishing moments 3 and 5 respectively. 

\begin{wrapfigure}{r}{.4\textwidth}
\includegraphics[width = .4\textwidth]{filters.jpg}
\caption{1D filters, up: LoD, down: LoR}
\label{fig: filters}
\end{wrapfigure}
The 1D filters are shown in Figure \ref{fig: filters}. Suppose we know the decomposition filter, and we want to find the real reconstruction filter, such that it has support as concentrated as possible. 
%The corresponding $m_0$ and $\widetilde{m_0}$ are complex, yet we can shift the phase of $m_0$ such that $m_0$ is real and apply the same phase shift to $\m{0}$. 
%Without loss of generality, \eqref{eq: identity-cond} can be solved assuming that $m_0$ is real.
%It is not necessary that the corredponding $\widetilde{m_0}$ is also real, but in this testing case, $m_0$ and $\widetilde{m_0}$ are both real.
%have the same phase, hence the phase-shifted $\m{0}$ is real as well. 
Figure \ref{fig: m-funcs} shows the ground truth $m_0$ and $\widetilde{m_0}$ considered in this simulation. %and in particular, $|m_0|$ is used as the known coefficients in \eqref{eq: bi-orth-eq}. Hereafter, we use $m_0(\omega)$ and $\m{0}$ to denote the real-valued functions.
\begin{figure}%{l}{.4\textwidth}
\begin{minipage}[t]{.45\textwidth}
\includegraphics[width = \linewidth]{m-funcs.jpg}
\caption{$m_0(\omega)$ and $\widetilde{m_0}(\omega)$}
\label{fig: m-funcs}
\end{minipage}
\hfill
\begin{minipage}[t]{.45\textwidth}
\vbox{
\includegraphics[width = \textwidth]{1d-m-compare.jpg}\\
\includegraphics[width = \textwidth]{1d-filter-compare.jpg}
}
\caption{$\widehat{\widetilde{m_0}}$ vs. $\widetilde{m_0}$, top: frequency domain, bottom: time domain}
\label{fig: 1d-compare}
\end{minipage}
\end{figure}

Let $\mhat{0}$ be the approximation of $\m{0}$, which is solution of the following optimization problem
\begin{align}
\min_{\xvec}\; \Vert \V{D}\xvec\Vert^2 + \Vert \xvec\Vert^2,\quad s.t. \; \V{A}\xvec = \mathbf{1} \label{eq: opt-1d}
\end{align}
where $\V{A}$ in the constraint is the matrix generated from \eqref{eq: identity-cond} (in 1D, only a single shift of $\pi$ appears in the condition, so each row of $\V{A}$ has two non-zero entries). 
%Notice that no symmetry constraint is imposed here, nevertheless, 
Figure \ref{fig: 1d-compare} compares the solution of \eqref{eq: opt-1d} and the ground truth. The support of the solution is slightly more spread out than the ground truth.
%The support of the solution shown in Fig.\ref{fig: 1d-compare} is almost symmetric. On the other hand, its support in the time domain is not as compact as that of $\m{0}$, see the bottom of Fig.\ref{fig: 1d-compare}.

\subsection{Numerical optimization of $\m{0}$ in 2D}
In the 2D case, we use the pair of biorthogonal low-pass filters that are the tensor products of the 1D filters in Section \ref{subsec: 1D-opt} as ground truth. We solve the 2D version of the optimization problem \eqref{eq: opt-1d}. Figure \ref{fig: 2d-compare-1} shows the solution and compares it with the ground truth. 

%{\it 2D version of \eqref{eq: opt-1d}}\\

%The 1D formulation can be easily extended to 2D, where $\V{D} = [\V{D}_x,\V{D}_y]$ consider 1st order derivative in both $x$ and $y$ directions, and $\V{A}$ is generated from \eqref{eq: identity-cond}, each row has four non-zero entries. 
%It is obvious that the solution is not $90^\circ$-rotation invariant. Even worse is the fact that there is much energy in the vertical high-frequency domain.

%{\it weighted L2 norm (Modulation Space$^{[\ref{app: modulation}]}$)}\\
To make the support of $\mhat{0}$ better concentrate within the low frequency domain, we change the squared $\ell_2$-norm penalty in \eqref{eq: opt-1d} to a weighted version (corresponding to Modulation space) as follows,
\begin{align}
\min_{\xvec}\; \Vert\V{ D}\xvec\Vert^2 + \lambda\Vert \wvec\circ\xvec\Vert^2,\quad s.t. \; \V{A}\xvec = \mathbf{1} \label{eq: opt-2d-weight}
\end{align} 
where $\circ$ is Hadamard product and $\wvec$ is a weight vector. In particular, we choose $\forall \V{\omega}, \; \wvec(\V{\omega}) = |\V{\omega}|$. Figure \ref{fig: 2d-compare-2.2} shows the solution of \eqref{eq: opt-2d-weight} with $\lambda=600$. % and $600$ respectively. As $\lambda$ increases, the support of the minimizer concentrates more within the low frequency region. As shown in Fig.\ref{fig: 2d-compare-2}, when $\lambda$ is not huge, the minimizer achieves a certain level of but not full symmetry, whereas Fig.\ref{fig: 2d-compare-2.2} shows that huge $\lambda$ imposes full symmetry.

Compared to \eqref{eq: opt} proposed to solve $\m{0}$, both optimization problems \eqref{eq: opt-1d} and \eqref{eq: opt-2d-weight} in this simulation minimize the squared $\ell_2$-norm of the gradient of $\widetilde{m_0}$ but have an extra (weighted) $\ell_2$ regularization term. Although \eqref{eq: opt-1d} and \eqref{eq: opt-2d-weight} work better than \eqref{eq: opt} for 1D and 2D tensor wavelet construction here, they do not provide solutions with better regularity in the construction of biorthogonal directional wavelets while increasing the computation cost.

\begin{comment}
\begin{minipage}{.9\textwidth}
\centering
\includegraphics[width = .9\textwidth]{2d-m-compare-2-1-eps-converted-to.pdf}\\
\includegraphics[width = .9\textwidth]{2d-filter-compare-2-1-eps-converted-to.pdf}
\captionof{figure}{result of \eqref{eq: opt-2d-weight} $\mhat{0}$ ($\lambda = 60$), target $\m{0}$ and their difference, Top: frequency domain, Bottom: time domain}
\label{fig: 2d-compare-2}
\end{minipage}
\end{comment}

\begin{minipage}{.9\textwidth}
\centering
\includegraphics[width = \textwidth]{2d-m-compare-vanilla.png}
\captionof{figure}{Left to right: solution of \eqref{eq: opt-1d} in 2D, ground truth and their difference}
\label{fig: 2d-compare-1}
\end{minipage}\\
\vspace*{2em}
\begin{minipage}{.9\textwidth}
\centering
\includegraphics[width = \textwidth]{2d-m-compare-weightedl2.png}\\
\includegraphics[width = \textwidth]{2d-filter-compare-weightedl2.png}
\captionof{figure}{Left to right: solution of \eqref{eq: opt-2d-weight} ($\lambda = 600$), ground truth and their difference; Top: frequency domain, bottom: time domain.}
\label{fig: 2d-compare-2.2}
\end{minipage}
\\[1em]
\begin{comment}
{\it weighted L2 norm with symmetry constraint}\\
If we hard constrain the symmetry by the following
\begin{align}
\min_{\xvec}\; \Vert \V{D}\xvec\Vert^2 + \lambda\Vert \wvec\circ\xvec\Vert^2,\quad s.t. \; \V{A}\xvec = \mathbf{1},\,\V{S}\xvec = \mathbf{0} \label{eq: opt-2d-weight-sym}
\end{align}
where each row of $\V{S}$ has an one entry and a negative one entry at the location of two points have the same value due to symmetry. In practice, we put symmetry constraints such that the upper half plane is symmetric to the lower half plane w.r.t. $x$ coordinate and the first quadrant is $90^{\circ}-$ rotational invariant w.r.t. the second quadrant. The symmetry constraint makes the optimization problem significantly harder, resulting in longer optimization algorithm running time and no near-optimal solution is found (the algorithm terminates as the maximum number of iterations is exceeded). Figure \ref{fig: 2d-compare-3} shows the result provided by the Matlab quadratic minimization solver, unfortunately, there are artifacts at the near endpoints of $x$ and $y$ coordinates.
\begin{figure}[h]
\includegraphics[width = .9\textwidth]{2d-m-compare-3-eps-converted-to.pdf}\\
\includegraphics[width = .9\textwidth]{2d-m-compare-3-2-eps-converted-to.pdf}
\caption{solution of \eqref{eq: opt-2d-weight-sym} (top: $\lambda=60$, bottom: $\lambda=600$) provided by Matlab solver \texttt{quadprog}}
\label{fig: 2d-compare-3}
\end{figure}
\\
On the other hand, asymmetric solution can always be symmetrized by the average of the solution and its dual
w.r.t. rotation, mirroring, etc. This approach increase the support of the solution, thus a well concentrated solution in the frequency domain is necessary to begin with.
\\[1em]
{\it Other potential formulations}\\
We may also putting weights in the first L2-norm of derivatives, such that
\begin{align}
\min_{\xvec}\; \Vert \wvec'\circ \V{D}\xvec\Vert^2 + \lambda\Vert \wvec\circ\xvec\Vert^2,\quad s.t. \; \V{A}\xvec = \mathbf{1} \label{eq: opt-2d-double-weight}
\end{align}
Clearly, $\wvec'(\V{\omega})\rightarrow +\infty$ as $|\V{\omega}|\rightarrow +\infty$, but its behavior near the origin is unclear.
\end{comment}
%\section{Proof of Theorem \ref{thm: conds}}\label{app: cond-thm}
Take the Fourier transform of both sides of \eqref{eq: PR}, we have 
\begin{align*}
\sum_{\V{k}}\langle f,\phi_{\V{k}}\rangle\hat{\phi}(\V{\omega})e^{-i\V{\omega}^T\V{k}} = \sum_{\V{k}}&\langle f,\phi_{1,\V{k}}\rangle e^{-i\V{\omega}^T\V{D_2k}}|\V{D_2}|^{1/2}\hat{\phi}(\V{D_2}^T\V{\omega}) \\
&+ \sum_{j=1}^J\sum_{\V{k}}\langle f,\psi^j_{1,\V{k}}\rangle e^{-i\V{\omega}^T\V{Dk}}|\V{D}|^{1/2}\hat{\phi}(\V{D}^T\V{\omega})
\end{align*}
Suppose $m_j$ are trigonometric series
\begin{align}\label{eq: mra1}
m_0(\V{\omega}) = \sum_{\V{k}} c_{\V{k}}e^{-i \V{\omega}^T\V{k}} \quad
m_j(\V{\omega}) = \sum_{\V{k}} g_{\V{k}}e^{-i \V{\omega}^T\V{k}},\quad j=1,\cdots,J
\end{align}
The first term on the right hand side can be represented by $\hat{\phi}(\V{\omega})$ and $\langle f,\phi_k\rangle$ using \eqref{eq: m0} and \eqref{eq: mra1}.

\begin{align*}
\text{the first term on R.H.S. } = \sum_{\V{k}}\langle f,\phi_{1,\V{k}}\rangle e^{-i\V{\omega}^T\V{D_2k}}|\V{D_2}|^{1/2}m_0(\V{\omega})\hat{\phi}(\V{\omega}) \\= \sum_{\V{k}}\Big(\sum_{\V{k}'}\langle f,\phi_{\V{k}'}\rangle\overline{c_{\V{k'-D_2k}}}|\V{D_2}|^{1/2}\Big)e^{-i\V{\omega}^T\V{D_2k}}|\V{D_2}|^{1/2}m_0(\V{\omega})\hat{\phi}(\V{\omega})\\
=\sum_{\V{k}'}\langle f,\phi_{\V{k}'}\rangle\Big(|\V{D_2}|\sum_{\V{k}}\overline{c_{\V{k'-D_2k}}}e^{i\V{\omega}^T(\V{k'-D_2k})}\Big)e^{-i\V{\omega} ^T\V{k}'} m_0(\V{\omega})\hat{\phi}(\V{\omega}).
\end{align*}
{\it Remark}.
If we have a shift $\V{k}_0$ in the down-sample scheme, i.e. $\V{D_2}\mathbb{Z}^2 - \V{k}_0$ instead of $\V{D_2}\mathbb{Z}^2$, so that we obtain coefficient of $\tilde{\phi}_{1,\V{k}} = \phi_{1,\V{k}+\V{k}_0}$ instead of $\phi_{1,\V{k}}$, and $\tilde{\phi}_1(\V{x}) =\phi_1(\V{x}-\V{k}_0)= |\V{D_2}|^{1/2}\sum_{\V{k}}c_{\V{k}}\phi(\V{x-k-k}_0) = |\V{D_2}|^{1/2}\sum_{\V{k}}c_{\V{k}-\V{k}_0}\phi(\V{x-k})$. This change of down-sample scheme results in an extra phase term $e^{-i\V{\omega}^T \V{k}_0}$ in $m_0$. Here, we use the down-sample scheme without translation.

Since $\bigcup_{\V{\beta}\in B} \{\V{\beta}\} :=\bigcup_{\V{\beta}\in B}(\V{D_2}\mathbb{Z}^2+\V{\beta}) = \mathbb{Z}^2$, where $B = \{ (0,0),\,(1,0),\,(0,1),\,(1,1)\}$, the summation over $\V{k}'\in \mathbb{Z}^2$ can be written as a double sum $\sum_{\V{\beta}\in B}\sum_{\V{k}'\in \{\V{\beta}\}}$,
\begin{align*}
\sum_{\V{\beta}\in B}\sum_{\V{k}'\in\{\V{\beta}\}} \langle f,\phi_{\V{k}'}\rangle\sum_{\V{k}}\overline{c_{\V{k'-D_2k}}}e^{i\V{\omega}^T(\V{k}'-\V{D_2k})}e^{-i\V{\omega}^T\V{k}'}|\V{D_2}|m_0(\V{\omega})\hat{\phi}(\V{\omega})\\
=\sum_{\V{\beta}\in B}\sum_{\V{k}'\in\{\V{\beta}\}} \langle f,\phi_{\V{k}'}\rangle\sum_{\V{k}\in\{\V{\beta}\}}\overline{c_{\V{k}}}e^{i\V{\omega}^T\V{k}}e^{-i\V{\omega}^T\V{k}'}|\V{D_2}|m_0(\V{\omega})\hat{\phi}(\V{\omega})
\end{align*}
The summation over $\V{k}$ in the middle is similar to the trigonometric form of $m_0$ in \eqref{eq: mra1}, but $\V{k}$ takes value on the shifted sub-lattice $\{\V{\beta}\}$ instead of $\mathbb{Z}^2$. Therefore, the summation equals to instead a linear combination of $m_0$ with shifts $\Gamma_0$,
\begin{align}\label{eq:eq1}
\sum_{\V{\pi}\in\Gamma_0}m_0(\V{\omega}+\V{\pi})\;e^{i\V{\beta}^T\V{\pi}} = \sum_{\V{k}\in \{\V{\beta}\}}c_{\V{k}}e^{-i\V{\omega}^T\V{k}}
\end{align}
Substitute \eqref{eq:eq1} into the previous expression,
\begin{align*}
\sum_{\V{\beta}\in B}\sum_{\V{k}'\in \{\V{\beta}\}}\langle f,\phi_{\V{k}'}\rangle\sum_{\V{\pi}\in\Gamma_0}\overline{m_0(\V{\omega}+\V{\pi})\;}e^{-i\V{\beta}^T\V{\pi}}\,e^{-i\V{\omega}^T\V{k}'}m_0(\V{\omega})\hat{\phi}(\V{\omega})
\end{align*}
Since $e^{i \V{\pi}^{T}\V{\beta}}=e^{i\V{\pi}^T\V{k}'},\; \forall \V{k}'\in \{\V{\beta}\} $, after rewriting the double sum over $\V{k}'$ back to a unit sum on $\mathbb{Z}^2$, we get
\begin{align*}
\sum_{\V{k}'}\langle f,\phi_{\V{k}'}\rangle e^{-i\V{\omega} ^T\V{k}'}\hat{\phi}(\V{\omega})\Big(\sum_{\V{\pi}\in\Gamma_0}\overline{m_0(\V{\omega}+\V{\pi})}m_0(\V{\omega})e^{-i\V{\pi}^T\V{k}'} \Big)
\end{align*}

Similarly, the second term on the R.H.S. of \eqref{eq: PR} equals to 
\begin{align*}
\sum_{j=1}^J\sum_{\V{k}'}\langle f,\phi_{\V{k}'}\rangle e^{-i\V{\omega}^T \V{k}'}\hat{\phi}(\V{\omega})\Big(\sum_{\V{\pi}\in\Gamma_1} \overline{m_j(\V{\omega}+\V{\pi})}m_j(\V{\omega})e^{-i\V{\pi}^T\V{k}'} \Big)
\end{align*}
(For Theorem \ref{thm: frame-conds} on frame construction, the summation of shifts $\V{\pi}$ is over $\Gamma_0$ instead of $\Gamma_1$.) 
Combining the two terms on the R.H.S. of \eqref{eq: PR}, and compare the coefficients of $\langle f,\phi_{\V{k}'}\rangle e^{-i\V{\omega}^T \V{k}'}\hat{\phi}(\V{\omega})$ on both sides, the perfect reconstruction condition is then equivalent to $\forall \V{k}'$,
\begin{align*}
\sum_{\V{\pi}\in\Gamma_0}e^{-i\V{\pi}^T\V{k}'}\overline{m_0(\V{\omega}+\V{\pi})}m_0(\V{\omega}) + \sum_j\sum_{\V{\pi}\in\Gamma_1} e^{-i\V{\pi}^T\V{k}'}\overline{m_j(\V{\omega}+\V{\pi})}m_j(\V{\omega}) = 1. 
%\sum_{l=0}^{3}e^{-i\gamma_l^T(k'-k_0)}\overline{M_0(\xi+\gamma_l)}M_0(\xi) + \sum_j\sum_{s=0}^7 e^{-i\nu_s^T(k'-k_j)}\overline{M_j(\xi+\nu_s)}M_j(\xi) = 1. 
\end{align*} 
This is equivalent to 
\begin{align*}
&|m_0(\V{\omega})|^2 + \sum_j|m_j(\V{\omega})|^2 = 1
\end{align*}
and
\begin{align*}
\sum_{j=0}^J\overline{m_j(\V{\omega}+\V{\pi})}m_j(\V{\omega}) = 0, 
%+ \overline{m_0(\V{\omega}+\V{\pi})}m_0(\V{\omega}) = 0, 
\,\V{\pi}\in \Gamma_0\setminus\{\V{0}\}\\
\sum_{j=1}^J\overline{m_j(\V{\omega}+\V{\pi})}m_j(\V{\omega}) = 0,\,\V{\pi}\in \Gamma_1\setminus \Gamma_0
\end{align*}

{\it Remark}.
Because each $m_j$ is $(2\pi,2\pi)$ periodic, we only need to check the above equality $\forall \V{\omega}\in S_0$.
If we downsample $\psi_1^j$ on a shifted sub-lattice $\V{D}\mathbb{Z}^2-\V{k}_j$, we then have an extra phase $e^{i\V{\pi}^T\V{k}_j}$ before $\overline{m_j(\V{\omega}+\V{\pi})}m_j(\V{\omega})$ in shift cancellation condition. This provides additional freedom in the construction yet it is not substantial.

%\section{Supplementary Numerical Results}\label{app: supp-numerical}
\subsection{Numerical optimization of $\m{0}$ in 1D}\label{subsec: 1D-opt}
To test whether numerical optimization is a practical way to solve \eqref{eq: identity-cond}, we experiment on $m_0$ and $\widetilde{m_0}$ of existing real biorthogonal wavelets. We consider a pair of low frequency filters corresponding to biorthogonal scaling functions $\phi,\, \tilde{\phi}$ with vanishing moments 3 and 5 respectively. 

\begin{wrapfigure}{r}{.4\textwidth}
\includegraphics[width = .4\textwidth]{filters.jpg}
\caption{1D filters, up: LoD, down: LoR}
\label{fig: filters}
\end{wrapfigure}
The 1D filters are shown in Figure \ref{fig: filters}. Suppose we know the decomposition filter, and we want to find the real reconstruction filter, such that it has support as concentrated as possible. 
%The corresponding $m_0$ and $\widetilde{m_0}$ are complex, yet we can shift the phase of $m_0$ such that $m_0$ is real and apply the same phase shift to $\m{0}$. 
%Without loss of generality, \eqref{eq: identity-cond} can be solved assuming that $m_0$ is real.
%It is not necessary that the corredponding $\widetilde{m_0}$ is also real, but in this testing case, $m_0$ and $\widetilde{m_0}$ are both real.
%have the same phase, hence the phase-shifted $\m{0}$ is real as well. 
Figure \ref{fig: m-funcs} shows the ground truth $m_0$ and $\widetilde{m_0}$ considered in this simulation. %and in particular, $|m_0|$ is used as the known coefficients in \eqref{eq: bi-orth-eq}. Hereafter, we use $m_0(\omega)$ and $\m{0}$ to denote the real-valued functions.
\begin{figure}%{l}{.4\textwidth}
\begin{minipage}[t]{.45\textwidth}
\includegraphics[width = \linewidth]{m-funcs.jpg}
\caption{$m_0(\omega)$ and $\widetilde{m_0}(\omega)$}
\label{fig: m-funcs}
\end{minipage}
\hfill
\begin{minipage}[t]{.45\textwidth}
\vbox{
\includegraphics[width = \textwidth]{1d-m-compare.jpg}\\
\includegraphics[width = \textwidth]{1d-filter-compare.jpg}
}
\caption{$\widehat{\widetilde{m_0}}$ vs. $\widetilde{m_0}$, top: frequency domain, bottom: time domain}
\label{fig: 1d-compare}
\end{minipage}
\end{figure}

Let $\mhat{0}$ be the approximation of $\m{0}$, which is solution of the following optimization problem
\begin{align}
\min_{\xvec}\; \Vert \V{D}\xvec\Vert^2 + \Vert \xvec\Vert^2,\quad s.t. \; \V{A}\xvec = \mathbf{1} \label{eq: opt-1d}
\end{align}
where $\V{A}$ in the constraint is the matrix generated from \eqref{eq: identity-cond} (in 1D, only a single shift of $\pi$ appears in the condition, so each row of $\V{A}$ has two non-zero entries). 
%Notice that no symmetry constraint is imposed here, nevertheless, 
Figure \ref{fig: 1d-compare} compares the solution of \eqref{eq: opt-1d} and the ground truth. The support of the solution is slightly more spread out than the ground truth.
%The support of the solution shown in Fig.\ref{fig: 1d-compare} is almost symmetric. On the other hand, its support in the time domain is not as compact as that of $\m{0}$, see the bottom of Fig.\ref{fig: 1d-compare}.

\subsection{Numerical optimization of $\m{0}$ in 2D}
In the 2D case, we use the pair of biorthogonal low-pass filters that are the tensor products of the 1D filters in Section \ref{subsec: 1D-opt} as ground truth. We solve the 2D version of the optimization problem \eqref{eq: opt-1d}. Figure \ref{fig: 2d-compare-1} shows the solution and compares it with the ground truth. 

%{\it 2D version of \eqref{eq: opt-1d}}\\

%The 1D formulation can be easily extended to 2D, where $\V{D} = [\V{D}_x,\V{D}_y]$ consider 1st order derivative in both $x$ and $y$ directions, and $\V{A}$ is generated from \eqref{eq: identity-cond}, each row has four non-zero entries. 
%It is obvious that the solution is not $90^\circ$-rotation invariant. Even worse is the fact that there is much energy in the vertical high-frequency domain.

%{\it weighted L2 norm (Modulation Space$^{[\ref{app: modulation}]}$)}\\
To make the support of $\mhat{0}$ better concentrate within the low frequency domain, we change the squared $\ell_2$-norm penalty in \eqref{eq: opt-1d} to a weighted version (corresponding to Modulation space) as follows,
\begin{align}
\min_{\xvec}\; \Vert\V{ D}\xvec\Vert^2 + \lambda\Vert \wvec\circ\xvec\Vert^2,\quad s.t. \; \V{A}\xvec = \mathbf{1} \label{eq: opt-2d-weight}
\end{align} 
where $\circ$ is Hadamard product and $\wvec$ is a weight vector. In particular, we choose $\forall \V{\omega}, \; \wvec(\V{\omega}) = |\V{\omega}|$. Figure \ref{fig: 2d-compare-2.2} shows the solution of \eqref{eq: opt-2d-weight} with $\lambda=600$. % and $600$ respectively. As $\lambda$ increases, the support of the minimizer concentrates more within the low frequency region. As shown in Fig.\ref{fig: 2d-compare-2}, when $\lambda$ is not huge, the minimizer achieves a certain level of but not full symmetry, whereas Fig.\ref{fig: 2d-compare-2.2} shows that huge $\lambda$ imposes full symmetry.

Compared to \eqref{eq: opt} proposed to solve $\m{0}$, both optimization problems \eqref{eq: opt-1d} and \eqref{eq: opt-2d-weight} in this simulation minimize the squared $\ell_2$-norm of the gradient of $\widetilde{m_0}$ but have an extra (weighted) $\ell_2$ regularization term. Although \eqref{eq: opt-1d} and \eqref{eq: opt-2d-weight} work better than \eqref{eq: opt} for 1D and 2D tensor wavelet construction here, they do not provide solutions with better regularity in the construction of biorthogonal directional wavelets while increasing the computation cost.

\begin{comment}
\begin{minipage}{.9\textwidth}
\centering
\includegraphics[width = .9\textwidth]{2d-m-compare-2-1-eps-converted-to.pdf}\\
\includegraphics[width = .9\textwidth]{2d-filter-compare-2-1-eps-converted-to.pdf}
\captionof{figure}{result of \eqref{eq: opt-2d-weight} $\mhat{0}$ ($\lambda = 60$), target $\m{0}$ and their difference, Top: frequency domain, Bottom: time domain}
\label{fig: 2d-compare-2}
\end{minipage}
\end{comment}

\begin{minipage}{.9\textwidth}
\centering
\includegraphics[width = \textwidth]{2d-m-compare-vanilla.png}
\captionof{figure}{Left to right: solution of \eqref{eq: opt-1d} in 2D, ground truth and their difference}
\label{fig: 2d-compare-1}
\end{minipage}\\
\vspace*{2em}
\begin{minipage}{.9\textwidth}
\centering
\includegraphics[width = \textwidth]{2d-m-compare-weightedl2.png}\\
\includegraphics[width = \textwidth]{2d-filter-compare-weightedl2.png}
\captionof{figure}{Left to right: solution of \eqref{eq: opt-2d-weight} ($\lambda = 600$), ground truth and their difference; Top: frequency domain, bottom: time domain.}
\label{fig: 2d-compare-2.2}
\end{minipage}
\\[1em]
\begin{comment}
{\it weighted L2 norm with symmetry constraint}\\
If we hard constrain the symmetry by the following
\begin{align}
\min_{\xvec}\; \Vert \V{D}\xvec\Vert^2 + \lambda\Vert \wvec\circ\xvec\Vert^2,\quad s.t. \; \V{A}\xvec = \mathbf{1},\,\V{S}\xvec = \mathbf{0} \label{eq: opt-2d-weight-sym}
\end{align}
where each row of $\V{S}$ has an one entry and a negative one entry at the location of two points have the same value due to symmetry. In practice, we put symmetry constraints such that the upper half plane is symmetric to the lower half plane w.r.t. $x$ coordinate and the first quadrant is $90^{\circ}-$ rotational invariant w.r.t. the second quadrant. The symmetry constraint makes the optimization problem significantly harder, resulting in longer optimization algorithm running time and no near-optimal solution is found (the algorithm terminates as the maximum number of iterations is exceeded). Figure \ref{fig: 2d-compare-3} shows the result provided by the Matlab quadratic minimization solver, unfortunately, there are artifacts at the near endpoints of $x$ and $y$ coordinates.
\begin{figure}[h]
\includegraphics[width = .9\textwidth]{2d-m-compare-3-eps-converted-to.pdf}\\
\includegraphics[width = .9\textwidth]{2d-m-compare-3-2-eps-converted-to.pdf}
\caption{solution of \eqref{eq: opt-2d-weight-sym} (top: $\lambda=60$, bottom: $\lambda=600$) provided by Matlab solver \texttt{quadprog}}
\label{fig: 2d-compare-3}
\end{figure}
\\
On the other hand, asymmetric solution can always be symmetrized by the average of the solution and its dual
w.r.t. rotation, mirroring, etc. This approach increase the support of the solution, thus a well concentrated solution in the frequency domain is necessary to begin with.
\\[1em]
{\it Other potential formulations}\\
We may also putting weights in the first L2-norm of derivatives, such that
\begin{align}
\min_{\xvec}\; \Vert \wvec'\circ \V{D}\xvec\Vert^2 + \lambda\Vert \wvec\circ\xvec\Vert^2,\quad s.t. \; \V{A}\xvec = \mathbf{1} \label{eq: opt-2d-double-weight}
\end{align}
Clearly, $\wvec'(\V{\omega})\rightarrow +\infty$ as $|\V{\omega}|\rightarrow +\infty$, but its behavior near the origin is unclear.
\end{comment}
\end{appendices}

\bibliographystyle{IEEEtran}%bib/te}
\bibliography{ref}
\end{document}