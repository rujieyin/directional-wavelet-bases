\section{Low-redundancy frame construction}\label{sec: frame}
Consider the $L-$level directional wavelet MRA system
 \begin{align}\label{eq: MRA-frame}
 \{\phi_{L,\boldsymbol{k}}\,,\psi^j_{l,\boldsymbol{k'}}\,, \, {\small 1\leq l \leq L,\, \boldsymbol{k},\,\boldsymbol{k'}\in \mathbb{Z}^2,\,1\leq j \leq J\}}.
\end{align}  
where $\phi,\psi^j$ satisfy \eqref{eq: m0} and \eqref{eq: mj} as before. Instead of taking the dilated quincunx subsampling of directional wavelet coefficients of  \eqref{eq: MRA}, a dyadic subsampling is taken instead. A 1-level MRA frame \eqref{eq: MRA-frame} has redundancy $\frac{1}{|D|} + \frac{J}{|D|} = 1/4 + 6/4 = 7/4$, and the redundancy for any $L-$level MRA frame doesn't exceed $\frac{J/|D|}{1-1/|D|} = \frac{6/4}{1-1/4} = 2$. 
We have now
\begin{thm}\label{thm: frame-conds}
%Set $\Gamma = (D\mathbb{Z}^2)^*/(\mathbb{Z}^2)^*.$ 
The perfect reconstruction condition holds for \eqref{eq: MRA-frame} iff the following both hold
\begin{align}
\textstyle |m_0(\boldsymbol{\omega})|^2 + \sum_{j = 1}^6|m_j(\boldsymbol{\omega})|^2 &= 1 \\
\textstyle\sum_{j = 0}^6\,m_j(\boldsymbol{\omega})\overline{m_j(\boldsymbol{\omega} + \boldsymbol{\pi})} &= 0,\quad  \boldsymbol{\pi}\in \Gamma_0\setminus\{\boldsymbol{0}\} \label{eq: reduced-shift-cancel}
\end{align}
\end{thm}
Thmeorem \ref{thm: frame-conds} can be proved analogously to Thmeorem \ref{thm: conds}, 
with fewer shift cancellation constraints now. 
We can define {\it singular} boundaries as before, %and Lemma \ref{lem: singular-bdy} holds for \eqref{eq: reduced-shift-cancel} as well.
but only $\{\mathcal{B}(j,\boldsymbol{\pi})\}_{\boldsymbol{\pi}\in\Gamma_0\setminus\{\boldsymbol{0}\}}$ need to be considered, which results in fewer singular boundaries $\{\mathcal{C}_s(j,\boldsymbol{\pi})\}_{\boldsymbol{\pi}\in\Gamma\setminus\{\boldsymbol{0}\}}$; 
% In particular, we construct a directional wavelet tight frame with redundancy of 2 by using the classical dyadic downsampling $D_2$, with shift cancellaiton constraints \eqref{eq: shift-cancel} only on set $\Gamma\setminus\{\boldsymbol{0}\}$. 
%We check that within these singular boundaries, 
and no "double" singular boundaries now.

This means that even though $supp(m_0)$ still cannot be extended outside of the four corners of $S_1$ due to $\mathcal{C}_s(0,(\pi,0))$ and $\mathcal{C}_s(0,(0,\pi))$, $m_1$ can penetrate into the inside of $S_1$ because $\mathcal{C}_s(1,(\pi/2,3\pi/2))$ is not a singular boundary in \eqref{eq: MRA-frame}. The same is true for $m_3,m_4$ and $m_6$. This makes smoothing the boundaries of $m_0$ inwards possible without violating \eqref{eq: id-sum}, see Fig. \ref{fig: many-squares}(c). At the price of double redundancy, we obtain directional wavelets with much better spatial localization; see Fig. \ref{fig: many-squares}(d)(e):
the discontinuities of a directional wavelets basis in the frequency domain around the singular boundaries can be removed in a low redundant directional wavelet tight frame.

Heretofore, we have considered two directional wavelet MRA systems \eqref{eq: MRA} and \eqref{eq: MRA-frame} such that the directional wavelets characterize 2D signals in six equi-angled directions. 
The orthonormal basis we construct has better frequency localization than the one constructed by Durand in \cite{durand2007} ( see Fig. \ref{fig: design} and \ref{fig: many-squares}(b)(c)), but has long tails in certain spatial directions, unavoidable because of "double" singular boundaries. 
By doubling the redundancy we obtain spatially well localized directional wavelets.
Furthermore, these wavelets are well localized in the frequency domain such that $supp(m_j)$ is convex and $\exists\epsilon\, s.t.$
\begin{align}\label{eq: no-alians}
 \sup_{\boldsymbol{\omega}'\in supp(m_j)}\inf_{\boldsymbol{\omega}\in C_j}\Vert\boldsymbol{\omega'} - \boldsymbol{\omega}\Vert < \epsilon,\quad  0\leq j\leq 6.
\end{align}
This desirable condition is hard to obtain by multi-directional filter bank assembly of several elementary filter banks.

In the next section, we analyze the more general case of directional bi-orthorgonal filters constructed with respect to the same frequency partition. 