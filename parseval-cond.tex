\section{Proof of Theorem \ref{thm: conds}}\label{app: cond-thm}
Take the Fourier transform of both sides of \eqref{eq: PR}, we have 
\begin{align*}
\sum_{\V{k}}\langle f,\phi_{\V{k}}\rangle\hat{\phi}(\V{\omega})e^{-i\V{\omega}^\top\V{k}} = \sum_{\V{k}}&\langle f,\phi_{1,\V{k}}\rangle e^{-i\V{\omega}^\top\V{Dk}}|\V{D}|^{1/2}\hat{\phi}(\V{D}^T\V{\omega}) \\
&+ \sum_{j=1}^J\sum_{\V{k}}\langle f,\psi^j_{1,\V{k}}\rangle e^{-i\V{\omega}^\top\V{QDk}}|\V{QD}|^{1/2}\hat{\phi}(\V{D}^\top\V{\omega}).
\end{align*}
We use $\sum_{\V{k}}$ for summation over $\mathbb{Z}^2$ without specifying the set $\mathbb{Z}^2$.
Suppose $m_j$ are trigonometric series
\begin{align}\label{eq: mra1}
m_0(\V{\omega}) = \sum_{\V{k}} c_{\V{k}}e^{-i \V{\omega}^\top\V{k}} \quad
m_j(\V{\omega}) = \sum_{\V{k}} g_{\V{k}}e^{-i \V{\omega}^\top\V{k}},\quad j=1,\cdots,J.
\end{align}
The first term on the right hand side can be represented by $\hat{\phi}(\V{\omega})$ and $\langle f,\phi_k\rangle$ using \eqref{eq: m0} and \eqref{eq: mra1}.

\begin{align*}
\text{the first term on R.H.S. } = \sum_{\V{k}}\langle f,\phi_{1,\V{k}}\rangle e^{-i\V{\omega}^\top\V{Dk}}|\V{D}|^{1/2}m_0(\V{\omega})\hat{\phi}(\V{\omega}) \\= \sum_{\V{k}}\Big(\sum_{\V{k}'}\langle f,\phi_{\V{k}'}\rangle\overlinespace{c_{\V{k'-Dk}}}|\V{D}|^{1/2}\Big)e^{-i\V{\omega}^\top\V{Dk}}|\V{D}|^{1/2}m_0(\V{\omega})\hat{\phi}(\V{\omega})\\
=\sum_{\V{k}'}\langle f,\phi_{\V{k}'}\rangle\Big(|\V{D}|\sum_{\V{k}}\overlinespace{c_{\V{k'-Dk}}}e^{i\V{\omega}^\top(\V{k'-Dk})}\Big)e^{-i\V{\omega} ^\top\V{k}'} m_0(\V{\omega})\hat{\phi}(\V{\omega}).
\end{align*}

Let $\{\V{\beta}\}\doteq \V{D}\mathbb{Z}^2+\V{\beta}$ for $\V{\beta}\in B, \,s.t.\,\bigcup_{\V{\beta}\in B} \{\V{\beta}\} = \mathbb{Z}^2$.\footnote{The choice of $B$ is not unique and one choice is $ \{ (0,0),\,(1,0),\,(0,1),\,(1,1)\}$.}  The sum over $\mathbb{Z}^2$ can then be written as a double sum $\sum_{\V{\beta}\in B}\sum_{\V{k}'\in \{\V{\beta}\}}$,
\begin{align*}
\sum_{\V{\beta}\in B}\sum_{\V{k}'\in\{\V{\beta}\}} \langle f,\phi_{\V{k}'}\rangle\sum_{\V{k}}\overlinespace{c_{\V{k'-Dk}}}e^{i\V{\omega}^\top(\V{k}'-\V{Dk})}e^{-i\V{\omega}^\top\V{k}'}|\V{D}|m_0(\V{\omega})\hat{\phi}(\V{\omega})\\
=\sum_{\V{\beta}\in B}\sum_{\V{k}'\in\{\V{\beta}\}} \langle f,\phi_{\V{k}'}\rangle\Big(\,\sum_{\V{k}\in\{\V{\beta}\}}\overlinespace{c_{\V{k}}}e^{i\V{\omega}^\top\V{k}}\,\Big)e^{-i\V{\omega}^\top\V{k}'}|\V{D}|m_0(\V{\omega})\hat{\phi}(\V{\omega}).
\end{align*}
Due to the identity $\sum_{\V{\pi}\in\Gamma_0}e^{i\V{\beta}^\top\V{\pi}} = |\Gamma_0| \, \scalebox{1.3}{$\chi$}_{\V{D}\mathbb{Z}^2}(\V{\beta})$, the sum $\sum_{\V{k}\in \{\V{\beta}\}}c_{\V{k}}e^{-i\V{\omega}^\top\V{k}}$ equals to a linear combination of  $m_0$ with shifts in $\Gamma_0$,
%, similar to the trigonometric expansion of $m_0$ in \eqref{eq: mra1}, but $\V{k}$ takes value on the shifted sub-lattice $\{\V{\beta}\}$ instead of $\mathbb{Z}^2$, equals to instead a linear combination of $m_0$ with shifts $\Gamma_0$,
\begin{align}\label{eq:eq1}
\sum_{\V{k}\in \{\V{\beta}\}}c_{\V{k}}e^{-i\V{\omega}^\top\V{k}}
= \frac{1}{|\Gamma_0|}\;\sum_{\V{\pi}\in\Gamma_0}m_0(\V{\omega}+\V{\pi})\;e^{i\V{\beta}^\top\V{\pi}} .
\end{align}
Substitute \eqref{eq:eq1} into the previous expression and notice $|\Gamma_0| = |D|=4$, we have
\begin{align*}
\sum_{\V{\beta}\in B}\sum_{\V{k}'\in \{\V{\beta}\}}\langle f,\phi_{\V{k}'}\rangle\sum_{\V{\pi}\in\Gamma_0}\overlinespace{m_0}(\V{\omega}+\V{\pi})\;e^{-i\V{\beta}^\top\V{\pi}}\,e^{-i\V{\omega}^\top\V{k}'}m_0(\V{\omega})\hat{\phi}(\V{\omega}).
\end{align*}
Since $e^{i \V{\pi}^\top\V{\beta}}=e^{i\V{\pi}^\top\V{k}'},$ for $\V{k}'\in \{\V{\beta}\} $, we can rewrite the double sum $\sum_{\V{\beta}\in B}\sum_{\V{k}'\in \{\V{\beta}\}} $  back to a unit sum over $\mathbb{Z}^2$ as follows.
\begin{align*}
\sum_{\V{k}'}\langle f,\phi_{\V{k}'}\rangle e^{-i\V{\omega} ^\top\V{k}'}\hat{\phi}(\V{\omega})\Big(\sum_{\V{\pi}\in\Gamma_0}\overlinespace{m_0}(\V{\omega}+\V{\pi})m_0(\V{\omega})e^{-i\V{\pi}^\top\V{k}'} \Big).
\end{align*}

Similarly, the second term on the R.H.S. of \eqref{eq: PR} equals to 
\begin{align*}
\sum_{j=1}^J\sum_{\V{k}'}\langle f,\phi_{\V{k}'}\rangle e^{-i\V{\omega}^\top \V{k}'}\hat{\phi}(\V{\omega})\Big(\sum_{\V{\pi}\in\Gamma_1} \overlinespace{m_j}(\V{\omega}+\V{\pi})m_j(\V{\omega})e^{-i\V{\pi}^\top\V{k}'} \Big)
\end{align*}
based on the following equality analogous to \eqref{eq:eq1}
\begin{align}
\sum_{\V{k}\in\{\V{\alpha}\}} g_{\V{k}'} e^{-i\V{\omega}^\top\V{k}} = \frac{1}{|\Gamma_1|}\, \sum_{\V{\pi}\in\Gamma_1} m_j(\V{\omega} + \V{\pi})e^{i\V{\alpha}^\top\V{\pi}},
\end{align}
where $\{\V{\alpha}\} \doteq \V{QD}\mathbb{Z}^2 + \V{\alpha}$ for $\V{\alpha}\in A,\, s.t.\, \bigcup_{\V{\alpha}\in A} \{\V{\alpha}\} = \mathbb{Z}^2$.
(For Theorem \ref{thm: frame-conds} on frame construction, the summation of shifts $\V{\pi}$ is over $\Gamma_0$ instead of $\Gamma_1$.) 
Combining the two terms on the R.H.S. of \eqref{eq: PR}, and compare the coefficients of $\langle f,\phi_{\V{k}'}\rangle e^{-i\V{\omega}^\top \V{k}'}\hat{\phi}(\V{\omega})$ on both sides, the perfect reconstruction condition is then equivalent to for all $\V{k}'$,
\begin{align*}
\sum_{\V{\pi}\in\Gamma_0}e^{-i\V{\pi}^\top\V{k}'}\overlinespace{m_0}(\V{\omega}+\V{\pi})m_0(\V{\omega}) + \sum_j\sum_{\V{\pi}\in\Gamma_1} e^{-i\V{\pi}^\top\V{k}'}\overlinespace{m_j}(\V{\omega}+\V{\pi})m_j(\V{\omega}) = 1. 
%\sum_{l=0}^{3}e^{-i\gamma_l^\top(k'-k_0)}\overline{M_0(\xi+\gamma_l)}M_0(\xi) + \sum_j\sum_{s=0}^7 e^{-i\nu_s^\top(k'-k_j)}\overline{M_j(\xi+\nu_s)}M_j(\xi) = 1. 
\end{align*} 
This is equivalent to 
\begin{align*}
&|m_0(\V{\omega})|^2 + \sum_j|m_j(\V{\omega})|^2 = 1
\end{align*}
and
\begin{align*}
\sum_{j=0}^J\overlinespace{m_j}(\V{\omega}+\V{\pi})m_j(\V{\omega}) = 0, 
%+ \overline{m_0(\V{\omega}+\V{\pi})}m_0(\V{\omega}) = 0, 
\,\V{\pi}\in \Gamma_0\setminus\{\V{0}\}\\
\sum_{j=1}^J\overlinespace{m_j}(\V{\omega}+\V{\pi})m_j(\V{\omega}) = 0,\,\V{\pi}\in \Gamma_1\setminus \Gamma_0
\end{align*}
\qed

\noindent{\it Remark}.
If we have a shift $\V{k}_0$ in the down-sample scheme for $\phi_1$, i.e. $\V{D}\mathbb{Z}^2 - \V{k}_0$ instead of $\V{D}\mathbb{Z}^2$, so that we obtain coefficient of $\tilde{\phi}_{1,\V{k}} = \phi_{1,\V{k}+\V{k}_0}$ instead of $\phi_{1,\V{k}}$, and $\tilde{\phi}_1(\V{x}) =\phi_1(\V{x}-\V{k}_0)= |\V{D}|^{1/2}\sum_{\V{k}}c_{\V{k}}\phi(\V{x-k-k}_0) = |\V{D}|^{1/2}\sum_{\V{k}}c_{\V{k}-\V{k}_0}\phi(\V{x-k})$. This change of down-sample scheme results in an extra phase term $e^{-i\V{\omega}^\top \V{k}_0}$ in $m_0$. 
Similarly, if we downsample $\psi_1^j$ on a shifted sub-lattice $\V{QD}\mathbb{Z}^2-\V{k}_j$, we then have an extra phase $e^{i\V{\pi}^\top\V{k}_j}$ before $\overlinespace{m_j}(\V{\omega}+\V{\pi})m_j(\V{\omega})$ in shift cancellation condition. This provides additional freedom in the construction yet it is not substantial. Here, we use the down-sample scheme without translation.
