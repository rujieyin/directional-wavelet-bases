\section{Conclusion and future work}\label{sec: end}
In this paper, we consider directional wavelet schemes on a dyadic quincunx sub-lattice and analyze their regularity. We show that filters in bi-orthogonal bases have the same discontinuity in the frequency domain as the orthonormal bases at the corners of $C_0 = [-\pi/2,\pi/2)\times[-\pi/2,\pi/2)$. 

%Our analysis is closely related to our proposed bases construction algorithms, 
%and we show that the construction method of orthonormal bases can be easily extended to build frames construction of redundancy 2, which achieve much better time frequency localization and thus practically useful.
 We propose a different approach to construct biorthogonal wavelets from our previous approach for the orthonormal bases construction \cite{yin2014orthshear}. The directional dual filters $\widetilde{m_j}$ are first designed such that they can be extended to a bi-orthogonal frame and the remaining filters are obtained by solving linear systems and a constrained quadratic optimization derived from the identity summation and shift cancellation conditions for a biorthogonal MRA. We show numerically that regularized dual wavelets $\widetilde{\psi^j}$ can be constructed, yet their corresponding wavelets $\psi^j$ are still discontinuous in frequency domain, which is unavoidable according to our analysis.

We have looked at extensions of orthonormal bases in two different directions: tight frames (which are self-dual but redundant) with low redundancy and bi-orthogonal bases (which remain non-redundant but are no longer self-dual). In both cases we can gain some regularity. The extension of the biorthogonal bases to low-redundancy dual frame construction is not studied here, achieve at least the same regularity as low-redundancy tight frames, but with more flexibility in the construction. We leave this further generalization to future work.