\section{Joint optimization of $c(\V{\omega})$ and $\m{0}$}\label{app: QCQP}

In \ref{alg}, $c(\V{\omega})$ is chosen in step 3. to construct $m_0'(\V{\omega})$, which replaces $m_0(\V{\omega})$ and is used to create the linear constraint in \eqref{eq: opt} in step 4. Since different $c(\V{\omega})$ correspond to different $m_0'(\V{\omega})$, hence different linear constraints \eqref{eq: m0-A} on $\m{0}$; $\m{0}$ obtained in step 4. is optimal with respect to the pre-fixed $c(\V{\omega})$ from step 3., but not necessarily global optimal considering all possible choices of $c(\V{\omega})$. Therefore, we propose an alternative approach that combines step 3. and step 4. in \ref{alg}, where $c(\V{\omega})$ and $\m{0}$ are jointly optimized to obtain $\m{0}$ with the best possible regularity given unregularized $m_0(\V{\omega})$ from step 2.

By the definition in Proposition \ref{prop: mc}, $m_0'(\V{\omega}) = m_0(\V{\omega})c(\V{\omega})$. Furthermore, since $c(\V{\omega})$ is $\pi$-periodic in both $\omega_1,\,\omega_2$, we have $m_0'(\V{\omega}+\V{\pi}_i) = m_0(\V{\omega}+\V{\pi}_i)c(\V{\omega}),\, i = 2,4,6$. Hence the constraint \eqref{eq: identity-cond} on $\m{0}$ with $m_0(\V{\omega})$ replaced by $m_0'(\V{\omega})$ can be reformulated as follows,
\begin{align}
1 & = m_0'\m{0} + m_0'\mp{0}{2} + m_0'\mp{0}{4} + m_0'\mp{0}{6}\notag\\
& = c(\V{\omega})\big(\,m_0\m{0} + m_0\mp{0}{2} + m_0\mp{0}{4} + m_0\mp{0}{6}\,\big).\label{eq: linear-cond_c}
\end{align}
Using the same setup of the optimization \eqref{eq: opt}, we convert \eqref{eq: linear-cond_c} to a constraint on a $2N\times 2N$ grid $\mathcal{G}=\{\V{\omega}_i\}_{i=1}^{4N^2}$ of $[-\pi,\pi)\times[-\pi,\pi)$.
Let $\widetilde{\mathbf{m}_0}\in\mathbb{C}^{4N^2}$ and $\V{A}_0\in\mathbb{C}^{N^2\times 4N^2}$ be the same as in \eqref{eq: m0-A} except that $\V{A}_0$ is constructed by unregularized $m_0$ instead of $m_0'$ for $\V{A}$.
Let $\V{C}\in\mathbb{C}^{N^2\times N^2}$ be a diagonal matrix whose $j$-th diagonal entry is $c(\V{\omega}_j)$, where $\V{\omega}_j\in \mathcal{G}\cap [-\pi,0)\times[-\pi,0)$ in the same order as the rows of $\V{A}_0$. Then \eqref{eq: linear-cond_c} is equivalent to the following constraint on the grid $\mathcal{G}$,
\begin{align}
\V{C}\V{A}_0\,\overlinespace{\widetilde{\mathbf{m}_0}} = \V{1}_{N^2}.
\end{align}
We formulate the joint optimization on $\V{C}$ and $\widetilde{\mathbf{m}_0}$ analogous to \eqref{eq: opt} as follows,
\begin{align}\label{eq: opt-C}
\min_{\xvec\in\mathbb{C}^{4N^2},\;\mathbf{c}\in\mathbb{C}^{N^2}}\; \Vert \V{D}\xvec\Vert^2,\quad 
s.t. \; \V{C}\V{A}_0\,\xvec = \mathbf{1}, \; \V{C} = diag(\mathbf{c}).
\end{align}
Since the objective function does not involve $\mathbf{c}$, $\mathbf{c}$ can be expressed in terms of $\xvec$ as long as $\V{A}_0\,\xvec$ has no zero entry. Therefore, solving \eqref{eq: linear-cond_c} is equivalent to solving the following optimization for $\widetilde{\mathbf{m}_0}$.
\begin{align}\label{eq: opt-ineq}
\min_{\xvec\in\mathbb{C}^{4N^2}}\; \Vert \V{D}\xvec\Vert^2,\quad 
s.t. \; |\V{A}_0\,\xvec| > 0,
\end{align}
where $|\cdot|$ in the constraint is a pointwise operator that computes the absolute value. The constraint $|\V{A}_0\,\xvec| > 0$ can be rewritten as a set of quadratic constraints $\xvec^*\V{Q}_i\xvec > 0,\, i =0,\cdots, N^2-1 $ where $\V{Q}_i = \V{A}_0[i,:]^*\V{A}_0[i,:]$. Therefore, \eqref{eq: opt-ineq} is a quadratically constrained quadratic program. Furthermore, since $\V{Q}_i$ is positive semi-definite, \eqref{eq: opt-ineq} is not convex and is NP-hard in general. One may solve the convex relaxation of \eqref{eq: opt-ineq} using semidefinite programming (SDP). Instead of solving $\xvec$, we solve $\V{X}\doteq \xvec\xvec^*$ and convert \eqref{eq: opt-ineq} into 
\begin{align}\label{eq: opt-trace}
\min_{\V{X}\in\mathbb{C}^{4N^2\times 4N^2}}\;tr(\, \V{D}^*\V{D}\,\V{X}\,),\quad
s.t. \; tr(\,\V{Q}_i\,\V{X}\,) > 0,\, \V{X}\succeq 0,\, rank(\V{X}) = 1,
\end{align}
where $\V{X}\succeq 0$ is the positive semidefinite constraint on $\V{X}$.
By removing the non-convex rank constraint $rank(\V{X}) = 1$, \eqref{eq: opt-trace} becomes a SDP and can be efficiently solved. Yet the solution $\V{X}$ may not be rank 1 and require post processing (e.g. singular value decomposition) to obtain an approximate solution of \eqref{eq: opt-ineq}.